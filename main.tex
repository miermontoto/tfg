\documentclass[12pt, oneside, openany]{book}

\usepackage{mathptmx} % Contiene una fuente similar a Times New Roman

\usepackage[spanish, es-tabla]{babel} % Permite escritura en castellano
\usepackage[utf8]{inputenc} % Permite utilizar caracteres UTF8

\usepackage{graphicx} % Para la inclusión de gráficos e imágenes
\graphicspath{ {images/} } % Ruta para buscar las imágenes
\usepackage[a4paper,top=30mm,left=30mm,right=25mm,bottom=25mm,headheight=20mm]{geometry} % Configuración de los margenes de la página

\PassOptionsToPackage{table}{xcolor}

% Paquetes para que funcione el formato.
\usepackage{pgfplots}
\usepackage{xcolor}
\usepackage{plantuml}
\usepackage{tikz}
\usepackage[toc,page]{appendix}
\usepackage{titlesec}
\usepackage{multirow}
\usepackage{setspace}
\usepackage{ragged2e}
\usepackage{fancyhdr}
\usepackage{lastpage}
\usepackage{stackengine}
\usepackage{array}
\usepackage[hidelinks]{hyperref}
\usepackage{enumitem}
\usepackage{float}
\usepackage{hypcap}
\usepackage{caption}
\usepackage{fancyvrb}
\usepackage{textcomp}
\usepackage{listings}

\pgfplotsset{compat=1.18}
\usepgfplotslibrary{dateplot} % Se define un color gris desde su código RGB
\definecolor{gris}{RGB}{220,220,220}

\setcounter{secnumdepth}{3} % Para permitir numerar las sub-subsecciones

% Modifica el nombre de los índices al castellano
\addto\captionsspanish{
  \renewcommand{\contentsname}{Índice de contenido}
  \renewcommand{\listfigurename}{Índice de figuras}
  \renewcommand{\listtablename}{Índice de tablas}
  \renewcommand{\lstlistingname}{Listado}
  \renewcommand{\lstlistlistingname}{Índice de listados}
}

% Formateo de los nombres de los apartados:
\titleformat{\chapter}[block]
{\normalfont\Huge\bfseries\singlespacing}{\thechapter.}{1em}{\Huge}
\titlespacing*{\chapter}{0pt}{-62pt}{0pt}

\titleformat{\section}[block]
{\normalfont\Large\bfseries}{\thesection.}{4pt}{\Large}
\titlespacing*{\section}{0pt}{\baselineskip}{0pt}

\titleformat{\subsection}[block]
{\normalfont\large\bfseries}{\thesubsection.}{4pt}{\normalsize\large}
\titlespacing*{\subsection}{0pt}{0pt}{0pt}

\titleformat{\subsubsection}[block]
{\normalfont\normalsize\bfseries}{\thesubsubsection.}{4pt}{\normalsize}
\titlespacing*{\subsubsection}{0pt}{0pt}{0pt}

\def\tablename{Tabla}

%% Variables para portada y cabeceras
%% Cambiar los valores para cada documento!!!
\def\title{Explotación, integración y visualización de múltiples fuentes de datos mediante un Data Lake}
\def\shortTitle{ETL de datos heterogéneos mediante un Data Lake}
\def\subject{Lenguajes y Sistemas Informáticos}
\def\author{Mier Montoto, Juan Francisco}
\def\date{julio 2024}
\def\org{Escuela Politécnica de Ingeniería de Gijón}
\def\area{Grado en Ingeniería Informática en Tecnologías de la Información}
\def\tutorOne{D. Augusto Alonso, Cristian}
\def\tutorTwo{D. Morán Barbón, Jesús}
\def\tutorThree{D. Vázquez Faes, Eduardo}
\def\tfgId{???}
\def\dni{71777658V}

\def\ORG{\expandafter\MakeUppercase\expandafter{\org}}
\def\AREA{\expandafter\MakeUppercase\expandafter{\area}}
\def\SUBJECT{\expandafter\MakeUppercase\expandafter{\subject}}


\captionsetup{justification=centering}
\setlength{\headheight}{65pt}

\fancyhf{}
\fancyhead[L]{\includegraphics[height=8mm]{style/square.png}
  \hspace{3em} \Longstack[l] {
    \textbf{\SUBJECT} \newline
    \textbf{\shortTitle}}
  \newline \leftmark{}
}
\fancyhead[R]{\bfseries{Hoja \hyperlink{toc}{\thepage}~de~\pageref{LastPage}}}
\fancyfoot[C]{\author}
\renewcommand{\headrulewidth}{0pt} % default is 0pt
\renewcommand{\footrulewidth}{0.4pt} % default is 0

\fancypagestyle{plain}{%
  \fancyhf{}
  \fancyhead[L]{\includegraphics[height=8mm]{style/square.png}
    \hspace{3em} \Longstack[l]{
      \textbf{\SUBJECT} \newline
      \textbf{\shortTitle}}}
  \fancyhead[R]{\bfseries{Hoja \hyperlink{toc}{\thepage}~de~\pageref{LastPage}}}
  \fancyfoot[C]{\author}
  \renewcommand{\headrulewidth}{0pt} % default is 0pt
  \renewcommand{\footrulewidth}{0.4pt} % default is 0pt
}

\newcommand*{\fullref}[1]{\textit{\hyperref[{#1}]{\ref*{#1} \nameref*{#1}}}}
\newcommand*{\halfref}[2]{\texttt{\hyperref[{#1}]{#2}}}
\newcommand{\blankpage}{
  \newpage
  \thispagestyle{empty}
  \mbox{}
  \newpage
}

\pagestyle{fancy}

\restylefloat{table}
\definecolor{backcolour}{rgb}{0.95,0.95,0.92}
\definecolor{darkgreen}{rgb}{0.0, 0.5, 0.0}

\lstdefinestyle{default}{
  basicstyle=\ttfamily\footnotesize,
  breakatwhitespace=false,
  breaklines=true,
  captionpos=b,
  keepspaces=true,
  numbers=left,
  numbersep=5pt,
  numberstyle=\tiny\color{black},
  showspaces=false,
  showstringspaces=false,
  showtabs=false,
  tabsize=2,
  backgroundcolor=\color{backcolour},
  postbreak=\mbox{\textcolor{red}{$\hookrightarrow$}\space},
}

\lstset{style=default}

\lstdefinestyle{yaml}{
  basicstyle=\ttfamily\footnotesize\color{darkgreen}\bfseries,
  rulecolor=\color{black},
  string=[s]{'}{'},
  stringstyle=\color{darkgreen},
  comment=[l]{:},
  commentstyle=\color{blue},
  morecomment=[l]{-},
  numbers=left,
  numbersep=5pt,
  numberstyle=\tiny\color{black},
  backgroundcolor=\color{backcolour},
  breaklines=true,
  captionpos=b,
  keepspaces=true,
  showspaces=false,
  showstringspaces=false,
  showtabs=false,
  tabsize=2,
  breakatwhitespace=true,
  postbreak=\mbox{\textcolor{red}{$\hookrightarrow$}\space},
}

\pgfplotsset{compat=1.18}
\usepgfplotslibrary{dateplot}



\begin{document}

\rmfamily % Fuente tipo Romana
\begin{titlepage}
    \centering
    \bfseries {
        \null{}
        \vspace{0cm}
        \begin{table}[h]
            \centering
            \begin{tabular}{m{10cm} m{1cm} m{3cm}}
                \vspace{0.2cm}
                \includegraphics[width=70mm]{style/full.png} &
                \vspace{1.5mm} \includegraphics[width=25mm]{style/square.png} \\
            \end{tabular}
        \end{table}

        \vspace{3\baselineskip}

        \Large{\ORG{} \\ \vspace{3\baselineskip}}
        \large {
            \AREA{} \\ \vspace{3\baselineskip}
            \subject{} \\ \vspace{2\baselineskip}

            TRABAJO FIN DE GRADO/MÁSTER Nº \tfgId{} \vspace{\baselineskip} \\
            \title{} \\ \vspace{1\baselineskip}

            \author{} \\ \vspace{1\baselineskip}
            TUTORES:\\
            \tutorOne{} \\
            \tutorTwo{} \\
            \tutorThree{} \\ \vspace{\baselineskip}

            \vspace{2\baselineskip}
            FECHA:\@ \date{}
        }
    }
\end{titlepage}
 % Portada de la memoria

% \frontmatter

% Índice de contenido
\addcontentsline{toc}{chapter}{Índice de contenido} % Añade la referencia al índice de contenido
\hypertarget{toc}{}
\tableofcontents
\newpage{}

% Índice de figuras
\newpage{}
\addcontentsline{toc}{chapter}{Índice de figuras}  % Añade la referencia al índice de contenido
\hypertarget{lof}{}
\listoffigures

% Índice de tablas
\newpage{}
\addcontentsline{toc}{chapter}{Índice de tablas}  % Añade la referencia al índice de contenido
\hypertarget{lot}{}
\listoftables

% Índice de listings
\newpage{}
\addcontentsline{toc}{chapter}{Índice de listados}  % Añade la referencia al índice de contenido
\hypertarget{lol}{}
\lstlistoflistings

\justify{} % Texto justificado
\setlength{\parskip}{\baselineskip} % Separación entre párrafos de 1 linea
\onehalfspacing{}

\blankpage

% declaración de autoría
\chapter*{Declaración de autoría}\label{chap:declaracion}

\noindent

Yo, Juan Francisco Mier Montoto, con DNI \texttt{\dni{}}, declaro que este
documento titulado \textsc{\title{}} y el trabajo presentado en él son de mi
propiedad. Afirmo que:

\begin{itemize}
	\item Este trabajo fue realizado completa y totalmente durante mi estancia
		en el \area{} en la \org{}.
	\item Aquellas partes de este documento que hayan sido previamente
		publicadas se encuentran debidamente indicadas.
	\item Aquellas partes e este documento que se apoyen en trabajos previamente
		publicados se encuentran debidmente referenciadas.
	\item Todas las fuentes utilizadas para la realización de este documento
		han sido debidamente citadas.
	\item Durante el desarrollo de este trabajo, se han utilizado correctores
		generativos basados en inteligencia artificial (GPT-4o, Claude 3.5 Sonet)
		para la generación de texto, pero siempre bajo la supervisión y posterior
		revisión del autor. El autor toma la responsabilidad de cualquier error
		que pueda haber sido introducido durante el proceso de revisión.
\end{itemize}

\vfill

\begin{flushright}
	\begin{minipage}{0.5\textwidth}
		\begin{center}
			\textbf{Firmado:} Juan Francisco Mier Montoto
			\textbf{Fecha:} 18 de julio de 2024
		\end{center}
	\end{minipage}
\end{flushright}


\blankpage
% \mainmatter

%% El contenido de la memoria, dividido en capítulos:
\chapter{Introducción}\label{chap:intro}
El proyecto que se presenta en este documento tiene como objetivo la
automatización de despliegue de la infraestructura y procesos que permitan hacer
un análisis masivo de datos. Para conseguir este objetivo, se hace un
análisis del contexto y se realiza un diseño para, posteriormente, implementar
una solución que permita la integración, almacenamiento y análisis de grandes
volúmenes de datos, provenientes de múltiples fuentes y en diferentes formatos.


\section{Antecedentes}\label{sec:antecedentes}
Hoy en día, el crecimiento de la cantidad de dispositivos conectados a Internet
(teléfonos móviles, dispositivos \textit{IoT}\ldots) ha provocado un aumento
exponencial de la cantidad de datos que se manejan \footnote{
	\url{https://www.statista.com/statistics/871513/worldwide-data-created/}
}, un hecho que se ve reflejado en el ámbito empresarial. Dicha cantidad de
datos genera una necesidad de análisis y tratamiento que las tecnologías
tradicionales de datos (bases de datos) no pueden suplir. La diversidad de
fuentes y formatos de estos datos introduce una complejidad significativa en su
manejo, conocida como \textit{heterogeneidad} \footnote{
	\url{https://www.sciencedirect.com/topics/computer-science/data-heterogeneity}
}, siendo las bases de datos, archivos de registros y APIs las fuentes más
habituales.

El término \textit{big data} describe este fenómeno de acumulación masiva de
datos, cuya magnitud y complejidad sobrepasan las capacidades de los métodos de
procesamiento convencionales. El \textit{big data} se caracteriza por tres
características principales: volumen, variedad y velocidad - su adecuada gestión
y análisis pueden otorgar ventajas competitivas significativas a las empresas,
tales como el descubrimiento de patrones ocultos, identificación de nuevas
oportunidades de mercado y optimización de procesos de toma de decisiones.

Uno de los procesos que permite la extracción de esta información es la pirámide
DIKW, \cite{enwiki:1211227190} es un modelo que describe la relación entre los
datos, la información, el conocimiento y la sabiduría. Según este modelo, los
datos son la materia prima de la información, que a su vez es la materia prima
del conocimiento, que a su vez es la materia prima de la sabiduría. Una
organización sin los procesos adecuados para la gestión y análisis de estos
datos, se enfrenta a importantes desafíos, como la dificultad para identificar
patrones y tendencias, la toma de decisiones incorrectas y la pérdida de
oportunidades de negocio. Por otro lado, una organización que logre extraer
información valiosa de sus datos, podrá mejorar su eficiencia, aumentar su
competitividad y adaptarse mejor a un entorno empresarial en constante cambio.

La evolución tecnológica ha propiciado el desarrollo de innovadoras herramientas
y metodologías diseñadas para enfrentar estos desafíos. Entre ellas, los
\textit{data lakes} (o \emph{lagos de información}) se destacan por su capacidad
para consolidar vastos volúmenes de datos heterogéneos, facilitando su posterior
análisis y aprovechamiento de manera más efectiva.

Sin embargo, a pesar de que existen herramientas de almacenamiento, el proceso
de integración, visualización y análisis de estos datos es una tarea desafiante,
ya que requiere de una gran cantidad de recursos y de un tiempo de desarrollo
considerable del que, normalmente, no se dispone en el ámbito empresarial.

Con la ingesta masiva de datos, se presentan nuevos problemas a la hora de
analizar y obtener información de ellos:

\begin{itemize}
	\item \textbf{Grandes cantidades de información:}
		la masificación de información impide el análisis manual de los
		mismos, requiriendo resúmenes estadísticos o representaciones gráficas
		como \textit{dashboards} para su correcta interpretación.
		La visualización de datos es una técnica que permite representar la
		información de manera visual, para facilitar su análisis y comprensión,
		una parte vital del proceso de análisis de datos, ya que permite
		identificar patrones, tendencias y anomalías en los mismos de forma más
		rápida y sencilla.
	\item \textbf{Heterogeneidad de los datos:}
		la heterogeneidad de los datos, tanto en formato como en origen,
		dificulta su consolidación y análisis, ya que requiere de un proceso de
		integración y transformación previo para homogeneizarlos y poder
		analizarlos de forma conjunta.
	\item \textbf{Decisiones de negocio erróneas debidas a un mal tratamiento:}
		sin la necesaria automatización y correcta aplicación de los procesos
		ETL~(ver \ref{sec:etl}), los resultados del análisis pueden ser
		incorrectos, lo que deriva en errores y decisiones de negocio
		equivocadas que impactan negativamente en la empresa.
\end{itemize}

\newpage{}
\section{Motivación}\label{sec:motivacion}
Actualmente, las empresas (especialmente aquellas en el sector IT), se enfrentan
a la necesidad de unificar, gestionar y analizar grandes volúmenes de datos,
provenientes de múltiples fuentes y en diferentes formatos. La correcta gestión
y análisis de estos datos es fundamental para la toma de decisiones y para la
mejora de los procesos internos de la empresa.

En la actualidad, Okticket (en adelante la empresa) dispone de una gran cantidad
de datos que se encuentran en diferentes formatos y en diferentes ubicaciones,
lo que dificulta su análisis y explotación. Por otra parte, se depende de la
consulta manual o de servicios de terceros para poder analizar estos datos, lo
que supone un coste adicional.

El proyecto surge de la necesidad de la empresa de extraer información y
conocimiento de las múltiples y heterogéneas fuentes de datos de las que se
disponen, tanto internas (e.g.~bases de datos, archivos de registros, APIs,
entre otros), como externas (e.g. APIs o datos de webs de terceros, datos de
fuentes públicas\ldots).

Además del uso interno, la empresa también quiere ofrecer a sus clientes la
posibilidad de consultar estos datos de forma visual y sencilla, para que puedan
analizarlos y explotarlos de forma autónoma, lo que supondría un valor añadido
para los mismos.

\newpage{}
\section{La empresa}\label{sec:empresa}
Okticket es una startup nacida en Gijón en 2017 cuyo producto principal es un
servicio software que escanea automáticamente tickets y facilita su gestión usando conceptos
contables como notas de gastos, anticipos y más. Esto
permite reducir los costes y el tiempo que invierten las empresas en
contabilizar y manejar los gastos de viaje profesionales.

La empresa tienen su suede principal en el Parque Tecnológico de Gijón, aunque
cuenta con un número de sedes creciente en varios países, como Francia, Portugal
o, más recientemente, México. En esta oficina principal se encuentran los
departamentos de ventas y marketing, así como el equipo de desarrollo y consultoría.

Okticket es una de las empresas que más crecen tanto del sector como del propio
Parque Tecnológico. Debido a este rápido crecimiento, el equipo está en
constante desarrollo y cambio, tanto aquí en España como en el resto de sedes.
Este crecimiento se refleja en la recepción de un gran número de galardones y
reconocimientos.
\footnote{\href
	{https://www.linkedin.com/posts/okticket_okticket-en-el-especial-startups-de-forbes-activity-7140622980618903552-UGWK}
	{Okticket en el especial startups 2023 de Forbes (LinkedIn)}
}
\footnote{\href
	{https://www.elcomercio.es/economia/arcelor-okticket-premios-20230222002438-ntvo.html}
	{Arcelor y Okticket, premios nacional de Ingeniería Informática (EL COMERCIO)}
}
\footnote{\href{
	https://www.okticket.es/blog/empresa-pyme-innovadora}
	{Okticket recibe el sello Pyme Innovadora (okticket.es)}
}
\footnote{\href
	{https://www.okticket.es/blog/okticket-empresa-emergente-certificada}
	{Okticket, empresa emergente certificada (okticket.es)}
}

La parte principal del negocio es el núcleo del software como servicio (Software
as a Service en inglés, en adelante \textit{SaaS}), es decir, la aplicación
completa tanto para administradores como para empleados. Este SaaS se oferta a
empresas de cualquier tamaño, cuyo precio final varía en función del número de
usuarios, las características e integraciones que requiera la empresa cliente y
el soporte que se ofrezca.

Recientemente se han añadido nuevas propuestas a la cartera de servicios
ofertada por Okticket, como la OKTCard {-} una tarjeta inteligente que gestiona
automáticamente los gastos, así como la inclusión de nuevos ``módulos'' de
gestión de gastos y viajes.

Debido al crecimiento acelerado de Okticket, la empresa maneja una gran cantidad
de datos de diversos tipos y almacenados en diferentes silos (programas de
gestión contable, ventas, consultoría, así como los datos que genera el SaaS),
que deben ser unificados para poder ser analizados y explotados de forma eficiente.
Por otra parte, actualmente se depende de la consulta manual o de servicios de
terceros para poder analizar estos datos, lo que es costoso, tedioso y muy
poco eficiente.

\section{Objetivo y alcance}\label{sec:objetivos}
El objetivo del proyecto es la creación de un proceso que permita el despliegue
automático de una infraestructura para la integración, almacenamiento y
análisis de grandes volúmenes de datos, provenientes de múltiples fuentes y en
diferentes formatos. La infraestructura de datos debe ser escalable, flexible y
robusta, para poder adaptarse a las necesidades cambiantes de la empresa.

La integración de datos debe ser automática y programable, para poder
automatizar el proceso de ingestión de datos y reducir el tiempo y los costes
asociados.

El resultado final del proyecto será la plataforma en sí, es decir, la
infraestructura automatizada que integre y almacene los datos. El entregable
final será la colección de ficheros y \textit{scripts} necesarios para el
despliegue de la infraestructura y el tratamiento de la información.

La plataforma que se desarrolle debe de ser capaz, además de manejar los datos
comentados anteriormente, de ofrecer una interfaz visual de consulta y análisis
de los mismos, para que los usuarios puedan explotar la información de forma
sencilla y rápida.

\chapter{Fundamento teórico}\label{chap:teo}
En este capítulo se presentan los conceptos y términos fundamentales que se
utilizan en el proyecto para proporcionar una base teórica sobre la que se
desarrolle. Se discuten los conceptos fundamentales.


\section{\textit{Big data}}\label{sec:bigdata}
El término \textit{big data} se refiere a la gestión y análisis de grandes
volúmenes de datos que no pueden ser tratados de manera convencional. La
evolución natural del progreso tecnológico, la digitalización de la sociedad y
la aparición de nuevas tecnologías han propiciado la generación de grandes
cantidades de datos en todo el mundo, lo que genera la necesidad de nuevas
formas de gestionar y tratar estos datos.

El término \textit{big data} no solo se refiere a la cantidad de datos que se
generan, sino a también otras características, a las que se refieren como ``las
uves del big data''. La cantidad de \textit{uves} depende del autor y de la
fuente~\cite{ishwarappa2015vs,sagiroglu2013bigdata}, variando desde 3 hasta 7,
pero las más comunes son las siguientes:

\begin{itemize}
	\item \textbf{Volumen:} la cantidad de datos que se generan y almacenan en
		un determinado periodo de tiempo. El volumen de datos que se maneja en
		el \textit{big data} es mucho mayor que el que se maneja en los sistemas
		tradicionales de gestión de datos, que además se encuentra en aumento
		constante.
	\item \textbf{Variedad:} se refiere a la diversidad de fuentes y formatos de
		los datos que se manejan. Los datos pueden provenir de diversas fuentes,
		como bases de datos, sensores o registros, pueden estar en
		diferentes formatos o tener diferentes estructura. Se pueden clasificar
		de la siguiente manera: \begin{itemize}
			\item \textbf{Estructurados:} datos que se encuentran en un formato
				estructurado, como una base de datos relacional.
			\item \textbf{Semi-estructurados:} datos que no se encuentran en un
				formato estructurado, pero que tienen una estructura interna
				que permite su análisis, como un archivo XML o JSON.
			\item \textbf{No estructurados:} datos que no tienen una estructura
				definida, como un archivo de texto o una imagen.
		\end{itemize}
	\item \textbf{Velocidad:} se refiere a la frecuencia con la que se generan y
		se procesan los datos. En este ámbito, los datos se tratan a una
		velocidad mucho mayor que en los sistemas tradicionales de gestión de
		datos. Dicha velocidad puede ser crítica en ámbitos como la bolsa,
		donde la velocidad de procesamiento de los datos puede ser la
		diferencia entre obtener beneficios o pérdidas. Según la frecuencia de
		procesamiento de los datos, los sistemas se pueden clasificar en:
		\begin{itemize}
			\item \textbf{Batch (en lotes):} los datos se procesan en lotes, de
				manera periódica, como cada hora o cada día. Este tipo de datos,
				frecuente en aplicaciones que se tratan en Okticket como los
				notas de viajes o las nóminas, no requieren un procesamiento
				inmediato.
			\item \textbf{Streaming (en tiempo real):} los datos se procesan en
				tiempo real, a medida que se generan. Este tipo de datos, que se
				puede encontrar en aplicaciones como los sensores o los logs,
				requieren un procesamiento inmediato si se quiere obtener
				información relevante sobre los mismos.
			\item \textbf{Near real-time (casi en tiempo real):} los datos se
				procesan con un pequeño retraso, de manera que se obtiene
				información relevante sobre los mismos en un tiempo muy corto.
		\end{itemize}
	\item \textbf{Veracidad:} se refiere a la calidad de los datos que se
		manejan. La veracidad de los datos es un factor crítico en el ámbito del
		\textit{big data}, pues la calidad de la salida depende directamente de
		la calidad de la información de entrada. La veracidad de los datos se
		puede ver afectada por diferentes factores, como la calidad de los
		datos, la precisión de los mismos, la integridad de los datos, etc.
\end{itemize}


\newpage{}
\section{Paradigmas de almacenamiento de datos}\label{sec:paradigmas}
En el ámbito del \textit{big data}, existen diferentes paradigmas de
almacenamiento de datos que se utilizan para almacenar y analizar grandes
cantidades de información. Los tres paradigmas a considerar para este proyecto
son los \textit{data warehouses}, los \textit{data lakes} y los
\textit{data lakehouses}.


\subsection{Data warehouse}\label{sec:warehouse}
Un \textit{data warehouse}\footnote{
	\url{https://aws.amazon.com/es/data-warehouse/}
}, también conocido en español como almacén de datos, es una base de datos que
se utiliza para almacenar y analizar grandes cantidades de datos de manera
eficiente. Los almacenes de datos proporcionan acceso rápido y compatible con
plataformas de consultas (como SQL) a grandes cantidades de datos, lo que
permite a los analistas y a los científicos de datos realizar análisis complejos
sobre los datos almacenados.

Todos los datos almacenados en un \textit{data warehouse} se encuentran en un
formato común, para lo que se aplican procesos ETL (extracción, transformación y
carga) que transforman los datos de diferentes fuentes en un formato común. Esto
significa que la información se encuentra en un formato o esquema optimizado y
específico, lo que facilita su manipulación y análisis pero limita la
flexibilidad al acceso de los datos y genera costes adicionales en el caso de
tener que modificar o transferir los mismos para su uso.


\subsection{Data lake}\label{sec:lake}
Los \textit{data lakes}\footnote{
	\url{https://aws.amazon.com/es/what-is/data-lake/}
} son almacenes de datos que guardan grandes cantidades de datos de manera no
estructurada~\cite{mier2023dashboards}. En el ámbito de una empresa, un
\textit{data lake} contiene datos de diferentes fuentes de valor no considerado
hasta su análisis, de manera que su explotación posterior y su análisis no
depende de una estructuración y transformación compleja, reduciendo los costes
de los procesos ETL derivados, una flujo de tareas que se aplican sobre la
información para ingestarla. Esto no quiere decir que no se apliquen estos
procesos a los datos, sino que se aplican de manera más flexible y básica que en
otras estructuras de almacenamiento de datos con esquemas predefinidos, como los
\textit{data warehouses}.~\cite{pwint2018data}

A diferencia de los \textit{data warehouses}, los \textit{data lakes} no tienen
un esquema definido, lo que permite almacenar datos \textit{heterogéneos}. Esto
permite almacenar grandes cantidades de información sin tener que definir un
esquema de antemano, lo que puede ser útil en aquellos casos en los que no se
conoce la estructura de los datos que se van a almacenar.

Estas características de los \textit{data lakes} hacen que sean más atractivos
en el sector empresarial, puesto que implica la gestión de un solo \textit{stack}
tecnológico que contiene toda la información, en contraste con las estructuras
planteadas normalmente en el campo de la investigación académica.

Para consultar esta gran cantidad de datos almacenados, se suelen utilizar
técnicas de visualización de datos, como los \textit{dashboards}, herramientas
de visualización que permiten observar los datos de manera sencilla y eficiente.


\subsection{Data lakehouse}\label{sec:lakehouse}
Los \textit{data lakehouses} son una combinación funcional de los dos paradigmas
vistos anteriormente, los \textit{data lakes} y los \textit{data warehouses}.
Los \textit{data lakehouses} permiten almacenar datos tanto de manera
estructurada como no estructurada, lo que facilita aprovechar la información al
contar con una única estructura de bajo coste que ofrece a los usuarios que lo
necesiten explorar y analizar los datos según sus necesidades.


\newpage{}
\section{Procesos ETL}\label{sec:etl}
Si anteriormente se presentaban los distintos paradigmas de almacenamiento de
datos, para su creación y mantenimiento se requieren aplicar unos ciertos
procesos que permitan la correcta ingesta y almacenamiento de los datos. Estos
procesos se conocen como \textit{procesos ETL}.

Formalmente se definen los procesos ETL~\cite{mier2023dashboards} como procesos
que combinan datos de múltiples fuentes en un único destino, transformando los
datos en un formato común. Estos procesos se utilizan para extraer datos de
diferentes fuentes, transformarlos en un formato común y cargarlos en un destino
común, como puede ser un \textit{data lake}.

Los procesos ETL, fundamentales en el ámbito de la gestión de datos, presentan
atributos distintivos que facilitan la integración eficaz de información
procedente de diversas fuentes:

\begin{itemize}
	\item \textbf{Adaptabilidad:} los procesos ETL deben de adaptarse a la
		estructura de los datos de la fuente de origen, ya que dichas fuentes
		pueden tener diferentes estructuras y tener tipos de datos diferentes
		(la característica de \textit{heterogeneidad} de los datos que ya se ha
		mencionado).
	\item \textbf{Escalabilidad:} otra de las características clave de los
		procesos ETL es que sean escalables, ya que los datos que se muestran en
		los dashboards suelen ser datos que se generan de manera continua, y por
		lo tanto los procesos ETL deben ser capaces de procesar grandes
		cantidades de datos de manera eficiente. En ocasiones, los procesos ETL
		se pueden realizar en \textit{streaming}, lo que significa que los datos
		se procesan en tiempo real a medida que se generan.
	\item \textbf{Eficiencia:} los procesos ETL deben ser eficientes, puesto que
		el tiempo de procesamiento de los datos es un factor vital en el ámbito
		del \textit{big data}. Los procesos ETL deben ser capaces de procesar
		grandes cantidades de datos en un tiempo razonable para que los datos
		estén disponibles en el menor tiempo posible.
	\item \textbf{Fiabilidad:} la fiabilidad es un componente crítico de todo el
		flujo de datos, ya que estos se utilizan para la toma de decisiones
		importantes de cualquier empresa. Los procesos ETL deben ser capaces de
		procesar los datos de manera fiable y consistente, para que los datos
		que se visualicen y analicen posteriormente sean correctos y fiables.
\end{itemize}


\subsection{Funcionamiento}
Los procesos ETL se dividen en tres fases principales: \textit{(1) Extraer},
\textit{(2) Transformar} y \textit{(3) Cargar}, como se muestra en el siguiente
diagrama:

\begin{minipage}{\linewidth}
	\centering
	\includegraphics[width=0.8\textwidth]{etl.png}
	\captionof{figure}{Fases de un proceso ETL}
\end{minipage}

Como entrada, se tienen datos presuntamente heterogéneos que no se pueden
analizar de manera eficiente. Tras aplicar todos los pasos de las fases
anteriores, se obtiene como salida un conjunto de datos corregidos y listos para
ser analizados en el destino indicado, sea cual sea el paradigma de
almacenamiento de datos elegido.

\paragraph{Extracción (1)}
En este proceso se obtienen los datos de las fuentes de datos, que pueden ser
bases de datos, logs, APIs, etc. En esta fase, se pueden aplicar filtros para
extraer solo los datos que se necesiten, y se pueden extraer datos de múltiples
fuentes \emph{heterogéneas}.

La fase de extracción se puede realizar de dos formas: continua o incremental.
Una extracción incremental se realiza de manera periódica, por ejemplo, cada
hora, cada día o cada semana, y se extraen los datos que se han generado desde
la última extracción. Esto es útil cuando los datos se generan de manera
periódica y se necesita mantener actualizada la información. Por otro lado, en
una extracción continua se extraen los datos en tiempo real según se van
generando. Esto puede ser útil para procesar datos que se generan en tiempo
real, como logs o datos de sensores.


\newpage{}
\paragraph{Transformación (2)}
Durante esta fase, se transforman los datos extraídos en la fase anterior,
normalmente aplicándoles un proceso de limpieza y transformación a un
formato común. En este paso, se pueden aplicar diferentes operaciones a los
datos, como la limpieza, la agregación, la normalización, la conversión de
formatos, etc.

Uno de los tipos de transformaciones de datos más comunes es la limpieza, que
consiste en la revisión y corrección de los datos extraídos, para asegurar que
se almacena información correcta y consistente. Durante esta fase se contemplan
operaciones más complejas, como pueden ser la agregación de datos, la conversión
de formatos, la normalización de datos, el cifrado, etc. La limpieza de datos
puede ser una tarea muy sencilla, como la eliminación de caracteres
delimitadores, o muy compleja, como la corrección de errores en los datos, la
detección de duplicados o la minimización~\cite{mezmir2020qualitative} y/o
compresión~\cite{lelewer1987data} de datos (eliminación de información no
relevante o redundante).

Estos procesos de transformación son vitales cuando el sistema maneja una gran
cantidad de datos heterogéneos de múltiples fuentes de manera simultánea, como
puede ser el caso de un \textit{data lake} o un \textit{data warehouse}.
En el caso del primero, no es necesaria la transformación de los
datos a un formato común, pero si otros procesos clave como la limpieza y la
normalización de los datos, entre otros.

\paragraph{Carga (3)}
En este proceso se vuelcan los datos transformados en el destino final.
Frecuentemente, los datos se almacenan, dependiendo del paradigma de
almacenamiento elegido, en una \textit{data lake}, \textit{data warehouse} o
\textit{data lakehouse} para su posterior análisis.

Las características de la carga de datos varían dependiendo de la arquitectura
de datos que se esté utilizando. Por ejemplo, en ciertos sistemas puede ser
necesario cifrar los datos antes de cargarlos en el destino, o puede ser
necesario realizar una carga incremental para mantener actualizada la información
en el destino. En otros casos, puede ser necesario realizar una carga masiva
para cargar grandes cantidades de datos en el destino de una sola vez. La
periodicidad de la carga es una característica clave del \textit{big data}, como
se ha mencionado anteriormente~(ver \fullref{sec:bigdata}).


\newpage{}
\subsection{Alternativas}
Aunque lo más común es el flujo anteriormente explicado de \textit{extracción},
\textit{transformación} y \textit{carga}, existen algunos flujos alternativos
que son útiles para ciertos procesos diferentes:

\begin{itemize}
	\item \textbf{Virtualización de datos:} capa virtual de abstracción que
		permite acceder a los datos de las fuentes sin necesidad de extraerlos.
		Esto permite ahorrar espacio de almacenamiento y tiempo de
		procesamiento, pero suele ser menos eficiente en términos de rendimiento
		y no es compatible con todas las arquitecturas de datos.

		\begin{minipage}{\linewidth}
			\centering
			\includegraphics[width=0.65\textwidth]{virt.png}
			\captionof{figure}{Ejemplo de flujo con virtualización}
		\end{minipage}
	\item \textbf{Proceso \textit{ELT}\footnote{\url{https://www.ibm.com/topics/elt}}:}
		en lugar de transformar los datos antes de cargarlos en el destino, se
		cargan los datos en bruto y se transforman en el destino. Funciona bien
		para grandes conjuntos de datos sin estructura que requieran una carga
		(o recarga) contínua, aunque, al igual que la virtualización, puede ser
		menos eficiente o incompatible con algunas arquitecturas de datos, como
		los \textit{data warehouses}.

		\begin{minipage}{\linewidth}
			\centering
			\includegraphics[width=0.75\textwidth]{elt.png}
			\captionof{figure}{Diagrama de flujo de un proceso \textit{ELT}}
		\end{minipage}
\end{itemize}


\newpage{}
\section{Cuadros de mandos (\textit{dashboards})}\label{sec:dashboards}
\paragraph{Definición}
Los cuadros de mandos, en adelante \textit{dashboards}, son soluciones en forma
de interfaz gráfica que muestran información relevante de manera visual sobre un
proceso o negocio. Aunque el término se utiliza en muchos ámbitos (indicadores
comerciales, de producción, de marketing, de calidad, de recursos humanos\ldots)
en este proyecto se utilizará en el ámbito de la monitorización de sistemas y
procesos de negocio.

En el ámbito de deste proyecto, los dashboards reflejan en tiempo real el
rendimiento de actividades o procesos de negocio, y se utilizan para tomar
decisiones informadas basándose en los mismos. Por ejemplo, el dashboard de una
empresa digital puede mostrar desde el rendimiento de la arquitectura en tiempo
real hasta el número de ventas conseguidas, y permitir a los directivos tomar
decisiones informadas sobre el futuro de la empresa (e.g. necesidad de aumentar
la capacidad de los servidores, lanzar una campaña de marketing, etc.).

\paragraph{Características}
Los dashboards cuentan con una serie de características que los hacen útiles
para la toma de decisiones:~\cite{mier2023dashboards}

\begin{itemize}
	\item \textbf{Visualización de datos:} es la característica fundamental de
		cualquier dashboard, y aquella que determina su utilidad.
		La visualización de datos es la ciencia de presentar los datos de manera
		que se pueda extraer información útil y realizar decisiones informadas
		sobre ellos. Un buen dashboard cuenta con gráficas, tablas, indicadores,
		etc. que permiten al usuario entender la información que se está
		presentando con un conocimiento técnico mínimo.
	\item \textbf{Interactividad y personalización:} un dashboard debe permitir
		al usuario interactuar con los datos (filtrarlos, ordenarlos,
		profundizar en ellos...) y ajustar la información que se muestra sobre
		cada proceso o negocio que se esté evaluando (granularidad de la
		información). Esta capacidad asegura que el dashboard se adapte tanto a
		las necesidades actuales como a las evoluciones futuras de lo que se
		esté analizando.
	\item \textbf{Accesibilidad y portabilidad:} un dashboard debe ser accesible
		desde una variedad de situaciones y dispositivos, manteniendo su
		funcionalidad y forma. Aunque normalmente los dashboards se analizan en
		pantallas grandes, es importante que también se puedan consultar en
		otras circunstancias, como dispositivos móviles.
\end{itemize}


\newpage{}
\section{Infraestructura como código}
La infraestructura como código (o \textit{IaC} por sus siglas en inglés) es una
práctica que consiste en gestionar la infraestructura de un sistema de manera
automática y programática mediante código, en lugar de configuraciones manuales.

La infraestructura como código permite gestionar la arquitectura global de un
sistema de manera eficiente y escalable, y facilita la creación y el mantenimiento
de entornos de desarrollo y producción, lo que elimina la necesidad de realizar
tareas repeititivas~\cite{beyer2016site} (\textit{TOIL} por sus siglas en
inglés), reduce la posibilidad de errores humanos y favorece la replicabilidad
de los procesos.

En el ámbito de este proyecto, la infraestructura como código se utilizará para
gestionar el despliegue y orquestación de los servicios requeridos para el
paradigma de almacenamiento que se escoja, como los servicios de ingesta o de
visualización de datos.

% La infraestructura como código es la parte más importante del desarrollo de este
% proyecto, ya que una buena configuración y toma de decisiones a la hora de
% desplegar un proyecto de este calibre es vital para el correcto funcionamiento
% posterior a la hora de ingestar y tratar con datos heterogéneos.

\chapter{Descripción general del proyecto}
Esta sección describe el proyecto en términos generales, incluyendo una descripción de los
problemas que se pretenden resolver, las partes interesadas en el proyecto y una valoración de
las alternativas consideradas.


\section{Partes interesadas (stakeholders)}\label{sec:stakeholders}
Las partes interesadas en el proyecto son aquellas personas o entidades que tienen un interés
en el mismo, ya sea porque se ven afectadas por el resultado del proyecto, o porque tienen
algún tipo de interés en el mismo. Las partes interesadas en este proyecto son las siguientes:

\begin{enumerate}
	\item \textbf{Okticket}: la empresa es la principal parte interesada en el proyecto, ya que
		es la que se beneficiará directamente de los resultados del mismo, así como de las
		oportunidades de negocio que se abren con la explotación de los datos. Dentro de la empresa,
		se pueden identificar dos entidades:
		\begin{itemize}
			\item \textbf{Equipo de desarrollo de la empresa}: el equipo de desarrollo es otra parte
				interesada en el proyecto, ya que son los encargados de llevar a cabo la implementación
				del sistema y de garantizar su correcto funcionamiento, además de gestionar el soporte
				de servicio a nivel técnico.
			\item \textbf{Equipo de soporte de la empresa}: el sistema planteado ahorraría tiempo al equipo de
				soporte, ya que les permitiría analizar los datos de forma más eficiente e identificar
				problemas antes de que tener que resolver las peticiones de los clientes afectados a
				nivel básico.
		\end{itemize}
	\item \textbf{Clientes}: los clientes de la empresa también son partes interesadas, puesto
		que se beneficiarán de los nuevos servicios que se ofrecen, como los dashboards de
		negocio que se han descrito anteriormente. Estos clientes no son necesariamente los
		usuarios finales, sino los administradores y gestores de las empresas que utilizan
		Okticket como herramienta de gestión de gastos.
	\item \textbf{Investigador y desarrollador (\emph{\author}):} el desarrollador del
		proyecto tiene la oportunidad de aplicar los conocimientos adquiridos en el desarrollo de un
		proyecto real, y de adquirir nuevos conocimientos en el proceso.
\end{enumerate}

\section{Valoración de alternativas}\label{sec:alternativas}
\subsection{Criterios de evaluación}\label{subsec:criterios}

\subsection{Alternativas consideradas}\label{subsec:alternativas}

\subsection{Resultados}\label{subsec:resultados}

\newpage{}
\section{Descripción del proyecto}\label{sec:descripcion}

\chapter{Planificación del proyecto}\label{chap:planif}
La planificación de un proyecto es fundamental para su correcto funcionamiento y
desarrollo, dentro de los plazos y costes establecidos. Se presenta un primer
apartado de metodología, un segundo apartado con la planificación inicial para
posteriormente inferir en base a esta el presupuesto.


\section{Metodología}\label{sec:metodología}
En este capítulo se aborda la metodología adoptada para el desarrollo del
proyecto, fundamentada en principios ágiles y enfocada en la entrega continua de
valor. La elección de \textit{Scrum}, una metodología que permite elaborar
productos software de manera incremental, revisando el producto continuamente y
adaptándolo a las necesidades del cliente, subraya el compromiso con la
adaptabilidad y la mejora continua del producto.

La estructura de este capítulo se organiza en torno a la descripción detallada
de la metodología \textit{Scrum}, la visualización de la planificación y las
estrategias de comunicación adoptadas. A través de esta metodología, se busca
optimizar los recursos disponibles, ajustarse a los plazos establecidos y
garantizar la calidad del producto final.

La implementación de \textit{Scrum} se complementa con herramientas de
visualización y gestión de proyectos, como los tableros \textit{Kanban}, que
facilitan la organización y seguimiento de las tareas. Además, se pone especial
énfasis en la comunicación efectiva dentro del equipo de desarrollo y con los
stakeholders, asegurando así una alineación constante con los objetivos del
proyecto.

Existen otras variantes de los tableros \textit{Kanban} que se pueden
utilizar para visualizar el progreso de las tareas, pero en este proyecto se ha
elegido esta alternativa para facilitar la visualización de las tareas y su
estado~(ver \fullref{subsec:visual_planif}). La visualización de la
planificación es esencial para el seguimiento y control del proyecto, ya que
permite identificar posibles desviaciones y tomar medidas correctivas de manera
temprana.

Este enfoque metodológico no solo refleja la planificación y ejecución del
proyecto, sino que también establece las bases para una gestión eficaz,
adaptativa y orientada a resultados.


\newpage{}
\subsection{Scrum}\label{subsec:scrum}
Para la planificación del proyecto se ha escogido \textit{Scrum}, una
metodología ``ágil'' que se basa en la realización de iteraciones cortas y en la
adaptación a los cambios. La metodología \textit{Scrum} se estructura en
\textit{sprints} (iteraciones cortas de una duración fija), en las que se llevan
a cabo una serie de tareas que se han planificado previamente.

El primer paso de la metodología \textit{Scrum} es la creación de un
\textit{product backlog}, una lista ordenada de las tareas a realizar durante el
desarrollo del producto, a partir de los requisitos del sistema, que a su vez
son una versión refinada de los requisitos iniciales del proyecto. A partir de
este \textit{product backlog} se planifican las tareas que se llevarán a cabo en
cada \textit{sprint}, de manera que sea posible cumplir con los objetivos del
proyecto en el tiempo establecido.

A diferencia de metodologías tradicionales o \emph{en cascada}, \textit{Scrum}
permite la adaptación a los cambios y la mejora continua del producto, ya que se
revisa y se adapta en cada \textit{sprint} según las necesidades del cliente y
del equipo de desarrollo. Por otro lado, \textit{Scrum} se diferencia de otras
metodologías ágiles como \textit{XP} en que no se centra tanto en las prácticas
de desarrollo, sino en la gestión del proyecto y en la entrega de valor al
cliente.

\begin{minipage}{\linewidth}
	\includegraphics[width=0.85\textwidth]{planif/scrum.png}
	\captionof{figure}{Diagrama de la metodología \textit{Scrum}}
\end{minipage}

\paragraph{Roles}
En \textit{Scrum} se distinguen tres roles principales:

\begin{itemize}
	\item \textbf{Product Owner:} es la persona responsable de definir los
		requisitos del producto y de priorizar las tareas del
		\textit{product backlog}. Es el enlace entre el equipo de desarrollo y
		el cliente, y es el responsable de garantizar que el producto cumple con
		las expectativas del cliente. En el caso de este proyecto, el
		\textit{Product Owner} es el director tecnológico de la empresa.
	\item \textbf{Scrum Master:} es la persona responsable de garantizar que el
		equipo de desarrollo sigue la metodología \textit{Scrum} y de eliminar
		los obstáculos que puedan surgir durante el desarrollo del proyecto. El
		\textit{Scrum Master} es el encargado de organizar las reuniones diarias
		y de asegurar que el equipo de desarrollo cumple con los plazos y los
		objetivos del proyecto. En este proyecto, el \textit{Scrum Master} son
		los tutores académicos del proyecto.
	\item \textbf{Equipo de desarrollo:} es el equipo encargado de llevar a cabo
		las tareas del \textit{product backlog} y de entregar el producto final.
		El equipo de desarrollo es autoorganizado y multidisciplinario, y se
		organiza en torno a las tareas que se van a realizar en cada
		\textit{sprint}. Para este proyecto, el ``equipo'' de desarrollo está
		constituido únicamente por el alumno, que se encarga de todas las tareas
		de desarrollo y documentación.
	\item \textbf{Stakeholders:} son las partes interesadas en el proyecto, como
		los clientes, los usuarios finales y los patrocinadores, que desconocen
		el proceso de desarrollo pero tienen un interés en el producto final y
		en su correcto funcionamiento.
\end{itemize}

\paragraph{Estimación}
En la metodología \textit{Scrum} se pueden utilizar diferentes técnicas de
estimación de tareas, como la estimación en puntos de historia, la estimación en
horas o la estimación en tallas de camiseta. En este proyecto se ha optado por
la estimación en tallas de camiseta, que consiste en asignar a cada tarea una
talla que representa su complejidad y su duración. Las tallas de camiseta se
suelen representar con letras (XS, S, M, L, XL), que se pueden traducir a puntos
de historia siguiendo la secuencia de Fibonacci, es decir, $XS = 1$, $S = 2$,
$M = 3$, $L = 5$, $XL = 8$.

La estimación en Scrum es esencial para la planificación de los \textit{sprints}
y para la asignación de tareas al equipo de desarrollo. La estimación en tallas
se considera óptima para este proyecto, ya que permite una estimación rápida y
sencilla de las tareas, al no necesitar una coordinación entre un equipo
completo de desarrollo.

Además de la estimación del tamaño de las tareas, también se realiza una
estimación sobre la \textit{prioridad} de las mismas, que se representa
siguiendo el equivalente de \textit{GitHub} al sistema de colores de semáforo,
donde el rojo (P0) es la máxima prioridad y el verde (P2) la mínima.

\newpage{}
\subsection{Visualización}\label{subsec:visual_planif}
Para la visualización de la planificación se ha utilizado la herramienta de
gestión de proyecto de \textit{GitHub}, que permite múltiples visualizaciones de
tareas e \textit{issues} en tableros separados.

\begin{itemize}
	\item Se utiliza un tablero de \textit{requisitos} al estilo \textit{Kanban}
		para visualizar los requisitos del proyecto y su estado, siguiendo con
		la metodología \textit{Scrum}. Un tablero \textit{Kanban} es una
		herramienta visual que permite gestionar el flujo de trabajo de un
		proyecto por ``sprints'', dividiendo las tareas en columnas y
		moviéndolas de una columna a otra según su estado.

		\begin{figure}[H]
			\centering
			\includegraphics[width=\textwidth]{planif/kanban3.png}
			\caption{Tablero \textit{Kanban} del proyecto}
			\label{fig:kanban}
		\end{figure}
	\item Adicionalmente, se utiliza un \textit{roadmap} de apartados de la
		memoria, separado del tablero de desarrollo normal, donde se visualiza
		su estado y sus fechas límite. Este \textit{roadmap} no está relacionado
		con la metodología \textit{Scrum}, sino que se ha creado para facilitar
		la visualización del progreso de cada sección y de la memoria en general.

		\begin{minipage}{\linewidth}
			\centering
			\includegraphics[width=0.9\textwidth]{planif/roadmap.png}
			\captionof{figure}{Roadmap de apartados de la memoria}
		\end{minipage}
\end{itemize}


\subsection{Comunicación}\label{subsec:comunicación}
La comunicación con los tutores y con el equipo de desarrollo se considera
fundamental para el correcto desarrollo del proyecto. Puesto que el trabajo se
desarrolla de manera presencial en la oficina de la empresa, la comunicación con
el equipo de desarrollo se realiza de manera frecuente y directa, mientras que
la comunicación con los tutores se realiza de manera remota pero igual de
frecuente, manteniendo el contacto mediante correo electrónico y Teams para
pedir revisiones e informar sobre el estado del trabajo en todo momento.


\subsection{Herramientas}\label{subsec:herr_planif}
Con el objetivo de facilitar las tareas de desarrollo y cumplimentar los
requisitos por parte de la empresa, se utilizan las siguientes plataformas y
herramientas de desarrollo para la fabricación del proyecto:

\begin{itemize}
	\item \textbf{GitHub:} plataforma de desarrollo colaborativo para el
		desarrollo del proyecto. Se utiliza para la gestión de tareas,
		seguimiento del desarrollo y la documentación del proyecto.
	\item \textbf{Suite de Atlassian (\emph{Jira, Bitbucket}):} Suite de
		herramientas de gestión de proyectos y desarrollo colaborativo. Se
		utiliza para el desarrollo y documentación del proyecto de parte de la
		empresa.
	\item \textbf{Suite de Microsoft (\emph{Teams, Outlook}):} se utilizan las
		plataformas de comunicación puestas a disposición por la universida.
\end{itemize}


\newpage{}
\section{Planificación inicial}\label{sec:planif_inicial}
Como se ha mencionado anteriormente, se utiliza la metodología \textit{Scrum}
para la planificación y desarrollo del proyecto. En la figura \ref{fig:backlog}
se puede ver el \textit{backlog} de tareas que se planifican en el proyecto.

\begin{figure}[H]
	\centering
	\includegraphics[width=\textwidth]{planif/backlog.png}
	\caption{Planificación inicial del proyecto}
	\label{fig:backlog}
\end{figure}

Las historias de usuario anteriores se clasifican y categorizan según su
prioridad y tamaño, haciendo uso de la estrategia de tallas de camiseta como
mencionado anteriormente. En el tablero \textit{Kanban}
(ver figura \ref{fig:kanban}) se puede ver en todo momento el estado de las HU,
su progreso y sus características. El listado inicial (ordenado según su prioridad)
es el siguiente:

\begin{table}[H]
	\centering
	\begin{tabular}{|p{0.7\linewidth}|c|c|}
		\hline
		\textbf{Nombre} & \textbf{Prioridad} & \textbf{Tamaño} \\
		\hline
		\hline
		Creación de la infraestructura base (técnica) & P0\cellcolor{red!50} & L\cellcolor{orange!50} \\
		\hline
		Como desarrollador de Okticket, quiero que la arquitectura se despliegue y orqueste de manera automática & P0\cellcolor{red!50} & XL\cellcolor{red!50} \\
		\hline
		Como desarrollador de Okticket, quiero que se ingesten de manera automática datos de la base de datos interna de MongoDB & P0\cellcolor{red!50} & M\cellcolor{yellow!50} \\
		\hline
		Como desarrollador de Okticket, quiero que se ingesten de manera automática datos de la base de datos interna de MySQL & P0\cellcolor{red!50} & M\cellcolor{yellow!50} \\
		\hline
		Como desarrollador de Okticket, quiero que los datos se limpien de manera automática & P0\cellcolor{red!50} & M\cellcolor{yellow!50} \\
		\hline
		Como trabajador de Okticket, quiero poder ver y consultar datos internos de la empresa & P1\cellcolor{orange!50} & L\cellcolor{orange!50} \\
		\hline
		Como desarrollador de Okticket, quiero que se ingesten de manera automática logs de balanceador de AWS & P1\cellcolor{orange!50} & S\cellcolor{green!25} \\
		\hline
		Como desarrollador de Okticket, quiero poder ver el estado general de la infraestructura & P1\cellcolor{orange!50} & L\cellcolor{orange!50} \\
		\hline
		Como desarrollador de Okticket, quiero que los datos contengan metadatos que faciliten su filtrado o búsqueda & P2\cellcolor{yellow!50} & S\cellcolor{green!25} \\
		\hline
		Como trabajador de Okticket, quiero poder ver y consultar datos de empresas cliente & P2\cellcolor{yellow!50} & M\cellcolor{yellow!50} \\
		\hline
		Como gestor de una empresa cliente, quiero poder ver información relevante sobre mi empresa que recoja Okticket & P2\cellcolor{yellow!50} & L\cellcolor{orange!50} \\
		\hline
		Como desarrollador de Okticket, quiero poder ingestar datos de APIs externas a la empresa & P2\cellcolor{yellow!50} & L\cellcolor{orange!50} \\
		\hline
		Como desarrollador de Okticket, quiero poder ingestar información de páginas web externas (\textit{scraping}) & P2\cellcolor{yellow!50} & XL\cellcolor{red!50} \\
		\hline
	\end{tabular}
	\caption{Historias de usuario iniciales}
	\label{tab:initial_tasks}
\end{table}

Siguiendo la tabla anterior, se pueden planear los \textit{sprints} y
asignar las tareas a cada uno de ellos.


\newpage{}
\section{Presupuesto}\label{sec:presupuesto}
Para poder llevar a cabo este proyecto, se realiza una estimación del coste
total neceario para su desarrollo, que se divide en dos partes: el coste del
material, que incluye el coste de los recursos necesarios para el desarrollo del
proyecto, y el coste del personal, que incluye el coste de las horas de trabajo
del desarrollador.

Se estima un horizonte de desarrollo de 3 meses, que es el tiempo estimado y
disponible para el desarrollo del proyecto.


\subsection{Presupuesto de material}\label{subsec:pres_material}
Puesto que el proyecto se desarrolla en la empresa, se dispone de todos los
recursos físicos necesarios para llevar a cabo el proyecto, es decir, que no se
incluirá el coste del ordenador o de la conexión a internet en el presupuesto.

Sin embargo, se incluirá el coste de las herramientas y servicios utilizados
durante el desarrollo del proyecto, como el coste de las licencias de software,
el coste de los servicios en la nube, el coste de las herramientas de
desarrollo, etc.

Es importante destacar de que los precios de los servicios en la nube son
aproximados y pueden variar en función de la región, el tipo de instancia, el
tipo de almacenamiento, etc. Por lo tanto, los precios presentados en este
presupuesto son orientativos y pueden variar en función de las necesidades del
proyecto. En este caso, se analizan los precios a junio de 2024 en la región
de Amazon Web Services (AWS) de \texttt{eu-north-1} (Estocolmo).

\begin{table}[H]
	\centering
	\small
	\begin{tabular}{|l|l|r|r|r|}
	\hline
	\textbf{Categoría} & \textbf{Ítem} & \textbf{Cantidad} & \textbf{Coste unitario} & \textbf{Coste total} \\
	\hline
	\hline
	AWS Compute & Fargate & 7168 vCPU-h/mes & 0,04928€/vCPU-h & 353,24€/mes \\
	 & & 18432 GB-h/mes & 0,00539€/GB-h & 99,35€/mes \\
	\hline
	AWS Storage & EFS & 512 GB/mes & 0,38€/GB-mes & 194,56€/mes \\
	 & S3 & 256 GB/mes & 0,0255€/GB-mes & 6,53€/mes \\
	\hline
	AWS Network & VPC & 4 NAT Gateways & 0,054€/h & 155,52€/mes \\
	 & ELB & 3 ALBs & 0,012€/h & 25,92€/mes \\
	 & & 1 NLB & 0,027€/h & 19,44€/mes \\
	 & Route 53 & 3 registros & 0,50€/registro-mes & 1,50€/mes \\
	\hline
	AWS Security & IAM & - & Sin cargo & 0,00€ \\
	 & KMS & 1 CMK & 1,00€/mes & 1,00€/mes \\
	 & Secret Manager & 12 secretos & 0,40€/secreto-mes & 4,80€/mes \\
	\hline
	Stack KELK & Kafka & - & Licencia gratuita & 0,00€ \\
	 & Elasticsearch & - & Licencia gratuita & 0,00€ \\
	 & Logstash & - & Licencia gratuita & 0,00€ \\
	 & Kibana & - & Licencia gratuita & 0,00€ \\
	\hline
	AWS Container & ECS & - & Sin cargo & 0,00€ \\
	\hline
	AWS Monitor & CloudWatch & 5 métricas & 0,30€/métrica-mes & 1,50€/mes \\
	 & X-Ray & 50,000 trazas/mes & 4,60€/1M trazas & 0,23€/mes \\
	\hline
	Despliegue & Terraform & - & Licencia gratuita & 0,00€ \\
	\hline
	\textbf{Subtotal} & \multicolumn{4}{r|}{863,59€/mes} \\
	\hline
	\hline
	Otros & Support & Plan Basic & Sin cargo & 0,00€ \\
	 & Optimización & - & 5\% del subtotal & 43,18€ \\
	 & Contingencia & - & 10\% del subtotal & 86,36€ \\
	\hline
	\textbf{Total} & \multicolumn{4}{r|}{993,13€/mes} \\
	\hline
	\end{tabular}
	\caption{Propuesta de presupuesto mensual de materiales (región eu-north-1)}
	\label{tab:presupuesto_material}
\end{table}

Asumiendo que el proyecto se desarrolla en la región de AWS de Estocolmo
(\texttt{eu-north-1}), el coste total del material asciende a 993,13€ (novecientos
noventa y tres euros con trece céntimos) al mes, que incluye
el coste de los servicios en la nube, el coste de las licencias de software y el
coste de las herramientas de desarrollo.

Suponiendo un horizonte de desarrollo de 3 meses, el coste total del material
durante el desarrollo del proyecto asciende a 2.979,39€ (dos mil novecientos
setenta y nueve euros con treinta y nueve céntimos).


\newpage{}
\subsection{Presupuesto de personal}\label{subsec:pres_personal}
A continuación, se presenta una propuesta de presupuesto de personal para el
desarrollo del proyecto, que incluye el coste de las horas de trabajo según
cada rol y el coste total del personal.

\begin{table}[H]
	\centering
	\small
	\begin{tabular}{|l|l|r|r|r|}
	\hline
	\textbf{Rol} & \textbf{Descripción} & \textbf{Horas/mes} & \textbf{CU (€/h)} & \textbf{Coste total} \\
	\hline
	\hline
	Arquitecto & Diseño de la arquitectura y supervisión & 40 & 60 & 2.400,00€/mes \\
	\hline
	Desarrollador & Desarrollo y mantenimiento & 160 & 45 & 7.200,00€/mes \\
	\hline
	Administrador & Gestión de sistemas y seguridad & 160 & 50 & 8.000,00€/mes \\
	\hline
	DevOps & Infraestructuras y monitorización & 80 & 55 & 4.400,00€/mes \\
	\hline
	\textbf{Subtotal} & \multicolumn{4}{r|}{22.000,00€/mes} \\
	\hline
	\hline
	Otros & \multicolumn{3}{|l|}{IVA (21\%)} & 4.620,00€/mes \\
	 & \multicolumn{3}{|l|}{Margen (5\%)} & 1.100,00€/mes \\
	\hline
	\textbf{Total} & \multicolumn{4}{r|}{27.720,00€/mes} \\
	\hline
	\end{tabular}
	\caption{Propuesta de presupuesto de personal}
	\label{tab:presupuesto_personal_aws}
\end{table}

El coste del personal es ficticio, pero se ha calculado en base a experiencias
previas de contratación y subcontratación de personal en la empresa, además de
tener en cuenta el coste medio de los roles en Asturias.

El coste total del personal asciende a 27.720,00€ (veintisiete mil setecientos
veinte euros) al mes, que incluye el coste de las horas de trabajo de cada rol,
el IVA y el margen de beneficio industrial.

Asumiendo un horizonte de desarrollo de 3 meses, el coste total del personal
durante el desarrollo del proyecto asciende a 83.160,00€ (ochenta y tres mil
ciento sesenta euros).


\newpage{}
\subsection{Presupuesto total}\label{subsec:pres_total}
Finalmente, se presenta el presupuesto total del proyecto, que incluye el coste
del material y el coste del personal, así como el coste total del proyecto,
durante el horizonte de desarrollo establecido de 3 meses.

\begin{table}[H]
	\centering
	\small
	\begin{tabular}{|l|r|}
	\hline
	\textbf{Concepto} & \textbf{Coste} \\
	\hline
	Presupuesto de materiales & 2.979,39€ \\
	\hline
	Presupuesto de personal & 83.160,00€ \\
	\hline
	\textbf{Subtotal} & \textbf{86.139,39€} \\
	\hline
	\hline
	Beneficio industrial (15\%) & 12.920,91€ \\
	\hline
	\textbf{Total} & \textbf{99.060,30€} \\
	\hline
	\end{tabular}
	\caption{Costes combinados de presupuesto y materiales con beneficio industrial}
	\label{tab:costes_combinados}
\end{table}

El presupuesto total del proyecto asciende a 99.060,30€ (noventa y nueve mil
sesenta euros con treinta céntimos), que incluye el coste del material, el coste
del personal y el margen de beneficio industrial.

\chapter{Análisis del sistema}
En este capítulo se presenta un análisis detallado del sistema a desarrollar,
desglosando las funcionalidades principales en epics y historias de usuario.
Este enfoque permite una visión estructurada del proyecto, facilitando su
planificación y desarrollo.

\section{Epics}
Los \textit{epics} representan las grandes áreas funcionales del proyecto.
Se han identificado las siguientes historias épicas:

\begin{enumerate}
    \item Infraestructura y despliegue
    \item Ingesta de datos
    \item Procesamiento y almacenamiento de datos
    \item Visualización y análisis
\end{enumerate}

Cada uno de estos epics engloba un conjunto de funcionalidades relacionadas
que, en conjunto, conforman el proyecto planteado.

\newpage{}
\section{Historias de usuario}
Las historias de usuario describen las funcionalidades específicas desde la
perspectiva del usuario final. A continuación, se detallan las historias de
usuario para cada epic, incluyendo sus criterios de aceptación:


\subsection{Epic 1: Infraestructura y despliegue}
Este epic se centra en la creación y gestión de la infraestructura necesaria
para el sistema.

\begin{itemize}
    \item \textbf{HU1.1:} Como desarrollador de Okticket, quiero poder
    desplegar un prototipo del sistema en mi entorno local para facilitar el
    desarrollo y las pruebas iniciales.
    \begin{itemize}
        \item El sistema se desplegará correctamente en un entorno local.
        \item Todos los servicios estarán funcionando.
        \item Se podrá acceder a las interfaces web de los servicios.
    \end{itemize}

    \item \textbf{HU1.2:} Como desarrollador de Okticket, quiero que la
    arquitectura se despliegue y orqueste de manera automática en la nube para
    facilitar la gestión y el paso a producción del sistema.
    \begin{itemize}
        \item El sistema se desplegará automáticamente con un solo comando.
        \item Todos los recursos en la nube se crearán correctamente.
        \item Los servicios serán accesibles a través de URLs públicas.
    \end{itemize}

    \item \textbf{HU1.3:} Como administrador del sistema, quiero que la
    infraestructura sea capaz de escalar automáticamente en función de la
    demanda para optimizar el rendimiento y los costos.
    \begin{itemize}
        \item Se habrán configurado políticas de auto-escalado para los
        servicios críticos.
        \item El sistema aumentará los recursos cuando la carga aumente.
        \item El sistema reducirá los recursos cuando la demanda disminuya.
    \end{itemize}
\end{itemize}


\newpage{}
\subsection{Epic 2: Ingesta de datos}
La ingesta de datos es fundamental para el funcionamiento del sistema,
abarcando diversas fuentes de información.

\begin{itemize}
    \item \textbf{HU2.1:} Como desarrollador de Okticket, quiero que se
    ingesten de manera automática datos de la base de datos interna de MongoDB
    para centralizar la información.
    \begin{itemize}
        \item Los datos de MongoDB se ingestarán correctamente en el sistema.
        \item La ingesta se realizará de forma periódica y automática.
        \item Se podrá verificar la integridad de los datos ingestados.
    \end{itemize}

    \item \textbf{HU2.2:} Como desarrollador de Okticket, quiero que se
    ingesten de manera automática datos de la base de datos interna de MySQL
    para tener una visión completa de los datos.
    \begin{itemize}
        \item Los datos de MySQL se ingestarán correctamente en el sistema.
        \item La ingesta se realizará de forma periódica y automática.
        \item Se podrá verificar la integridad de los datos ingestados.
    \end{itemize}

    \item \textbf{HU2.3:} Como desarrollador de Okticket, quiero que se
    ingesten de manera automática logs de balanceador de AWS para monitorear
    el rendimiento de la infraestructura.
    \begin{itemize}
        \item Los logs del balanceador de AWS se ingestarán correctamente.
        \item La ingesta se realizará en tiempo real o con un retraso mínimo.
        \item Los logs ingestados incluirán toda la información relevante.
    \end{itemize}

    \item \textbf{HU2.4:} Como desarrollador de Okticket, quiero poder ingestar
    datos de APIs externas a la empresa para enriquecer nuestros análisis.
    \begin{itemize}
        \item Se podrán configurar conexiones a múltiples APIs externas.
        \item Los datos de las APIs se ingestarán correctamente en el sistema.
        \item La ingesta de APIs se realizará de forma programada o bajo demanda.
    \end{itemize}

    \item \textbf{HU2.5:} Como desarrollador de Okticket, quiero poder ingestar
    información de páginas web externas (scraping) para obtener datos
    adicionales relevantes.
    \begin{itemize}
        \item Se podrán configurar tareas de scraping para múltiples sitios web.
        \item Los datos obtenidos por scraping se almacenarán correctamente.
        \item El proceso de scraping respetará las políticas de los sitios web.
    \end{itemize}
\end{itemize}


\newpage{}
\subsection{Epic 3: Procesamiento y almacenamiento de datos}
Este epic se enfoca en la manipulación y organización eficiente de los datos
ingresados.

\begin{itemize}
    \item \textbf{HU3.1:} Como desarrollador de Okticket, quiero que los datos
    se limpien de manera automática para garantizar la calidad de la
    información.
    \begin{itemize}
        \item Se implementarán procesos de limpieza de datos para cada fuente.
        \item Los datos limpiados no contendrán valores nulos o incorrectos.
        \item Se mantendrá un registro de las transformaciones aplicadas.
    \end{itemize}

    \item \textbf{HU3.2:} Como desarrollador de Okticket, quiero que los datos
    contengan metadatos que faciliten su filtrado o búsqueda para mejorar la
    eficiencia en el análisis.
    \begin{itemize}
        \item Cada registro de datos incluirá metadatos relevantes.
        \item Los metadatos permitirán filtrar y buscar eficientemente.
        \item Se podrán realizar búsquedas complejas utilizando los metadatos.
    \end{itemize}
\end{itemize}


\newpage{}
\subsection{Epic 4: Visualización y análisis}
La visualización y análisis de datos es crucial para extraer valor de la
información recopilada.

\begin{itemize}
    \item \textbf{HU4.1:} Como trabajador de Okticket, quiero poder ver y
    consultar datos internos de la empresa para tomar decisiones informadas.
    \begin{itemize}
        \item Existirán paneles de control que mostrarán datos internos relevantes.
        \item Los paneles se actualizarán en tiempo real o con una frecuencia adecuada.
        \item Los usuarios podrán personalizar las visualizaciones según sus necesidades.
    \end{itemize}

    \item \textbf{HU4.2:} Como desarrollador de Okticket, quiero poder ver el
    estado general de la infraestructura para monitorear su salud y rendimiento.
    \begin{itemize}
        \item Existirá un panel que mostrará el estado de todos los servicios del sistema.
        \item Se visualizarán métricas clave como CPU, memoria y uso de red.
        \item El panel incluirá alertas visuales para problemas críticos.
    \end{itemize}

    \item \textbf{HU4.3:} Como trabajador de Okticket, quiero poder ver y
    consultar datos de empresas cliente para ofrecer un mejor servicio y
    soporte.
    \begin{itemize}
        \item Se podrán visualizar datos específicos de cada empresa cliente.
        \item La información se presentará de forma clara y organizada.
        \item Se podrán generar informes personalizados para cada cliente.
    \end{itemize}

    \item \textbf{HU4.4:} Como gestor de una empresa cliente, quiero poder ver
    información relevante sobre mi empresa que recoja Okticket para optimizar
    mis procesos y tomar decisiones estratégicas.
    \begin{itemize}
        \item Existirá un panel de control específico para cada empresa cliente.
        \item Los datos se presentarán de forma comprensible para usuarios no técnicos.
        \item Se incluirán métricas y KPIs relevantes para la gestión empresarial.
    \end{itemize}
\end{itemize}

\newpage{}
\section{Story Mapping}
El Story Mapping proporciona una visión de cómo las historias de usuario se
traducen en tareas concretas de desarrollo. Esta estrategia permite una
planificación más precisa y un seguimiento efectivo del progreso del proyecto.

\begin{figure}[h]
	\centering
	\includegraphics[width=\textwidth]{storymapping.png}
	\caption{Diagrama \textit{Story Mapping} del proyecto}
	\label{fig:story_mapping}
\end{figure}

El diagrama anterior muestra cómo las historias de usuario se organizan en
epics y se desglosan en tareas específicas. Esta representación visual ayuda a
comprender la estructura del proyecto y las dependencias entre las diferentes
historias y tareas.

Este \textit{story mapping} establece una hoja de ruta clara para el desarrollo
del proyecto, asegurando que todas las historias de usuario se aborden de manera
sistemática y eficiente.

\chapter{Diseño del sistema}\label{chap:diseño}
\section{Estudio de alternativas}\label{sec:estudio}

\section{Arquitectura del sistema}\label{sec:arquitectura}

\section{Modelo de datos}\label{sec:modelo}

\chapter{Implementación}
Durante la implementación, se ha seguido la planificación y metodologías
anteriormente descritas, dividiendo el proyecto en tareas más pequeñas y
manejables, para que se puedan realizar en un periodo de tiempo razonable.

Al emplearse la metodología \textit{Scrum}, se realizan primero las tareas más
prioritarias, como la creación de la infraestructura y la ingesta de las fuentes
esenciales, y se dejan para más adelante tareas como la visualización para
clientes externos o fuentes menos críticas y más complejas, como las APIs de
terceros o el \textit{web scraping}.

Con el objetivo de realizar un desarrollo iterativo y ágil, se desarrolla
primero de todo un \textbf{prototipo de despliegue en local}, para
posteriormente \underline{migrar} este despliegue a la \textbf{nube}.
A continuación, se desarrollan los scripts de ingesta de datos para las
diferentes fuentes y se finaliza con la visualización de los datos en Kibana.

El burndown chart de la figura \ref{fig:burndown} muestra la evolución de las
tareas a lo largo del tiempo, y se puede observar cómo se han ido completando
las tareas planificadas.

\begin{figure}[htbp]
	\centering
	\begin{tikzpicture}
		\begin{axis}[
			title={Diagrama burndown del proyecto},
			date coordinates in=x,
			xmin={2024-03-18},
			xmax={2024-07-18},
			xtick={2024-03-18,2024-04-18,2024-05-18,2024-06-18,2024-07-18},
			xticklabel style={rotate=45,anchor=east},
			xticklabel={\day-\month},
			ylabel={Puntos de estimación restantes},
			ymin=0,
			ymax=41,
			grid=both,
			legend pos=north east,
			width=0.8\textwidth,
			height=0.6\textwidth,
		]
			% Línea ideal
			\addplot[blue] coordinates {
				(2024-03-18,41)
				(2024-07-18,0)
			};

			% Línea MVP
			\addplot[blue,dashed] coordinates {
				(2024-03-18,41)
				(2024-07-18,11)
			};

			% Progreso real
			\addplot[red,mark=*] coordinates {
				(2024-03-18,41)
				(2024-04-18,36)
				(2024-05-18,28)
				(2024-06-18,17)
				(2024-07-18,10)
			};

			\legend{Ideal,MVP,Real}
		\end{axis}
	\end{tikzpicture}
	\caption{Diagrama Burndown que representa el progreso del proyecto}
	\label{fig:burndown}
\end{figure}

La estimación de las tareas se realizó en el apartado \fullref{sec:planif_inicial}.

\newpage{}
La línea azul representa el progreso ideal, mientras que la línea roja muestra
el progreso real. Se toma el 18 de marzo como fecha de inicio del proyecto y el
18 de julio como fecha de finalización. Se puede observar cómo el progreso real
sigue la línea ideal, aunque con algunas desviaciones debido a la naturaleza
iterativa del desarrollo.

Pese a que el progreso del desarrollo no ha sido completamente óptimo, se ha
logrado superar los objetivos principales del proyecto, entregando un producto
mínimo viable (MVP) de calidad y sentando las bases para futuras iteraciones.
Para una explicación más detallada sobre el progreso del proyecto contra las
expectativas iniciales, se puede consultar el apartado \fullref{chap:concl}.

\newpage{}
\input{sections/71_local.tex}

\newpage{}
\section{Despliegue \textit{cloud}}\label{sec:impl_cloud}
El desarrollo principal del despliegue en la nube se concentra en la creación
de los scripts de \textit{Terraform} necesarios para la implementación de la
infraestructura planteada en el apartado \fullref{sec:arquitectura}. Para ello,
se divide el proyecto en scripts separados de manera que se puedan gestionar
los recursos y los servicios de manera independiente.

El diseño de una infraestructura base y el desarrollo de un prototipo de manera
local permiten tener una idea clara de los recursos necesarios y de las
características específicas de cada servicio, facilitándo la tarea de
desarrollo.

Para el desarollo, se hace uso de un repositorio privado en \textit{Bitbucket}
para el control de versiones y facilitar a la empresa la revisión y uso del
código. El código completo se encuentra en \fullref{anexo:cloud}.

Esta sección de la memoria documenta el desarrollo de las siguientes historias
de usuario, siguiendo la planificación establecida en la sección \fullref{sec:planif_inicial}:

\begin{table}[H]
	\centering
	\begin{tabular}{|p{0.7\linewidth}|c|c|}
		\hline
		\textbf{Nombre} & \textbf{Prioridad} & \textbf{Tamaño} \\
		\hline
		\hline
		Como desarrollador de Okticket, quiero que la arquitectura se despliegue y orqueste de manera automática & P0\cellcolor{red!50} & XL\cellcolor{red!50} \\
		\hline
  \end{tabular}
  \caption{Lista de HUs cumplimentadas con el despliegue en la nube}
  \label{tab:impl_cloud}
\end{table}


\newpage{}
\subsection{Proceso de desarrollo}\label{subsec:impl_cloud_desarrollo}
El proceso de desarrollo de los scripts de \textit{Terraform} parte de la
implementación original de la infraestructura en local, y se va adaptando a
las necesidades de la infraestructura en la nube, puesto que ambos comparten
similaridades (como la mayoría de la configuración de los servicios, la
estructura general de los mismos, las imágenes y versiones utilizadas, etc.).

Al igual que con el desarrollo local, se sigue un proceso iterativo, comenzando
por la creación de un solo servicio, en este caso Kafka, y continuando con el
resto de la arquitectura. Los primeros despliegues son tan solo pruebas de
concepto, con el objetivo de adaptarse a la infraestructura de la nube, el
funcionamiento de Terraform y la configuración de AWS.

Pese a que Terraform suele encargarse de la creación, modificación y destrucción
de los recursos de manera automática, existen casos en los que es necesaria la
intervención manual, como en la destrucción de los contenedores de
\textit{Secret Manager} o en la actualización de algunas configuraciones de los
recursos. Estos casos ocurrirán solo durante la fase de desarrollo, puesto que
se espera que, en la fase de producción, no sea necesario la reconfiguración de
los recursos y servicios.

La definición de las tareas de ECS durante el desarrollo queda registrado en la
sección correspondiente de AWS, cuyo código y configuraciones se puede consultar
si así se desea.

\begin{figure}[H]
	\centering
	\includegraphics[width=\textwidth]{impl/definitions.png}
	\caption{Ejemplo de definciones de tareas de ECS en AWS}
	\label{fig:definitions}
\end{figure}

\begin{figure}[H]
	\centering
	\includegraphics[width=\textwidth]{impl/ejemplo_definition.png}
	\caption{Ejemplo de definción de tarea (Kafka)}
	\label{fig:definition}
\end{figure}

Durante el desarrollo del despliegue, se utilizan las herramientas de
monitorización de AWS para comprobar el estado de los recursos y servicios
creados, y se realizan pruebas básicas de funcionamiento para asegurar que los
servicios se han desplegado correctamente. En la figura \ref{fig:task_logs}, se muestran
los logs de una tarea de ECS.

\begin{figure}[H]
	\centering
	\includegraphics[width=\textwidth]{impl/task_logs.png}
	\caption{Ejemplo de logs de una tarea de ECS}
	\label{fig:task_logs}
\end{figure}

En la figura \ref{fig:estado_elb}, se muestra el estado actual de todos los balanceadores
de carga, junto a más detalles como sus zonas de disponibilidad o el estado de
los nodos.

\begin{figure}[H]
	\centering
	\includegraphics[width=\textwidth]{impl/estado_elb.png}
	\caption{Estado de los balanceadores de carga}
	\label{fig:estado_elb}
\end{figure}

Por último, en la figura \ref{fig:deploy_state} se muestra el estado de un
despliegue de un servicio, con información sobre el estado de los contenedores,
la versión de la imagen, la cantidad de tareas en ejecución y la cantidad de
tareas deseadas.

\begin{figure}[H]
	\centering
	\includegraphics[width=\textwidth]{impl/deploy_state.png}
	\caption{Métricas de estado del despliegue de un servicio}
	\label{fig:deploy_state}
\end{figure}

Por supuesto, los propios servicios cuentan con sus herramientas de
monitorización básicas a las que se puede acceder a través del navegador (en
caso de que funcionen correctamente).

\begin{figure}[H]
	\centering
	\includegraphics[width=\textwidth]{impl/kibana_status.png}
	\caption{Estado de Kibana}
	\label{fig:kibana_status}
\end{figure}

Estas herramientas se utilizan para el desarrollo incremental y la comprobación
de que los servicios se despliegan correctamente, y se espera que, en la fase de
producción, no sea necesario su uso, puesto que se cuenta con herramientas de
monitorización más avanzadas y específicas para cada servicio.

\newpage{}
\subsection{Despliegue de la infraestructura}\label{subsec:impl_cloud_despliegue}
% TODO: desarrollar
% Aquí podría estar bien poner un diagrama de despliegue. Pero que se vea en el
% modelo que ese diagrama de despliegue no es el despliegue de tu proyecto.
% Es decir, el proyecto es crear un proceso de despliegue que hace un despliegue.
% Entonces que se vea que hay ese proceso y luego el diagrama de despliegue de lo
% que el proyecto permite desplegar
Para desplegar la infraestructura diseñada en AWS utilizando Terraform, se
siguen los siguientes pasos:

\begin{enumerate}
    \item \textbf{Preparación del entorno:} se asegura que se tienen las
      herramientas y configuraciones necesarias para llevar a cabo el
      despliegue.

    \item \textbf{Inicialización:} se inicializa el directorio de trabajo con
      los módulos y configuraciones necesarios de la herramienta de despliegue.

    \item \textbf{Planificación:} se planifican los cambios que se van a
      realizar en el proveedor en la nube.

    \item \textbf{Aplicación:} se aplican los cambios planificados en el
      proveedor en la nube.

    \item \textbf{Verificación:} una vez completado el despliegue,
      se verifican los recursos y servicios creados para asegurar que se han
      desplegado correctamente.

    \item \textbf{Pruebas:} se realizan pruebas básicas de conectividad y
      funcionalidad para asegurar que la infraestructura está operativa y
      cumple con los requisitos establecidos.
\end{enumerate}

\emph{Ver: \fullref{sec:manual_despliegue}.}

Como se ha mencionado anteriormente, durante cualquier fase del despliegue
pueden ocurrir erorres o problemas que requieran intervención manual. En caso
de que ocurran, se debe revisar el estado de los recursos y servicios en la
consola de AWS, y se debe corregir el problema manualmente si es necesario.

Se mantiene un control de versiones de los archivos de
configuración de Terraform para facilitar el seguimiento de cambios y la
posible futura colaboración en el desarrollo de la infraestructura.

Para visualizar el proceso de despliegue de la infraestructura, se ha creado
un diagrama que ilustra los principales componentes y pasos. Este diagrama se
muestra en la Figura \ref{fig:diagrama_despliegue}.

\begin{figure}[H]
    \centering
    \includegraphics[width=\textwidth]{uml/deploy.png}
    \caption{Diagrama de despliegue de la infraestructura}
    \label{fig:diagrama_despliegue}
\end{figure}

El diagrama \ref{fig:diagrama_despliegue} muestra el flujo desde los archivos de
configuración de Terraform hasta los recursos desplegados en AWS, pasando por
las diferentes etapas del proceso de despliegue.


\newpage{}
\subsection{Explicación del código}\label{sec:impl_configuracion}
\emph{El código completo se encuentra en el anexo \fullref{anexo:cloud}.}

Los scripts de Terraform se dividen en varios archivos, cada uno de ellos con
una función específica, con el objetivo de facilitar la gestión y configuración
de los recursos y servicios. A continuación, se detallan dichos archivos y su
función en el proyecto.


\subsubsection{Recursos generales}
Como previamente descrito, la definción de los recursos generales se divide en
ficheros, cada uno de ellos con una función específica. A continuación, se
detallan dichos ficheros y su función en el proyecto.


\paragraph{Fichero principal}
El fichero principal de Terraform, \halfref{lst:main}{main.tf}, se encarga de
las configuraciones más esenciales o que no tienen cabida en otros ficheros,
como la definción de la región, el cluster, el grupo de logs o el bucket S3 de
logs.


\paragraph{Variables}
El fichero de variables \halfref{lst:variables}{variables.tf}
se encarga de la definición de todas las variables necesarias para el despliegue
de la infraestructura, como nombres, regiones, versiones, puertos, etc., de
manera que se puedan modificar y
reutilizar de manera sencilla a lo largo del resto de ficheros de definción.

También se definen contraseñas de manera aleatoria, para mejorar la seguridad
de los servicios.


\paragraph{Salidas}
El fichero de salidas \halfref{lst:outputs}{outputs.tf} se encarga de la
definición de las salidas de Terraform, es decir, de las variables que se pueden
consultar una vez que se ha desplegado la infraestructura, como las direcciones
URL de los servicios o las contraseñas generadas alteatoriamente.

Un ejemplo de estas salidas se encuentra en la figura
\fullref{fig:terraform_output}.


\newpage{}
\paragraph{Volúmenes lógicos (EFS)}
El fichero de volúmenes lógicos \halfref{lst:efs}{efs.tf} se encarga de la
definción de los volúmenes lógicos necesarios para la persistencia de los datos
de los servicios, como los datos de Elasticsearch o los logs de los servicios.


\paragraph{Roles, políticas y permisos (IAM)}
El fichero de roles, políticas y permisos \halfref{lst:iam}{iam.tf} se encarga
de la definción de los roles y políticas necesarios para el correcto
funcionamiento de los servicios, como los roles de ejecución de tareas de ECS,
los permisos de acceso a los servicios de AWS o las políticas de acceso a los
recursos.


\paragraph{Secretos}
El fichero de secretos \halfref{lst:secrets}{secrets.tf} se encarga de la
definción de las claves necesarias para el correcto funcionamiento de los
servicios, como los certificados de Elasticsearch o las claves de acceso a los
servicios.

Estos secretos se almacenan en \textit{AWS Secrets Manager}, un servicio de AWS
que permite el almacenamiento seguro de información sensible, como contraseñas,
claves de acceso o certificados. Los certificados se almacenan en formato
\texttt{base64} para facilitar su almacenamiento y evitar problemas de
codificación en el paso de los mismos. Se utilizan cadenas generadas
aleatoriamente para evitar la exposición de contraseñas y claves de acceso en el
código.


\paragraph{Redes}
El fichero de redes \halfref{lst:network}{network.tf} se encarga de la definción
de los elementos de red necesarios para la correcta comunicación entre los
servicios y su seguridad.

La lógica y arquitectura de red se explica en el apartado
\fullref{subsec:redes}, lo que facilita la definición de las redes en Terraform.


\paragraph{Grupos de seguridad}
El fichero de grupos de seguridad \halfref{lst:security}{security.tf} se encarga
de la definción de los grupos de seguridad necesarios para la correcta
comunicación entre los servicios y su seguridad.

La lógica y arquitectura de seguridad se explica en el apartado
\fullref{subsec:seguridad}.


\newpage{}
\subsubsection{Servicios de ELK}
Puesto que la estructura de definción de servicios en Terraform es similar para
todos los servicios, se analiza el más completo, Elasticsearch, para explicar el
funcionamiento y la lógica de los mismos.

Cada fichero de definción de servicios está separado en tres partes:
\begin{enumerate}
	\item \textbf{Definción de la tarea}, que contiene la configuración del
		contenedor al estilo de \textit{Docker Compose} (imagen, variables de
    entorno, volúmenes, etc.)
	\item \textbf{Definción del servicio}, que asigna la tarea al resto de la
		configuración del servicio.
	\item \textbf{Definción de recursos}, los necesarios para cada servicio:
		balanceadores de carga, \textit{listeners}, configuraciones DNS, etc.
\end{enumerate}

El código completo de los servicios se encuentra en el anexo \fullref{anexo:cloud}.

\paragraph{Definción de la tarea}
A continuación, se muestra un ejemplo de la definción de la tarea de Elastic:

\begin{lstlisting}[caption={Definción de la tarea de Elastic}, label={lst:elastic_task}]
	resource "aws_ecs_task_definition" "elastic" {
  family                   = "elastic"
  network_mode             = "awsvpc"
  requires_compatibilities = [var.launch_type]
  cpu                      = "2048"
  memory                   = "4096"
  execution_role_arn       = aws_iam_role.ecs_task_execution.arn
  task_role_arn            = aws_iam_role.ecs_task_execution.arn

  container_definitions = jsonencode([
    {
      name      = "es01"
      image     = "docker.elastic.co/elasticsearch/elasticsearch:${var.stack_version}"
      cpu       = 2048
      memory    = 4096
      essential = true
      environment = [
        { name = "cluster.name", value = var.cluster_name },
        { name = "xpack.security.enabled", value = "true" },
        { name = "xpack.security.http.ssl.enabled", value = "true" },
		<...>
      ]
      secrets = [
        {
          name      = "CA_CRT"
          valueFrom = "${aws_secretsmanager_secret.es_certs.arn}:ca.crt::"
        },
        {
          name      = "ES01_KEY"
          valueFrom = "${aws_secretsmanager_secret.es_certs.arn}:es01.key::"
        },
        <...>
      ]
      entrypoint = [
        "/bin/sh",
        "-c",
        <<-EOT
        #!/bin/bash
        set -e

        echo "Configurando credenciales..."
        mkdir -p /usr/share/elasticsearch/config/certs
        echo $CA_CRT | base64 -d > /usr/share/elasticsearch/config/certs/ca.crt
        <...>
        EOT
      ]
      # mountPoints = [
      #   {
      #     sourceVolume  = "elastic-data"
      #     containerPath = "/usr/share/elasticsearch/data"
      #     readOnly      = false
      #   }
      # ]
      portMappings = [
        {
          containerPort = var.elastic_port,
          hostPort      = var.elastic_port
        }
      ]
      logConfiguration = {
        logDriver = var.log_driver
        options = {
          "awslogs-group"         = var.log_group
          "awslogs-region"        = var.region
          "awslogs-stream-prefix" = "elastic"
        }
      }
      ulimits = [
        {
          name      = "memlock"
          softLimit = -1
          hardLimit = -1
        },
        {
          name      = "nofile"
          softLimit = 65536
          hardLimit = 65536
        }
      ]
    }
  ])

  volume {
    name = "elastic-data"
    efs_volume_configuration {
      file_system_id = aws_efs_file_system.elastic_data.id
      root_directory = "/"
    }
  }
}
\end{lstlisting}

Como se puede comprobar, pese a que la sintaxis de configuración es distinta,
la estructura de la misma es muy similar a aquella ya definida durante el
\fullref{sec:impl_local}, con similares configuraciones de entorno, mapeos de
puertos y volúmenes, y configuraciones de red. Ese es el objetivo del
planteamiento iterativo y de desarrollo incremental, que permite reutilizar
código y configuraciones ya definidas.

Al igual que en el desarrollo local, se obvia la configuración de los recursos
necesaria para la escalabilidad y replicación de los servicios, puesto que de
momento no se necesitan más que una instancia de cada servicio. Sin embargo,
esto no significa que no se pueda añadir de manera sencilla en un futuro, como
así lo demuestran, por ejemplo, la asignación de los servicios a volúmenes EFS.

A diferencia del desarrollo local, la preparación del servicio se realiza desde
la declaración del \texttt{entrypoint}, en lugar de requerir un segundo
contenedor de configuración.


\newpage{}
\paragraph{Definción del servicio}
A continuación, se detalla la definición del servicio de Elasticsearch:

\begin{lstlisting}[caption={Definción del servicio de Elastic}, label={lst:elastic_service}]
	resource "aws_ecs_service" "elastic" {
  name                   = "elastic"
  cluster                = aws_ecs_cluster.cluster.id
  task_definition        = aws_ecs_task_definition.elastic.arn
  desired_count          = 1
  launch_type            = var.launch_type
  enable_execute_command = true

  network_configuration {
    subnets          = [aws_subnet.private.id]
    security_groups  = [aws_security_group.elastic.id]
    assign_public_ip = true
  }

  load_balancer {
    target_group_arn = aws_lb_target_group.elastic.arn
    container_name   = "es01"
    container_port   = var.elastic_port
  }

  depends_on = [
    aws_lb_listener.elastic,
    aws_ecs_task_definition.elastic
  ]
}
\end{lstlisting}

La definción del servicio es mucho más breve que el resto de definiciones,
puesto que la mayoría de la configuración se realiza en la tarea. En este caso,
se asigna la tarea a un grupo de seguridad y a una subred privada, y se asigna
el servicio a un balanceador de carga, que se encargará de distribuir el tráfico
entre los contenedores.


\newpage{}
\paragraph{Definción de recursos}
Todos los recursos necesarios para el servicio, como el balanceador de carga, el
grupo de seguridad, el \textit{listener} o la configuración DNS, se definen a
continuación.

\begin{lstlisting}[caption={Definción de recursos de Elastic}, label={lst:elastic_resources}]
resource "aws_lb" "elastic" {
  name               = "tahoe-alb-elastic"
  internal           = false
  load_balancer_type = "application"
  security_groups    = [aws_security_group.elastic.id]
  subnets            = [aws_subnet.public_a.id, aws_subnet.public_b.id]

  access_logs {
    bucket  = aws_s3_bucket.access_logs.bucket
    prefix  = "elastic"
    enabled = true
  }
}

resource "aws_lb_target_group" "elastic" {
  name        = "tahoe-tg-elastic"
  port        = var.elastic_port
  protocol    = "HTTPS"
  vpc_id      = aws_vpc.main.id
  target_type = "ip"

  health_check {
    enabled             = true
    path                = "/"
    protocol            = "HTTPS"
    matcher             = "200"
    interval            = 300
    timeout             = 60
    healthy_threshold   = 2
    unhealthy_threshold = 5
    port                = tostring(var.elastic_port)
  }
}

resource "aws_lb_listener" "elastic" {
  load_balancer_arn = aws_lb.elastic.arn
  port              = var.elastic_port
  protocol          = "HTTPS"
  ssl_policy        = "ELBSecurityPolicy-2016-08"
  certificate_arn   = aws_acm_certificate.elastic.arn

  default_action {
    type             = "forward"
    target_group_arn = aws_lb_target_group.elastic.arn
  }
}

resource "aws_lb_listener" "elastic_https" {
  load_balancer_arn = aws_lb.elastic.arn
  port              = 443
  protocol          = "HTTPS"
  ssl_policy        = "ELBSecurityPolicy-2016-08"
  certificate_arn   = aws_acm_certificate.elastic.arn

  default_action {
    type = "redirect"
    redirect {
      port        = tostring(var.elastic_port)
      protocol    = "HTTPS"
      status_code = "HTTP_301"
    }
  }
}

resource "aws_acm_certificate" "elastic" {
  domain_name       = local.elastic_url
  validation_method = "DNS"

  lifecycle {
    create_before_destroy = true
  }
}

resource "aws_route53_record" "elastic" {
  zone_id = var.route53_zone_id
  name    = local.elastic_url
  type    = "A"

  alias {
    name                   = aws_lb.elastic.dns_name
    zone_id                = aws_lb.elastic.zone_id
    evaluate_target_health = false
  }
}

resource "aws_route53_record" "elastic_validation" {
  for_each = {
    for dvo in aws_acm_certificate.elastic.domain_validation_options : dvo.domain_name => {
      name   = dvo.resource_record_name
      record = dvo.resource_record_value
      type   = dvo.resource_record_type
    }
  }

  allow_overwrite = true
  name            = each.value.name
  records         = [each.value.record]
  ttl             = 60
  type            = each.value.type
  zone_id         = var.route53_zone_id
}

resource "aws_acm_certificate_validation" "elastic" {
  certificate_arn         = aws_acm_certificate.elastic.arn
  validation_record_fqdns = [for record in aws_route53_record.elastic_validation : record.fqdn]
}
\end{lstlisting}

Como para todos los servicios, se necesitan definir los recursos de red que
permitan acceder a los mismos, es decir:

\begin{itemize}
	\item Un balanceador de carga que distribuya el tráfico entre los
		contenedores de Elasticsearch.
	\item Un \textit{target group} que asigne los contenedores al balanceador de
		carga y permita la comprobación de la salud de los contenedores.
	\item \textit{Listeners} que permitan el acceso a los servicios desde el
		exterior o redirijan el tráfico a los contenedores dependiendo del
		puerto de conexión.
\end{itemize}

En el caso de Elastic (y el de otros servicios como Kibana), también se requiere
la configuración de un subdominio DNS y sus certificados pertinentes, para
facilitar el acceso a los servicios desde el exterior. Sin dichos subdominios,
la dirección URL de los servicios cambiaría cada vez que se desplegaran, lo que
dificultaría el acceso a los mismos. El tiempo de vida (\textit{TTL}) de estas
definiciones es de un minuto por defecto, lo que permite la actualización de
los certificados y la dirección URL de manera rápida y sencilla.


\newpage{}
\input{sections/73_ingesta.tex}

\newpage{}
\section{Visualización de datos}\label{sec:impl_visualizacion}
Una vez se cuentan con datos en Elasticsearch, se puede comenzar el desarrollo
de la visualización de los mismos mediante Kibana. Para ello, se han desarrollado
paneles de visualización que permiten la monitorización de los datos en tiempo
real y la creación de informes y \textit{dashboards} personalizados para cada
modelo de datos contemplado.

Esta sección de la memoria documenta el desarrollo de las siguientes historias
de usuario, siguiendo la planificación establecida en la sección \fullref{sec:planif_inicial}:

\begin{table}[H]
	\centering
	\begin{tabular}{|p{0.7\linewidth}|c|c|}
		\hline
		\textbf{Nombre} & \textbf{Prioridad} & \textbf{Tamaño} \\
		\hline
		\hline
		Como trabajador de Okticket, quiero poder ver y consultar datos internos de la empresa & P1\cellcolor{orange!50} & M\cellcolor{yellow!50} \\
		\hline
		Como desarrollador de Okticket, quiero poder ver el estado general de la infraestructura & P1\cellcolor{orange!50} & M\cellcolor{yellow!50} \\
		\hline
  \end{tabular}
  \caption{Lista de HUs cumplimentadas con la visualización de datos}
  \label{tab:impl_visualizacion}
\end{table}


\newpage{}
\subsection{Desarrollo de dashboards}
Para el desarrollo de los dashboards, se han seguido los siguientes pasos:

\begin{enumerate}
    \item \textbf{Identificación de métricas clave}: se identifican las
    métricas más relevantes incluyendo: número de solicitudes, latencia,
	códigos de estados, tipos de errores, etc.

    \item \textbf{Diseño de visualizaciones}: Para cada métrica identificada,
    se decide el tipo de visualización más adecuada. Por ejemplo, gráficos
	temporales para métricas que varían con el tiempo, mapas para visualizar
	distribuciones geográficas, etc.

    \item \textbf{Creación de paneles}: se crean dos paneles principales:
    \begin{itemize}
        \item \textbf{Dashboard de estado general de infraestructura}: Incluye
        visualizaciones del tráfico total, distribución de carga, y estado de
        salud del servicio dependiendo de la infraestructura.
        \item \textbf{Dashboard de métricas detalladas}: muestra información
        más específica como latencias por URL, análisis de errores, tendencias
		temporales, etc.
    \end{itemize}

    \item \textbf{Configuración de filtros y controles}: gracias al sistema de
		filtros y controles de Kibana, los usuarios pueden personalizar las
		visualizaciones para ver solo la información relevante, además de
		ajustar el rango temporal de los datos. (ver \fullref{sec:manual_usuario}).
\end{enumerate}

\subsubsection{Personalización y acceso}
Actualmente, todos los usuarios tienen acceso completo a los dashboards
desarrollados. Se ha identificado la necesidad futura de implementar un sistema
de gestión de usuarios y permisos en Kibana. Esto permitirá asegurar que cada
usuario tenga acceso solo a los dashboards y datos relevantes para su función,
especialmente importante para la visualización de métricas sensibles de los
balanceadores de carga.

La implementación de este sistema de permisos, así como la capacidad de que los
usuarios puedan personalizar sus propios dashboards, se ha planificado como una
tarea para futuras iteraciones del proyecto.


\newpage{}
\subsection{Ejemplos de dashboards}
\begin{figure}[H]
	\centering
	\includegraphics[width=\textwidth]{impl/dashboard_a.png}
	\caption{Dashboard de estado general de infraestructura}
	\label{fig:dashboard_a}
\end{figure}

\begin{figure}[H]
	\centering
	\includegraphics[width=\textwidth]{impl/dashboard_b.png}
	\caption{Dashboard de métricas detalladas}
	\label{fig:dashboard_b}
\end{figure}


\chapter{Pruebas}
Las pruebas son una parte fundamental del desarrollo de software, ya que
permiten verificar el correcto funcionamiento del sistema y detectar posibles
errores antes de su implementación en un entorno de producción.

Con el objetivo de mejorar la calidad del sistema y tratar de encontrar posibles
defectos, se ha diseñado un plan de pruebas que permita verificar y
validar las diferentes funcionalidades del data lake. Este plan incluye pruebas
de despliegue, ingesta de datos, visualización y rendimiento, abarcando los
aspectos más críticos del sistema.

El plan de pruebas se ha diseñado siguiendo la técnica de clases de
equivalencia, que permite reducir el número de casos de prueba necesarios
mientras se mantiene una cobertura adecuada. Se han considerado diferentes
aspectos del sistema, desde el despliegue inicial hasta el rendimiento bajo
carga.


\newpage{}
\section{Plan de pruebas}
El plan de pruebas se divide en dos categorías principales: pruebas funcionales,
que evalúan el comportamiento del sistema en función de sus especificaciones,
y pruebas no funcionales, que evalúan aspectos como la calidad del código y el
rendimiento del sistema.


\subsection{Pruebas funcionales}
El plan de pruebas funcional se ha dividido en tres categorías principales,
cada una enfocada en un aspecto crítico del sistema:

\begin{enumerate}
    \item Pruebas de despliegue
    \item Pruebas de ingesta de datos
    \item Pruebas de visualización
\end{enumerate}

A continuación, se detalla cada una de estas categorías, especificando los
objetivos, las clases de equivalencia consideradas y los casos de prueba
propuestos.


\newpage{}
\subsubsection{Pruebas de despliegue}
Las pruebas de despliegue son cruciales para asegurar que el sistema puede ser
implementado correctamente en diferentes escenarios. Estas pruebas verifican
la robustez del proceso de despliegue y su capacidad para manejar diversas
condiciones iniciales.

\textbf{Objetivo:} Verificar que el sistema se despliega correctamente en
diferentes escenarios.

\paragraph{Clases de equivalencia}
\begin{itemize}
    \item Estado del sistema: Sin desplegar, en progreso de despliegue,
		desplegado
    \item Existencia de volúmenes: Sin volúmenes virtuales, con volúmenes
		virtuales montados
    \item Estado de puertos: Libres, Ocupados
    \item Cuenta en AWS: Sin cuenta configurada, con cuenta configurada
    incorrectamente, con cuenta configurada correctamente
\end{itemize}

\paragraph{Casos de prueba}
\begin{enumerate}
    \item \textbf{DP-01:} Despliegue desde cero con todos los recursos libres
    \item \textbf{DP-02:} Intento de despliegue con sistema ya desplegado
    \item \textbf{DP-03:} Despliegue con volúmenes existentes y montados
    \item \textbf{DP-04:} Despliegue con puertos ocupados (\textcolor{red}{Inválido})
    \item \textbf{DP-05:} Despliegue sin cuenta de AWS configurada (\textcolor{red}{Inválido})
    \item \textbf{DP-06:} Despliegue con cuenta de AWS mal configurada (\textcolor{red}{Inválido})
\end{enumerate}


\newpage{}
\subsubsection{Pruebas de ingesta de datos}
La ingesta de datos es una funcionalidad central del data lake. Estas pruebas
están diseñadas para verificar la capacidad del sistema para procesar datos de
diferentes fuentes, volúmenes y frecuencias.

\textbf{Objetivo:} Comprobar que el sistema ingesta correctamente datos de
diferentes fuentes.

\paragraph{Clases de equivalencia}
\begin{itemize}
    \item Tipo de fuente: MySQL, MongoDB, Logs de Laravel, Logs de AWS ELB
    \item Volumen de datos: Pequeño ($<1000$ registros), Mediano ($1000-100000$
    	registros), Grande ($>100000$ registros)
    \item Frecuencia de ingesta: Baja (cada hora), Media (cada minuto), Alta
    	(tiempo real)
\end{itemize}

\paragraph{Casos de prueba}
\begin{enumerate}
    \item \textbf{IN-01:} Ingesta de pequeño volumen de logs de MySQL
    \item \textbf{IN-02:} Ingesta de gran volumen de datos de MongoDB
    \item \textbf{IN-03:} Ingesta en tiempo real de logs de Laravel
    \item \textbf{IN-04:} Ingesta periódica de logs de AWS ELB
\end{enumerate}


\newpage{}
\subsubsection{Pruebas de visualización}
La visualización efectiva de los datos es esencial para que los usuarios
puedan obtener insights valiosos. Estas pruebas evalúan la capacidad básica
de Kibana para visualizar los datos ingestados.

\textbf{Objetivo:} Verificar que los datos ingestados se visualizan
correctamente en Kibana.

\paragraph{Clases de equivalencia}
\begin{itemize}
    \item Tipo de visualización: Gráficos de barras, gráficos de líneas,
        tablas, paneles de control
    \item Complejidad de la consulta: Simple, ``compleja''
    \item Volumen de datos visualizados: Pequeño, ``grande''
\end{itemize}

\paragraph{Casos de prueba}
\begin{enumerate}
    \item \textbf{VIS-01:} Creación de un gráfico de barras simple con pocos datos
    \item \textbf{VIS-02:} Generación de un panel de control complejo con datos de múltiples fuentes
    \item \textbf{VIS-03:} Visualización de una tabla con gran volumen de datos
\end{enumerate}


\newpage{}
\subsection{Pruebas no funcionales}
Además de las pruebas funcionales, se han diseñado pruebas no funcionales para
evaluar aspectos críticos del sistema que no están directamente relacionados
con su funcionalidad, como la calidad del código y el rendimiento.

\subsubsection{Análisis estático de código}
El análisis estático de código es una técnica que permite detectar posibles
errores y problemas en el código fuente sin necesidad de ejecutarlo. Estas
pruebas evalúan la calidad del código y su cumplimiento de las normas de
programación.

En el caso de este proyecto, se analiza de manera constante el código fuente
mediante SonarQube, una herramienta de análisis estático de código que permite
identificar problemas de calidad y seguridad en el código.

\begin{figure}[H]
	\centering
	\includegraphics[width=\textwidth]{pruebas/sonar_gate.png}
	\caption{Análisis estático de código con SonarQube}
	\label{fig:sonarqube}
\end{figure}

Se utiliza la máquina especializada de SonarQube de la empresa, desplegada y
mantenida por el alumno, para analizar el código de manera automatizada mediante
las \textit{pipelines} de Bitbucket cada vez que se publica un cambio al
repositorio.


\newpage{}
\subsubsection{Pruebas de rendimiento}
El rendimiento del sistema es crítico, especialmente cuando se manejan grandes
volúmenes de datos o múltiples consultas simultáneas. Estas pruebas evalúan
cómo responde el sistema bajo diferentes condiciones de carga.

Para realizar estas pruebas de carga, se manipulan los scripts de ingesta de
carga (ver \fullref{anexo:kafka}) y se utiliza la herramienta de estrés de
carga \textit{Locust}, otra herramienta desplegada y mantenida por el alumno
dentro de la nube de servicios de la empresa. Al crear picos de carga en el
\textit{backend}, se aumenta la cantidad de registros en \textit{AWS CloudWatch},
que a su vez se ingestan a través del sistema establecido.

También se prueba la capacidad de respuesta del sistema mediante la ejecución
de múltiples consultas simultáneas en Kibana mediante múltiples navegadores y
usuarios.

\textbf{Objetivo:} Evaluar el rendimiento del sistema bajo diferentes cargas.

\paragraph{Clases de equivalencia}

\begin{itemize}
    \item Carga de ingesta: Baja ($<100$ eventos/min), media ($100-1000$ eventos/min),
    	alta ($>=1000$ eventos/min)
    \item Carga de consultas: Pocas consultas simultáneas ($<10$), Muchas
    consultas simultáneas ($>=10$)
\end{itemize}

\paragraph{Casos de prueba}
\begin{enumerate}
    \item \textbf{PERF-01:} Ingesta de alta carga de eventos durante 1 hora
    \item \textbf{PERF-02:} Ejecución de múltiples consultas complejas simultáneas
    \item \textbf{PERF-03:} Combinación de alta carga de ingesta y consultas simultáneas
\end{enumerate}


\newpage{}
\section{Ejecución de pruebas}
La ejecución de las pruebas se lleva a cabo de manera sistemática, siguiendo
el plan establecido. Para cada caso de prueba, se documentaron los siguientes
aspectos:

\begin{enumerate}
    \item Pasos de ejecución
    \item Resultado esperado
    \item Resultado obtenido
    \item Estado (Pasado/Fallido)
\end{enumerate}

A continuación, se presenta un ejemplo de ejecución de una prueba, incluyendo
los pasos y resultados esperados y obtenidos, y posteriormente se presenta una
tabla resumen con los resultados de todas las pruebas realizadas.


\newpage{}
\subsection{Ejemplo de ejecución}
\textbf{Despliegue desde cero con todos los recursos libres (DP-01)}
Esta prueba es la más básica de las pruebas de despliegue, y tiene como
objetivo verificar que el sistema se puede desplegar correctamente en un
entorno limpio, sin recursos previamente desplegados.

\subsubsection{Pasos de ejecución}
\begin{itemize}
    \item Asegurar que no hay recursos desplegados previamente
    \item Configurar correctamente las credenciales de AWS
    \item Ejecutar el comando \texttt{terraform init}
    \item Ejecutar el comando \texttt{terraform apply}
\end{itemize}

\subsubsection{Resultado esperado}
\begin{itemize}
    \item Terraform debe completar el despliegue sin errores
    \item Todos los recursos (ECS, Elasticsearch, Kibana, Logstash, Kafka)
    deben estar en estado "running"
    \item Se deben poder acceder a las interfaces web de Kibana y Elasticsearch
\end{itemize}

\subsubsection{Resultado obtenido}
\begin{itemize}
    \item Terraform completó el despliegue sin errores
    \item Todos los recursos se desplegaron correctamente y están en estado
    "running"
    \item Se pudo acceder a Kibana y Elasticsearch a través de sus respectivas
    URLs
\end{itemize}

\subsubsection{Estado} \textcolor{green}{Pasado}

Este resultado exitoso indica que el proceso de despliegue automatizado
funciona correctamente y es capaz de configurar todos los componentes
necesarios del sistema de manera eficiente.


\newpage{}
\subsection{Tabla de resultados}
A continuación, se presenta una tabla resumen con los resultados de la
ejecución de todas las pruebas planificadas, incluyendo las pruebas de
despliegue, ingesta de datos, visualización y rendimiento.

\begin{table}[h]
    \centering
    \begin{tabular}{|p{1.5cm}|p{5cm}|p{5cm}|p{2cm}|}
        \hline
        \textbf{ID} & \textbf{Prueba} & \textbf{Resultado} & \textbf{Estado} \\
        \hline
        DP-01 & Despliegue desde cero &
        Recursos desplegados y accesibles &
        \cellcolor{green!25}OK \\
        \hline
        DP-02 & Redespliegue con sistema existente &
        Sin cambios realizados &
        \cellcolor{green!25}OK \\
        \hline
        DP-03 & Despliegue con volúmenes existentes &
        Volúmenes reutilizados &
        \cellcolor{green!25}OK \\
        \hline
        DP-04 & Despliegue con puertos ocupados &
        Error de puertos mostrado &
        \cellcolor{green!25}Error (OK) \\
        \hline
        DP-05 & Despliegue sin cuenta AWS &
        Error de credenciales &
        \cellcolor{green!25}Error (OK) \\
        \hline
        DP-06 & Despliegue con cuenta AWS incorrecta &
        Error de credenciales &
        \cellcolor{green!25}Error (OK) \\
        \hline
        IN-01 & Ingesta logs MySQL &
        Logs en Elasticsearch &
        \cellcolor{green!25}OK \\
        \hline
        IN-02 & Ingesta masiva MongoDB &
        Datos ingestados sin pérdidas &
        \cellcolor{green!25}OK \\
        \hline
        IN-03 & Ingesta tiempo real Laravel &
        Logs visibles, retraso $<5$s &
        \cellcolor{green!25}OK \\
        \hline
        IN-04 & Ingesta periódica AWS ELB &
        Logs cada 5min &
        \cellcolor{green!25}OK \\
        \hline
        \hline
        VIS-01 & Gráfico de barras simple &
        Gráfico creado correctamente &
        \cellcolor{green!25}OK \\
        \hline
        VIS-02 & Panel control complejo &
        Panel con 4 fuentes de datos &
        \cellcolor{green!25}OK \\
        \hline
        VIS-03 & Tabla gran volumen &
        1M registros en $<3$s &
        \cellcolor{green!25}OK \\
        \hline
        \hline
        PERF-01 & Ingesta alta carga 1h &
        1M eventos/min sin pérdidas &
        \cellcolor{green!25}OK \\
        \hline
        PERF-02 & Consultas complejas simultáneas &
        20 consultas, respuesta $<2$s &
        \cellcolor{green!25}OK \\
        \hline
        PERF-03 & Carga mixta &
        500k/min + 10 consultas, $<3$s &
        \cellcolor{green!25}OK \\
        \hline
    \end{tabular}
    \caption{Ejecución de pruebas del sistema}
    \label{tab:pruebas}
\end{table}

La ejecución completa de las pruebas se realiza una vez completada la
implementación del sistema, pero son pruebas similares a las que se han
realizado durante el desarrollo del mismo, con el objetivo de asegurar que el
sistema cumple con los requisitos y especificaciones establecidas.

\chapter{Manuales}\label{chap:manual}
\input{sections/91_usuario.tex}

\newpage{}
\input{sections/92_install.tex}

\chapter{Resultados y trabajo futuro}
El propósito de este capítulo es presentar las conclusiones obtenidas a partir
del desarrollo del proyecto, recopilar las dificultades encontradas y proponer
líneas de trabajo futuro en vista a la amplicación y mejora del sistema.

\section{Resultados}
El resultado del proyecto es un sistema de monitorización y análisis de datos
funcional y escalable, que permite a los usuarios obtener insights valiosos a
partir de la información recopilada.

Se ha logrado implementar una arquitectura robusta y flexible, basada en
tecnologías modernas y en la nube, que facilita la ingesta, procesamiento y
visualización de datos de manera eficiente y sencilla.

El sistema es capaz de ingestar datos de diversas fuentes, como bases de datos
internas, logs de aplicaciones y servicios externos, y de presentarlos de forma
clara y útil a través de Kibana.

La infraestructura se despliega y orquesta de manera automática en la nube,
permitiendo una gestión sencilla y eficiente del sistema para los
administradores.

Pese a que no se han completado todas las historias de usuario planificadas,
se han logrado los objetivos principales del proyecto, sentando las bases para
futuras iteraciones y logrando así entregar un producto mínimo viable (MVP) de
calidad.


\newpage{}
\section{Trabajo futuro}
El proyecto ha sentado las bases para un sistema de monitorización y análisis de
datos robusto y escalable. Sin embargo, existen áreas de mejora y ampliación que
podrían ser abordadas en futuras iteraciones.


\subsection{Historias de usuario restantes}
Lo primero de todo sería completar las historias de usuario menos críticas que
han quedado pendientes, como la ingesta de datos de APIs externas o el
\textit{web scraping}. Estas funcionalidades permitirían enriquecer los datos
disponibles y ampliar las fuentes de información.

\begin{table}[H]
	\centering
	\begin{tabular}{|p{0.7\linewidth}|c|c|}
		\hline
		\textbf{Nombre} & \textbf{Prioridad} & \textbf{Tamaño} \\
		\hline
		\hline
		Como desarrollador de Okticket, quiero que los datos contengan metadatos que faciliten su filtrado o búsqueda & P2\cellcolor{yellow!50} & S\cellcolor{green!25} \\
		\hline
		Como trabajador de Okticket, quiero poder ver y consultar datos de empresas cliente & P2\cellcolor{yellow!50} & M\cellcolor{yellow!50} \\
		\hline
		Como gestor de una empresa cliente, quiero poder ver información relevante sobre mi empresa que recoja Okticket & P2\cellcolor{yellow!50} & L\cellcolor{orange!50} \\
		\hline
		Como desarrollador de Okticket, quiero poder ingestar datos de APIs externas a la empresa & P2\cellcolor{yellow!50} & L\cellcolor{orange!50} \\
		\hline
		Como desarrollador de Okticket, quiero poder ingestar información de páginas web externas (\textit{scraping}) & P2\cellcolor{yellow!50} & XL\cellcolor{red!50} \\
		\hline
	\end{tabular}
	\caption{Historias de usuario restantes para futuras iteraciones}
	\label{tab:remaining_tasks}
\end{table}


\newpage{}
\subsection{Integración de lenguaje natural para búsqueda (DSL)}
Sería interesante explorar la posibilidad de integrar el sistema con
un sistema de búsqueda y filtrado de datos a través de lenguaje natural, como
\textit{Natural Language Processing} (NLP)\footnote{
	\url{https://www.elastic.co/guide/en/elasticsearch/reference/current/query-dsl-query-string-query.html}
}

Esto permitiría a los usuarios realizar consultas de manera más intuitiva y
eficiente, sin necesidad de conocer la sintaxis de Kibana Query Language (KQL).


\subsection{Aplicación de modelos de Lenguaje de Gran Escala (LLM)}
Desde la empresa, se ha propuesto la posibilidad de aplicar modelos de LLM
para la generación de texto automática, presumiblemente de manera agnóstica al
proveedor (OpenAI, Anthropic, Google\ldots).

Esto permitiría la generación de informes y análisis de manera automática a
partir de los datos recopilados, facilitando la toma de decisiones y la
comunicación de insights a los usuarios.

\subsection{Más perspectivas futuras}
Gracias a la flexibilidad del stack ELK, se podrían añadir nuevas fuentes de
datos y visualizaciones, como logs de otras partes de la aplicación (gestor web,
otros servicios internos como Hubspot o Holded, aplicación móvil\ldots) o
visualizaciones más avanzadas y personalizadas.

Lo bueno de haber diseñado la arquitectura de la manera que se ha hecho
es que se pueden añadir nuevas funcionalidades sin necesidad de modificar
la infraestructura existente, simplemente añadiendo nuevos servicios y
configuraciones.

La escalabilidad del sistema también es un punto a tener en cuenta. En caso de
necesitar más capacidad de procesamiento o almacenamiento, se podría establecer
un sistema de autoescalabilidad en base a las definiciones ya realizadas.

Por último, se podría explotar la funcionalidad del stack de generar alertas
en base a la información ingestada y procesada, para notificar a los usuarios
de eventos críticos o anomalías detectadas en los datos.


\newpage{}
\section{Conclusiones y retrospectiva}


%% Anexos
\clearpage
\addcontentsline{toc}{chapter}{Anexos}
\appendix % Inicia los anexos
\chapter*{Anexos}
\chapter{Código de despliegue local}\label{anexo:local}
\lstinputlisting[style=yaml]{input/docker-compose.yml}
\lstinputlisting{input/.env.example}

\chapter{Script de creación de certificados}\label{anexo:certificados}
\lstinputlisting[language=bash, caption={Script de creación de credenciales}]{input/certs.sh}
% TODO: fix estilo bash no funcional

\chapter{Scripts de despliegue cloud}\label{anexo:cloud}
\section{Recursos de AWS}
\subsection{Fichero principal}
\lstinputlisting[caption={Fichero \emph{main.tf} de despliegue cloud}, label={lst:main}]{input/tf/main.tf}
\newpage{}
\subsection{Variables}
\lstinputlisting[caption={Fichero \emph{variables.tf} de despliegue cloud}, label={lst:variables}]{input/tf/variables.tf}
\newpage{}
\subsection{Salidas}
\lstinputlisting[caption={Fichero \emph{outputs.tf} de despliegue cloud}, label={lst:outputs}]{input/tf/outputs.tf}
\newpage{}
\subsection{Volúmenes lógicos (EFS)}
\lstinputlisting[caption={Fichero \emph{efs.tf} de despliegue cloud}, label={lst:efs}]{input/tf/efs.tf}
\newpage{}
\subsection{Roles, usuarios y políticas (IAM)}
\lstinputlisting[caption={Fichero \emph{iam.tf} de despliegue cloud}, label={lst:iam}]{input/tf/iam.tf}
\newpage{}
\subsection{Secretos (Secret Manager)}
\lstinputlisting[caption={Fichero \emph{secrets.tf} de despliegue cloud}, label={lst:secrets}]{input/tf/secrets.tf}
\newpage{}
\subsection{Recursos de red}
\lstinputlisting[caption={Fichero \emph{network.tf} de despliegue cloud}, label={lst:network}]{input/tf/network.tf}
\newpage{}
\subsection{Grupos de seguridad (SG)}
\lstinputlisting[caption={Fichero \emph{secrets.tf} de despliegue cloud}, label={lst:security}]{input/tf/security.tf}

\newpage{}
\section{Servicios de ELK}
\subsection{Elasticsearch}
\lstinputlisting[caption={Fichero \emph{elastic.tf} de despliegue cloud}, label={lst:elastic}]{input/tf/elastic.tf}
\newpage{}
\subsection{Kibana}
\lstinputlisting[caption={Fichero \emph{kibana.tf} de despliegue cloud}, label={lst:kibana}]{input/tf/kibana.tf}
\newpage{}
\subsection{Logstash}
\lstinputlisting[caption={Fichero \emph{logstash.tf} de despliegue cloud}, label={lst:logstash}]{input/tf/logstash.tf}
\newpage{}
\subsection{Kafka}
\lstinputlisting[caption={Fichero \emph{kafka.tf} de despliegue cloud}, label={lst:kafka}]{input/tf/kafka.tf}

\chapter{Scripts de ingesta de datos con Kafka}\label{anexo:kafka}
\section{Ingesta de logs de Laravel}
\lstinputlisting[language=python, caption={Script de ingesta de logs de Laravel}]{input/kafka/laravel.py}
\lstinputlisting[caption={Fichero \texttt{.env} de configuración de ingesta de logs de Laravel}]{input/kafka/laravel.env}

\newpage{}
\section{Ingesta de logs de MongoDB}
\lstinputlisting[language=python, caption={Script de ingesta de logs de MongoDB}]{input/kafka/mongo.py}
\lstinputlisting[caption={Fichero \texttt{.env} de configuración de ingesta de logs de MongoDB}]{input/kafka/mongo.env}

\newpage{}
\section{Ingesta de logs de MySQL}
\lstinputlisting[language=python, caption={Script de ingesta de logs de MySQL}]{input/kafka/mysql.py}
\lstinputlisting[caption={Fichero \texttt{.env} de configuración de ingesta de logs de MySQL}]{input/kafka/mysql.env}

\chapter{Código de ingesta de logs de ELB (AWS)}\label{anexo:elb}
\section{Lambda de ingesta}
\lstinputlisting[language=python, caption={Lambda de ingesta de logs de ELB}]{input/elb/lambda.py}

\newpage{}
\section{Filtro de suscripción}
\lstinputlisting[language=bash, caption={Comando para añadir el filtro de suscripción}]{input/elb/filter.sh}

\newpage{}
\section{Configuración de Logstash}
\lstinputlisting[caption={Configuración de Logstash para ingesta de logs de ELB}]{input/elb/logstash.conf}

\newpage{}
\section{Creación del índice}
\lstinputlisting[caption={Estructura del índice de Elasticsearch}]{input/elb/index.json}


%% Esto incluirá la bibliografía correctamente en nuestro trabajo
\newpage % En una nueva página
\addcontentsline{toc}{chapter}{Bibliografía} % Añade la referencia al índice de contenido
\bibliographystyle{ieeetr} % Define el estilo de la bibliografía
\bibliography{biblio} % Indica el archivo que contiene la colección de citas

\nocite{template}
\nocite{mier2024anomalias}

\end{document}
