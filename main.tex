\documentclass[12pt, oneside, openany]{book}

\usepackage{mathptmx} % Contiene una fuente similar a Times New Roman

\usepackage[spanish, es-tabla]{babel} % Permite escritura en castellano
\usepackage[utf8]{inputenc} % Permite utilizar caracteres UTF8

\usepackage{graphicx} % Para la inclusión de gráficos e imágenes
\graphicspath{ {images/} } % Ruta para buscar las imágenes
\usepackage[a4paper,top=30mm,left=30mm,right=25mm,bottom=25mm,headheight=20mm]{geometry} % Configuración de los margenes de la página

\PassOptionsToPackage{table}{xcolor}

% Paquetes para que funcione el formato.
\usepackage{xcolor}
\usepackage{plantuml}
\usepackage{tikz}
\usepackage[toc,page]{appendix}
\usepackage{titlesec}
\usepackage{multirow}
\usepackage{setspace}
\usepackage{ragged2e}
\usepackage{fancyhdr}
\usepackage{lastpage}
\usepackage{stackengine}
\usepackage{array}
\usepackage[hidelinks]{hyperref}
\usepackage{enumitem}
\usepackage{float}
\usepackage{hypcap}
\usepackage{caption}
\usepackage{fancyvrb}
\usepackage{textcomp}
\usepackage{listings}

\pgfplotsset{compat=1.18}
\usepgfplotslibrary{dateplot} % Se define un color gris desde su código RGB
\definecolor{gris}{RGB}{220,220,220}

\setcounter{secnumdepth}{3} % Para permitir numerar las sub-subsecciones

% Modifica el nombre de los índices al castellano
\addto\captionsspanish{
  \renewcommand{\contentsname}{Índice de contenido}
  \renewcommand{\listfigurename}{Índice de figuras}
  \renewcommand{\listtablename}{Índice de tablas}
  \renewcommand{\lstlistingname}{Listado}
  \renewcommand{\lstlistlistingname}{Índice de listados}
}

% Formateo de los nombres de los apartados:
\titleformat{\chapter}[block]
{\normalfont\Huge\bfseries\singlespacing}{\thechapter.}{1em}{\Huge}
\titlespacing*{\chapter}{0pt}{-62pt}{0pt}

\titleformat{\section}[block]
{\normalfont\Large\bfseries}{\thesection.}{4pt}{\Large}
\titlespacing*{\section}{0pt}{\baselineskip}{0pt}

\titleformat{\subsection}[block]
{\normalfont\large\bfseries}{\thesubsection.}{4pt}{\normalsize\large}
\titlespacing*{\subsection}{0pt}{0pt}{0pt}

\titleformat{\subsubsection}[block]
{\normalfont\normalsize\bfseries}{\thesubsubsection.}{4pt}{\normalsize}
\titlespacing*{\subsubsection}{0pt}{0pt}{0pt}

\def\tablename{Tabla}

%% Variables para portada y cabeceras
%% Cambiar los valores para cada documento!!!
\def\title{Explotación, integración y visualización de múltiples fuentes de datos mediante un Data Lake}
\def\shortTitle{ETL de datos heterogéneos mediante un Data Lake}
\def\subject{Lenguajes y Sistemas Informáticos}
\def\author{Mier Montoto, Juan Francisco}
\def\date{julio 2024}
\def\org{Escuela Politécnica de Ingeniería de Gijón}
\def\area{Grado en Ingeniería Informática en Tecnologías de la Información}
\def\tutorOne{D. Augusto Alonso, Cristian}
\def\tutorTwo{D. Morán Barbón, Jesús}
\def\tutorThree{D. Vázquez Faes, Eduardo}
\def\tfgId{???}
\def\dni{71777658V}

\def\ORG{\expandafter\MakeUppercase\expandafter{\org}}
\def\AREA{\expandafter\MakeUppercase\expandafter{\area}}
\def\SUBJECT{\expandafter\MakeUppercase\expandafter{\subject}}


\captionsetup{justification=centering}
\setlength{\headheight}{65pt}

\fancyhf{}
\fancyhead[L]{\includegraphics[height=8mm]{style/square.png}
  \hspace{3em} \Longstack[l] {
    \textbf{\SUBJECT} \newline
    \textbf{\shortTitle}}
  \newline \leftmark{}
}
\fancyhead[R]{\bfseries{Hoja \hyperlink{toc}{\thepage}~de~\pageref{LastPage}}}
\fancyfoot[C]{\author}
\renewcommand{\headrulewidth}{0pt} % default is 0pt
\renewcommand{\footrulewidth}{0.4pt} % default is 0

\fancypagestyle{plain}{%
  \fancyhf{}
  \fancyhead[L]{\includegraphics[height=8mm]{style/square.png}
    \hspace{3em} \Longstack[l]{
      \textbf{\SUBJECT} \newline
      \textbf{\shortTitle}}}
  \fancyhead[R]{\bfseries{Hoja \hyperlink{toc}{\thepage}~de~\pageref{LastPage}}}
  \fancyfoot[C]{\author}
  \renewcommand{\headrulewidth}{0pt} % default is 0pt
  \renewcommand{\footrulewidth}{0.4pt} % default is 0pt
}

\newcommand*{\fullref}[1]{\textit{\hyperref[{#1}]{\ref*{#1} \nameref*{#1}}}}
\newcommand*{\halfref}[2]{\texttt{\hyperref[{#1}]{#2}}}
\newcommand{\blankpage}{
  \newpage
  \thispagestyle{empty}
  \mbox{}
  \newpage
}

\pagestyle{fancy}

\restylefloat{table}
\definecolor{backcolour}{rgb}{0.95,0.95,0.92}
\definecolor{darkgreen}{rgb}{0.0, 0.5, 0.0}

\lstdefinestyle{default}{
  basicstyle=\ttfamily\footnotesize,
  breakatwhitespace=false,
  breaklines=true,
  captionpos=b,
  keepspaces=true,
  numbers=left,
  numbersep=5pt,
  numberstyle=\tiny\color{black},
  showspaces=false,
  showstringspaces=false,
  showtabs=false,
  tabsize=2,
  backgroundcolor=\color{backcolour},
  postbreak=\mbox{\textcolor{red}{$\hookrightarrow$}\space},
}

\lstset{style=default}

\lstdefinestyle{yaml}{
  basicstyle=\ttfamily\footnotesize\color{darkgreen}\bfseries,
  rulecolor=\color{black},
  string=[s]{'}{'},
  stringstyle=\color{darkgreen},
  comment=[l]{:},
  commentstyle=\color{blue},
  morecomment=[l]{-},
  numbers=left,
  numbersep=5pt,
  numberstyle=\tiny\color{black},
  backgroundcolor=\color{backcolour},
  breaklines=true,
  captionpos=b,
  keepspaces=true,
  showspaces=false,
  showstringspaces=false,
  showtabs=false,
  tabsize=2,
  breakatwhitespace=true,
  postbreak=\mbox{\textcolor{red}{$\hookrightarrow$}\space},
}

\pgfplotsset{compat=1.18}
\usepgfplotslibrary{dateplot}



\begin{document}

\rmfamily % Fuente tipo Romana
\begin{titlepage}
    \centering
    \bfseries {
        \null{}
        \vspace{0cm}
        \begin{table}[h]
            \centering
            \begin{tabular}{m{10cm} m{1cm} m{3cm}}
                \vspace{0.2cm}
                \includegraphics[width=70mm]{style/full.png} &
                \vspace{1.5mm} \includegraphics[width=25mm]{style/square.png} \\
            \end{tabular}
        \end{table}

        \vspace{3\baselineskip}

        \Large{\ORG{} \\ \vspace{3\baselineskip}}
        \large {
            \AREA{} \\ \vspace{3\baselineskip}
            \subject{} \\ \vspace{2\baselineskip}

            TRABAJO FIN DE GRADO/MÁSTER Nº \tfgId{} \vspace{\baselineskip} \\
            \title{} \\ \vspace{1\baselineskip}

            \author{} \\ \vspace{1\baselineskip}
            TUTORES:\\
            \tutorOne{} \\
            \tutorTwo{} \\
            \tutorThree{} \\ \vspace{\baselineskip}

            \vspace{2\baselineskip}
            FECHA:\@ \date{}
        }
    }
\end{titlepage}
 % Portada de la memoria

% Índice de contenido
\addcontentsline{toc}{chapter}{Índice de contenido} % Añade la referencia al índice de contenido
\hypertarget{toc}{}
\tableofcontents
\newpage{}

% Índice de figuras
\newpage{}
\addcontentsline{toc}{chapter}{Índice de figuras}  % Añade la referencia al índice de contenido
\hypertarget{lof}{}
\listoffigures

% Índice de tablas
\newpage{}
\addcontentsline{toc}{chapter}{Índice de tablas}  % Añade la referencia al índice de contenido
\hypertarget{lot}{}
\listoftables
\newpage{}

\justify{} % Texto justificado
\setlength{\parskip}{\baselineskip} % Separación entre párrafos de 1 linea
\onehalfspacing{}

%% El contenido de la memoria, dividido en capítulos:
\chapter{Introducción}\label{chap:intro}
El proyecto que se presenta en este documento tiene como objetivo la
automatización de despliegue de la infraestructura y procesos que permitan hacer
un análisis masivo de datos. Para conseguir este objetivo, se hace un
análisis del contexto y se realiza un diseño para, posteriormente, implementar
una solución que permita la integración, almacenamiento y análisis de grandes
volúmenes de datos, provenientes de múltiples fuentes y en diferentes formatos.


\section{Antecedentes}\label{sec:antecedentes}
Hoy en día, el crecimiento de la cantidad de dispositivos conectados a Internet
(teléfonos móviles, dispositivos \textit{IoT}\ldots) ha provocado un aumento
exponencial de la cantidad de datos que se manejan \footnote{
	\url{https://www.statista.com/statistics/871513/worldwide-data-created/}
}, un hecho que se ve reflejado en el ámbito empresarial. Dicha cantidad de
datos genera una necesidad de análisis y tratamiento que las tecnologías
tradicionales de datos (bases de datos) no pueden suplir. La diversidad de
fuentes y formatos de estos datos introduce una complejidad significativa en su
manejo, conocida como \textit{heterogeneidad} \footnote{
	\url{https://www.sciencedirect.com/topics/computer-science/data-heterogeneity}
}, siendo las bases de datos, archivos de registros y APIs las fuentes más
habituales.

El término \textit{big data} describe este fenómeno de acumulación masiva de
datos, cuya magnitud y complejidad sobrepasan las capacidades de los métodos de
procesamiento convencionales. El \textit{big data} se caracteriza por tres
características principales: volumen, variedad y velocidad - su adecuada gestión
y análisis pueden otorgar ventajas competitivas significativas a las empresas,
tales como el descubrimiento de patrones ocultos, identificación de nuevas
oportunidades de mercado y optimización de procesos de toma de decisiones.

Uno de los procesos que permite la extracción de esta información es la pirámide
DIKW, \cite{enwiki:1211227190} es un modelo que describe la relación entre los
datos, la información, el conocimiento y la sabiduría. Según este modelo, los
datos son la materia prima de la información, que a su vez es la materia prima
del conocimiento, que a su vez es la materia prima de la sabiduría. Una
organización sin los procesos adecuados para la gestión y análisis de estos
datos, se enfrenta a importantes desafíos, como la dificultad para identificar
patrones y tendencias, la toma de decisiones incorrectas y la pérdida de
oportunidades de negocio. Por otro lado, una organización que logre extraer
información valiosa de sus datos, podrá mejorar su eficiencia, aumentar su
competitividad y adaptarse mejor a un entorno empresarial en constante cambio.

La evolución tecnológica ha propiciado el desarrollo de innovadoras herramientas
y metodologías diseñadas para enfrentar estos desafíos. Entre ellas, los
\textit{data lakes} (o \emph{lagos de información}) se destacan por su capacidad
para consolidar vastos volúmenes de datos heterogéneos, facilitando su posterior
análisis y aprovechamiento de manera más efectiva.

Sin embargo, a pesar de que existen herramientas de almacenamiento, el proceso
de integración, visualización y análisis de estos datos es una tarea desafiante,
ya que requiere de una gran cantidad de recursos y de un tiempo de desarrollo
considerable del que, normalmente, no se dispone en el ámbito empresarial.

Con la ingesta masiva de datos, se presentan nuevos problemas a la hora de
analizar y obtener información de ellos:

\begin{itemize}
	\item \textbf{Grandes cantidades de información:}
		la masificación de información impide el análisis manual de los
		mismos, requiriendo resúmenes estadísticos o representaciones gráficas
		como \textit{dashboards} para su correcta interpretación.
		La visualización de datos es una técnica que permite representar la
		información de manera visual, para facilitar su análisis y comprensión,
		una parte vital del proceso de análisis de datos, ya que permite
		identificar patrones, tendencias y anomalías en los mismos de forma más
		rápida y sencilla.
	\item \textbf{Heterogeneidad de los datos:}
		la heterogeneidad de los datos, tanto en formato como en origen,
		dificulta su consolidación y análisis, ya que requiere de un proceso de
		integración y transformación previo para homogeneizarlos y poder
		analizarlos de forma conjunta.
	\item \textbf{Decisiones de negocio erróneas debidas a un mal tratamiento:}
		sin la necesaria automatización y correcta aplicación de los procesos
		ETL~(ver \ref{sec:etl}), los resultados del análisis pueden ser
		incorrectos, lo que deriva en errores y decisiones de negocio
		equivocadas que impactan negativamente en la empresa.
\end{itemize}

\newpage{}
\section{Motivación}\label{sec:motivacion}
Actualmente, las empresas (especialmente aquellas en el sector IT), se enfrentan
a la necesidad de unificar, gestionar y analizar grandes volúmenes de datos,
provenientes de múltiples fuentes y en diferentes formatos. La correcta gestión
y análisis de estos datos es fundamental para la toma de decisiones y para la
mejora de los procesos internos de la empresa.

En la actualidad, Okticket (en adelante la empresa) dispone de una gran cantidad
de datos que se encuentran en diferentes formatos y en diferentes ubicaciones,
lo que dificulta su análisis y explotación. Por otra parte, se depende de la
consulta manual o de servicios de terceros para poder analizar estos datos, lo
que supone un coste adicional.

El proyecto surge de la necesidad de la empresa de extraer información y
conocimiento de las múltiples y heterogéneas fuentes de datos de las que se
disponen, tanto internas (e.g.~bases de datos, archivos de registros, APIs,
entre otros), como externas (e.g. APIs o datos de webs de terceros, datos de
fuentes públicas\ldots).

Además del uso interno, la empresa también quiere ofrecer a sus clientes la
posibilidad de consultar estos datos de forma visual y sencilla, para que puedan
analizarlos y explotarlos de forma autónoma, lo que supondría un valor añadido
para los mismos.

\newpage{}
\section{La empresa}\label{sec:empresa}
Okticket es una startup nacida en Gijón en 2017 cuyo producto principal es un
servicio software que escanea automáticamente tickets y facilita su gestión usando conceptos
contables como notas de gastos, anticipos y más. Esto
permite reducir los costes y el tiempo que invierten las empresas en
contabilizar y manejar los gastos de viaje profesionales.

La empresa tienen su suede principal en el Parque Tecnológico de Gijón, aunque
cuenta con un número de sedes creciente en varios países, como Francia, Portugal
o, más recientemente, México. En esta oficina principal se encuentran los
departamentos de ventas y marketing, así como el equipo de desarrollo y consultoría.

Okticket es una de las empresas que más crecen tanto del sector como del propio
Parque Tecnológico. Debido a este rápido crecimiento, el equipo está en
constante desarrollo y cambio, tanto aquí en España como en el resto de sedes.
Este crecimiento se refleja en la recepción de un gran número de galardones y
reconocimientos.
\footnote{\href
	{https://www.linkedin.com/posts/okticket_okticket-en-el-especial-startups-de-forbes-activity-7140622980618903552-UGWK}
	{Okticket en el especial startups 2023 de Forbes (LinkedIn)}
}
\footnote{\href
	{https://www.elcomercio.es/economia/arcelor-okticket-premios-20230222002438-ntvo.html}
	{Arcelor y Okticket, premios nacional de Ingeniería Informática (EL COMERCIO)}
}
\footnote{\href{
	https://www.okticket.es/blog/empresa-pyme-innovadora}
	{Okticket recibe el sello Pyme Innovadora (okticket.es)}
}
\footnote{\href
	{https://www.okticket.es/blog/okticket-empresa-emergente-certificada}
	{Okticket, empresa emergente certificada (okticket.es)}
}

La parte principal del negocio es el núcleo del software como servicio (Software
as a Service en inglés, en adelante \textit{SaaS}), es decir, la aplicación
completa tanto para administradores como para empleados. Este SaaS se oferta a
empresas de cualquier tamaño, cuyo precio final varía en función del número de
usuarios, las características e integraciones que requiera la empresa cliente y
el soporte que se ofrezca.

Recientemente se han añadido nuevas propuestas a la cartera de servicios
ofertada por Okticket, como la OKTCard {-} una tarjeta inteligente que gestiona
automáticamente los gastos, así como la inclusión de nuevos ``módulos'' de
gestión de gastos y viajes.

Debido al crecimiento acelerado de Okticket, la empresa maneja una gran cantidad
de datos de diversos tipos y almacenados en diferentes silos (programas de
gestión contable, ventas, consultoría, así como los datos que genera el SaaS),
que deben ser unificados para poder ser analizados y explotados de forma eficiente.
Por otra parte, actualmente se depende de la consulta manual o de servicios de
terceros para poder analizar estos datos, lo que es costoso, tedioso y muy
poco eficiente.

\section{Objetivo y alcance}\label{sec:objetivos}
El objetivo del proyecto es la creación de un proceso que permita el despliegue
automático de una infraestructura para la integración, almacenamiento y
análisis de grandes volúmenes de datos, provenientes de múltiples fuentes y en
diferentes formatos. La infraestructura de datos debe ser escalable, flexible y
robusta, para poder adaptarse a las necesidades cambiantes de la empresa.

La integración de datos debe ser automática y programable, para poder
automatizar el proceso de ingestión de datos y reducir el tiempo y los costes
asociados.

El resultado final del proyecto será la plataforma en sí, es decir, la
infraestructura automatizada que integre y almacene los datos. El entregable
final será la colección de ficheros y \textit{scripts} necesarios para el
despliegue de la infraestructura y el tratamiento de la información.

La plataforma que se desarrolle debe de ser capaz, además de manejar los datos
comentados anteriormente, de ofrecer una interfaz visual de consulta y análisis
de los mismos, para que los usuarios puedan explotar la información de forma
sencilla y rápida.

\chapter{Fundamento teórico}\label{chap:teo}
En este capítulo se presentan los conceptos y términos fundamentales que se
utilizan en el proyecto para proporcionar una base teórica sobre la que se
desarrolle. Se discuten los conceptos fundamentales.


\section{\textit{Big data}}\label{sec:bigdata}
El término \textit{big data} se refiere a la gestión y análisis de grandes
volúmenes de datos que no pueden ser tratados de manera convencional. La
evolución natural del progreso tecnológico, la digitalización de la sociedad y
la aparición de nuevas tecnologías han propiciado la generación de grandes
cantidades de datos en todo el mundo, lo que genera la necesidad de nuevas
formas de gestionar y tratar estos datos.

El término \textit{big data} no solo se refiere a la cantidad de datos que se
generan, sino a también otras características, a las que se refieren como ``las
uves del big data''. La cantidad de \textit{uves} depende del autor y de la
fuente~\cite{ishwarappa2015vs,sagiroglu2013bigdata}, variando desde 3 hasta 7,
pero las más comunes son las siguientes:

\begin{itemize}
	\item \textbf{Volumen:} la cantidad de datos que se generan y almacenan en
		un determinado periodo de tiempo. El volumen de datos que se maneja en
		el \textit{big data} es mucho mayor que el que se maneja en los sistemas
		tradicionales de gestión de datos, que además se encuentra en aumento
		constante.
	\item \textbf{Variedad:} se refiere a la diversidad de fuentes y formatos de
		los datos que se manejan. Los datos pueden provenir de diversas fuentes,
		como bases de datos, sensores o registros, pueden estar en
		diferentes formatos o tener diferentes estructura. Se pueden clasificar
		de la siguiente manera: \begin{itemize}
			\item \textbf{Estructurados:} datos que se encuentran en un formato
				estructurado, como una base de datos relacional.
			\item \textbf{Semi-estructurados:} datos que no se encuentran en un
				formato estructurado, pero que tienen una estructura interna
				que permite su análisis, como un archivo XML o JSON.
			\item \textbf{No estructurados:} datos que no tienen una estructura
				definida, como un archivo de texto o una imagen.
		\end{itemize}
	\item \textbf{Velocidad:} se refiere a la frecuencia con la que se generan y
		se procesan los datos. En este ámbito, los datos se tratan a una
		velocidad mucho mayor que en los sistemas tradicionales de gestión de
		datos. Dicha velocidad puede ser crítica en ámbitos como la bolsa,
		donde la velocidad de procesamiento de los datos puede ser la
		diferencia entre obtener beneficios o pérdidas. Según la frecuencia de
		procesamiento de los datos, los sistemas se pueden clasificar en:
		\begin{itemize}
			\item \textbf{Batch (en lotes):} los datos se procesan en lotes, de
				manera periódica, como cada hora o cada día. Este tipo de datos,
				frecuente en aplicaciones que se tratan en Okticket como los
				notas de viajes o las nóminas, no requieren un procesamiento
				inmediato.
			\item \textbf{Streaming (en tiempo real):} los datos se procesan en
				tiempo real, a medida que se generan. Este tipo de datos, que se
				puede encontrar en aplicaciones como los sensores o los logs,
				requieren un procesamiento inmediato si se quiere obtener
				información relevante sobre los mismos.
			\item \textbf{Near real-time (casi en tiempo real):} los datos se
				procesan con un pequeño retraso, de manera que se obtiene
				información relevante sobre los mismos en un tiempo muy corto.
		\end{itemize}
	\item \textbf{Veracidad:} se refiere a la calidad de los datos que se
		manejan. La veracidad de los datos es un factor crítico en el ámbito del
		\textit{big data}, pues la calidad de la salida depende directamente de
		la calidad de la información de entrada. La veracidad de los datos se
		puede ver afectada por diferentes factores, como la calidad de los
		datos, la precisión de los mismos, la integridad de los datos, etc.
\end{itemize}


\newpage{}
\section{Paradigmas de almacenamiento de datos}\label{sec:paradigmas}
En el ámbito del \textit{big data}, existen diferentes paradigmas de
almacenamiento de datos que se utilizan para almacenar y analizar grandes
cantidades de información. Los tres paradigmas a considerar para este proyecto
son los \textit{data warehouses}, los \textit{data lakes} y los
\textit{data lakehouses}.


\subsection{Data warehouse}\label{sec:warehouse}
Un \textit{data warehouse}\footnote{
	\url{https://aws.amazon.com/es/data-warehouse/}
}, también conocido en español como almacén de datos, es una base de datos que
se utiliza para almacenar y analizar grandes cantidades de datos de manera
eficiente. Los almacenes de datos proporcionan acceso rápido y compatible con
plataformas de consultas (como SQL) a grandes cantidades de datos, lo que
permite a los analistas y a los científicos de datos realizar análisis complejos
sobre los datos almacenados.

Todos los datos almacenados en un \textit{data warehouse} se encuentran en un
formato común, para lo que se aplican procesos ETL (extracción, transformación y
carga) que transforman los datos de diferentes fuentes en un formato común. Esto
significa que la información se encuentra en un formato o esquema optimizado y
específico, lo que facilita su manipulación y análisis pero limita la
flexibilidad al acceso de los datos y genera costes adicionales en el caso de
tener que modificar o transferir los mismos para su uso.


\subsection{Data lake}\label{sec:lake}
Los \textit{data lakes}\footnote{
	\url{https://aws.amazon.com/es/what-is/data-lake/}
} son almacenes de datos que guardan grandes cantidades de datos de manera no
estructurada~\cite{mier2023dashboards}. En el ámbito de una empresa, un
\textit{data lake} contiene datos de diferentes fuentes de valor no considerado
hasta su análisis, de manera que su explotación posterior y su análisis no
depende de una estructuración y transformación compleja, reduciendo los costes
de los procesos ETL derivados, una flujo de tareas que se aplican sobre la
información para ingestarla. Esto no quiere decir que no se apliquen estos
procesos a los datos, sino que se aplican de manera más flexible y básica que en
otras estructuras de almacenamiento de datos con esquemas predefinidos, como los
\textit{data warehouses}.~\cite{pwint2018data}

A diferencia de los \textit{data warehouses}, los \textit{data lakes} no tienen
un esquema definido, lo que permite almacenar datos \textit{heterogéneos}. Esto
permite almacenar grandes cantidades de información sin tener que definir un
esquema de antemano, lo que puede ser útil en aquellos casos en los que no se
conoce la estructura de los datos que se van a almacenar.

Estas características de los \textit{data lakes} hacen que sean más atractivos
en el sector empresarial, puesto que implica la gestión de un solo \textit{stack}
tecnológico que contiene toda la información, en contraste con las estructuras
planteadas normalmente en el campo de la investigación académica.

Para consultar esta gran cantidad de datos almacenados, se suelen utilizar
técnicas de visualización de datos, como los \textit{dashboards}, herramientas
de visualización que permiten observar los datos de manera sencilla y eficiente.


\subsection{Data lakehouse}\label{sec:lakehouse}
Los \textit{data lakehouses} son una combinación funcional de los dos paradigmas
vistos anteriormente, los \textit{data lakes} y los \textit{data warehouses}.
Los \textit{data lakehouses} permiten almacenar datos tanto de manera
estructurada como no estructurada, lo que facilita aprovechar la información al
contar con una única estructura de bajo coste que ofrece a los usuarios que lo
necesiten explorar y analizar los datos según sus necesidades.


\newpage{}
\section{Procesos ETL}\label{sec:etl}
Si anteriormente se presentaban los distintos paradigmas de almacenamiento de
datos, para su creación y mantenimiento se requieren aplicar unos ciertos
procesos que permitan la correcta ingesta y almacenamiento de los datos. Estos
procesos se conocen como \textit{procesos ETL}.

Formalmente se definen los procesos ETL~\cite{mier2023dashboards} como procesos
que combinan datos de múltiples fuentes en un único destino, transformando los
datos en un formato común. Estos procesos se utilizan para extraer datos de
diferentes fuentes, transformarlos en un formato común y cargarlos en un destino
común, como puede ser un \textit{data lake}.

Los procesos ETL, fundamentales en el ámbito de la gestión de datos, presentan
atributos distintivos que facilitan la integración eficaz de información
procedente de diversas fuentes:

\begin{itemize}
	\item \textbf{Adaptabilidad:} los procesos ETL deben de adaptarse a la
		estructura de los datos de la fuente de origen, ya que dichas fuentes
		pueden tener diferentes estructuras y tener tipos de datos diferentes
		(la característica de \textit{heterogeneidad} de los datos que ya se ha
		mencionado).
	\item \textbf{Escalabilidad:} otra de las características clave de los
		procesos ETL es que sean escalables, ya que los datos que se muestran en
		los dashboards suelen ser datos que se generan de manera continua, y por
		lo tanto los procesos ETL deben ser capaces de procesar grandes
		cantidades de datos de manera eficiente. En ocasiones, los procesos ETL
		se pueden realizar en \textit{streaming}, lo que significa que los datos
		se procesan en tiempo real a medida que se generan.
	\item \textbf{Eficiencia:} los procesos ETL deben ser eficientes, puesto que
		el tiempo de procesamiento de los datos es un factor vital en el ámbito
		del \textit{big data}. Los procesos ETL deben ser capaces de procesar
		grandes cantidades de datos en un tiempo razonable para que los datos
		estén disponibles en el menor tiempo posible.
	\item \textbf{Fiabilidad:} la fiabilidad es un componente crítico de todo el
		flujo de datos, ya que estos se utilizan para la toma de decisiones
		importantes de cualquier empresa. Los procesos ETL deben ser capaces de
		procesar los datos de manera fiable y consistente, para que los datos
		que se visualicen y analicen posteriormente sean correctos y fiables.
\end{itemize}


\subsection{Funcionamiento}
Los procesos ETL se dividen en tres fases principales: \textit{(1) Extraer},
\textit{(2) Transformar} y \textit{(3) Cargar}, como se muestra en el siguiente
diagrama:

\begin{minipage}{\linewidth}
	\centering
	\includegraphics[width=0.8\textwidth]{etl.png}
	\captionof{figure}{Fases de un proceso ETL}
\end{minipage}

Como entrada, se tienen datos presuntamente heterogéneos que no se pueden
analizar de manera eficiente. Tras aplicar todos los pasos de las fases
anteriores, se obtiene como salida un conjunto de datos corregidos y listos para
ser analizados en el destino indicado, sea cual sea el paradigma de
almacenamiento de datos elegido.

\paragraph{Extracción (1)}
En este proceso se obtienen los datos de las fuentes de datos, que pueden ser
bases de datos, logs, APIs, etc. En esta fase, se pueden aplicar filtros para
extraer solo los datos que se necesiten, y se pueden extraer datos de múltiples
fuentes \emph{heterogéneas}.

La fase de extracción se puede realizar de dos formas: continua o incremental.
Una extracción incremental se realiza de manera periódica, por ejemplo, cada
hora, cada día o cada semana, y se extraen los datos que se han generado desde
la última extracción. Esto es útil cuando los datos se generan de manera
periódica y se necesita mantener actualizada la información. Por otro lado, en
una extracción continua se extraen los datos en tiempo real según se van
generando. Esto puede ser útil para procesar datos que se generan en tiempo
real, como logs o datos de sensores.


\newpage{}
\paragraph{Transformación (2)}
Durante esta fase, se transforman los datos extraídos en la fase anterior,
normalmente aplicándoles un proceso de limpieza y transformación a un
formato común. En este paso, se pueden aplicar diferentes operaciones a los
datos, como la limpieza, la agregación, la normalización, la conversión de
formatos, etc.

Uno de los tipos de transformaciones de datos más comunes es la limpieza, que
consiste en la revisión y corrección de los datos extraídos, para asegurar que
se almacena información correcta y consistente. Durante esta fase se contemplan
operaciones más complejas, como pueden ser la agregación de datos, la conversión
de formatos, la normalización de datos, el cifrado, etc. La limpieza de datos
puede ser una tarea muy sencilla, como la eliminación de caracteres
delimitadores, o muy compleja, como la corrección de errores en los datos, la
detección de duplicados o la minimización~\cite{mezmir2020qualitative} y/o
compresión~\cite{lelewer1987data} de datos (eliminación de información no
relevante o redundante).

Estos procesos de transformación son vitales cuando el sistema maneja una gran
cantidad de datos heterogéneos de múltiples fuentes de manera simultánea, como
puede ser el caso de un \textit{data lake} o un \textit{data warehouse}.
En el caso del primero, no es necesaria la transformación de los
datos a un formato común, pero si otros procesos clave como la limpieza y la
normalización de los datos, entre otros.

\paragraph{Carga (3)}
En este proceso se vuelcan los datos transformados en el destino final.
Frecuentemente, los datos se almacenan, dependiendo del paradigma de
almacenamiento elegido, en una \textit{data lake}, \textit{data warehouse} o
\textit{data lakehouse} para su posterior análisis.

Las características de la carga de datos varían dependiendo de la arquitectura
de datos que se esté utilizando. Por ejemplo, en ciertos sistemas puede ser
necesario cifrar los datos antes de cargarlos en el destino, o puede ser
necesario realizar una carga incremental para mantener actualizada la información
en el destino. En otros casos, puede ser necesario realizar una carga masiva
para cargar grandes cantidades de datos en el destino de una sola vez. La
periodicidad de la carga es una característica clave del \textit{big data}, como
se ha mencionado anteriormente~(ver \fullref{sec:bigdata}).


\newpage{}
\subsection{Alternativas}
Aunque lo más común es el flujo anteriormente explicado de \textit{extracción},
\textit{transformación} y \textit{carga}, existen algunos flujos alternativos
que son útiles para ciertos procesos diferentes:

\begin{itemize}
	\item \textbf{Virtualización de datos:} capa virtual de abstracción que
		permite acceder a los datos de las fuentes sin necesidad de extraerlos.
		Esto permite ahorrar espacio de almacenamiento y tiempo de
		procesamiento, pero suele ser menos eficiente en términos de rendimiento
		y no es compatible con todas las arquitecturas de datos.

		\begin{minipage}{\linewidth}
			\centering
			\includegraphics[width=0.65\textwidth]{virt.png}
			\captionof{figure}{Ejemplo de flujo con virtualización}
		\end{minipage}
	\item \textbf{Proceso \textit{ELT}\footnote{\url{https://www.ibm.com/topics/elt}}:}
		en lugar de transformar los datos antes de cargarlos en el destino, se
		cargan los datos en bruto y se transforman en el destino. Funciona bien
		para grandes conjuntos de datos sin estructura que requieran una carga
		(o recarga) contínua, aunque, al igual que la virtualización, puede ser
		menos eficiente o incompatible con algunas arquitecturas de datos, como
		los \textit{data warehouses}.

		\begin{minipage}{\linewidth}
			\centering
			\includegraphics[width=0.75\textwidth]{elt.png}
			\captionof{figure}{Diagrama de flujo de un proceso \textit{ELT}}
		\end{minipage}
\end{itemize}


\newpage{}
\section{Cuadros de mandos (\textit{dashboards})}\label{sec:dashboards}
\paragraph{Definición}
Los cuadros de mandos, en adelante \textit{dashboards}, son soluciones en forma
de interfaz gráfica que muestran información relevante de manera visual sobre un
proceso o negocio. Aunque el término se utiliza en muchos ámbitos (indicadores
comerciales, de producción, de marketing, de calidad, de recursos humanos\ldots)
en este proyecto se utilizará en el ámbito de la monitorización de sistemas y
procesos de negocio.

En el ámbito de deste proyecto, los dashboards reflejan en tiempo real el
rendimiento de actividades o procesos de negocio, y se utilizan para tomar
decisiones informadas basándose en los mismos. Por ejemplo, el dashboard de una
empresa digital puede mostrar desde el rendimiento de la arquitectura en tiempo
real hasta el número de ventas conseguidas, y permitir a los directivos tomar
decisiones informadas sobre el futuro de la empresa (e.g. necesidad de aumentar
la capacidad de los servidores, lanzar una campaña de marketing, etc.).

\paragraph{Características}
Los dashboards cuentan con una serie de características que los hacen útiles
para la toma de decisiones:~\cite{mier2023dashboards}

\begin{itemize}
	\item \textbf{Visualización de datos:} es la característica fundamental de
		cualquier dashboard, y aquella que determina su utilidad.
		La visualización de datos es la ciencia de presentar los datos de manera
		que se pueda extraer información útil y realizar decisiones informadas
		sobre ellos. Un buen dashboard cuenta con gráficas, tablas, indicadores,
		etc. que permiten al usuario entender la información que se está
		presentando con un conocimiento técnico mínimo.
	\item \textbf{Interactividad y personalización:} un dashboard debe permitir
		al usuario interactuar con los datos (filtrarlos, ordenarlos,
		profundizar en ellos...) y ajustar la información que se muestra sobre
		cada proceso o negocio que se esté evaluando (granularidad de la
		información). Esta capacidad asegura que el dashboard se adapte tanto a
		las necesidades actuales como a las evoluciones futuras de lo que se
		esté analizando.
	\item \textbf{Accesibilidad y portabilidad:} un dashboard debe ser accesible
		desde una variedad de situaciones y dispositivos, manteniendo su
		funcionalidad y forma. Aunque normalmente los dashboards se analizan en
		pantallas grandes, es importante que también se puedan consultar en
		otras circunstancias, como dispositivos móviles.
\end{itemize}


\newpage{}
\section{Infraestructura como código}
La infraestructura como código (o \textit{IaC} por sus siglas en inglés) es una
práctica que consiste en gestionar la infraestructura de un sistema de manera
automática y programática mediante código, en lugar de configuraciones manuales.

La infraestructura como código permite gestionar la arquitectura global de un
sistema de manera eficiente y escalable, y facilita la creación y el mantenimiento
de entornos de desarrollo y producción, lo que elimina la necesidad de realizar
tareas repeititivas~\cite{beyer2016site} (\textit{TOIL} por sus siglas en
inglés), reduce la posibilidad de errores humanos y favorece la replicabilidad
de los procesos.

En el ámbito de este proyecto, la infraestructura como código se utilizará para
gestionar el despliegue y orquestación de los servicios requeridos para el
paradigma de almacenamiento que se escoja, como los servicios de ingesta o de
visualización de datos.

% La infraestructura como código es la parte más importante del desarrollo de este
% proyecto, ya que una buena configuración y toma de decisiones a la hora de
% desplegar un proyecto de este calibre es vital para el correcto funcionamiento
% posterior a la hora de ingestar y tratar con datos heterogéneos.

\chapter{Descripción general del proyecto}
Esta sección describe el proyecto en términos generales, incluyendo una descripción de los
problemas que se pretenden resolver, las partes interesadas en el proyecto y una valoración de
las alternativas consideradas.


\section{Partes interesadas (stakeholders)}\label{sec:stakeholders}
Las partes interesadas en el proyecto son aquellas personas o entidades que tienen un interés
en el mismo, ya sea porque se ven afectadas por el resultado del proyecto, o porque tienen
algún tipo de interés en el mismo. Las partes interesadas en este proyecto son las siguientes:

\begin{enumerate}
	\item \textbf{Okticket}: la empresa es la principal parte interesada en el proyecto, ya que
		es la que se beneficiará directamente de los resultados del mismo, así como de las
		oportunidades de negocio que se abren con la explotación de los datos. Dentro de la empresa,
		se pueden identificar dos entidades:
		\begin{itemize}
			\item \textbf{Equipo de desarrollo de la empresa}: el equipo de desarrollo es otra parte
				interesada en el proyecto, ya que son los encargados de llevar a cabo la implementación
				del sistema y de garantizar su correcto funcionamiento, además de gestionar el soporte
				de servicio a nivel técnico.
			\item \textbf{Equipo de soporte de la empresa}: el sistema planteado ahorraría tiempo al equipo de
				soporte, ya que les permitiría analizar los datos de forma más eficiente e identificar
				problemas antes de que tener que resolver las peticiones de los clientes afectados a
				nivel básico.
		\end{itemize}
	\item \textbf{Clientes}: los clientes de la empresa también son partes interesadas, puesto
		que se beneficiarán de los nuevos servicios que se ofrecen, como los dashboards de
		negocio que se han descrito anteriormente. Estos clientes no son necesariamente los
		usuarios finales, sino los administradores y gestores de las empresas que utilizan
		Okticket como herramienta de gestión de gastos.
	\item \textbf{Investigador y desarrollador (\emph{\author}):} el desarrollador del
		proyecto tiene la oportunidad de aplicar los conocimientos adquiridos en el desarrollo de un
		proyecto real, y de adquirir nuevos conocimientos en el proceso.
\end{enumerate}

\section{Valoración de alternativas}\label{sec:alternativas}
\subsection{Criterios de evaluación}\label{subsec:criterios}

\subsection{Alternativas consideradas}\label{subsec:alternativas}

\subsection{Resultados}\label{subsec:resultados}

\newpage{}
\section{Descripción del proyecto}\label{sec:descripcion}

\chapter{Planificación del proyecto}\label{chap:planif}
La planificación de un proyecto es fundamental para su correcto funcionamiento y
desarrollo, dentro de los plazos y costes establecidos. Se presenta un primer
apartado de metodología, un segundo apartado con la planificación inicial para
posteriormente inferir en base a esta el presupuesto.


\section{Metodología}\label{sec:metodología}
En este capítulo se aborda la metodología adoptada para el desarrollo del
proyecto, fundamentada en principios ágiles y enfocada en la entrega continua de
valor. La elección de \textit{Scrum}, una metodología que permite elaborar
productos software de manera incremental, revisando el producto continuamente y
adaptándolo a las necesidades del cliente, subraya el compromiso con la
adaptabilidad y la mejora continua del producto.

La estructura de este capítulo se organiza en torno a la descripción detallada
de la metodología \textit{Scrum}, la visualización de la planificación y las
estrategias de comunicación adoptadas. A través de esta metodología, se busca
optimizar los recursos disponibles, ajustarse a los plazos establecidos y
garantizar la calidad del producto final.

La implementación de \textit{Scrum} se complementa con herramientas de
visualización y gestión de proyectos, como los tableros \textit{Kanban}, que
facilitan la organización y seguimiento de las tareas. Además, se pone especial
énfasis en la comunicación efectiva dentro del equipo de desarrollo y con los
stakeholders, asegurando así una alineación constante con los objetivos del
proyecto.

Existen otras variantes de los tableros \textit{Kanban} que se pueden
utilizar para visualizar el progreso de las tareas, pero en este proyecto se ha
elegido esta alternativa para facilitar la visualización de las tareas y su
estado~(ver \fullref{subsec:visual_planif}). La visualización de la
planificación es esencial para el seguimiento y control del proyecto, ya que
permite identificar posibles desviaciones y tomar medidas correctivas de manera
temprana.

Este enfoque metodológico no solo refleja la planificación y ejecución del
proyecto, sino que también establece las bases para una gestión eficaz,
adaptativa y orientada a resultados.


\newpage{}
\subsection{Scrum}\label{subsec:scrum}
Para la planificación del proyecto se ha escogido \textit{Scrum}, una
metodología ``ágil'' que se basa en la realización de iteraciones cortas y en la
adaptación a los cambios. La metodología \textit{Scrum} se estructura en
\textit{sprints} (iteraciones cortas de una duración fija), en las que se llevan
a cabo una serie de tareas que se han planificado previamente.

El primer paso de la metodología \textit{Scrum} es la creación de un
\textit{product backlog}, una lista ordenada de las tareas a realizar durante el
desarrollo del producto, a partir de los requisitos del sistema, que a su vez
son una versión refinada de los requisitos iniciales del proyecto. A partir de
este \textit{product backlog} se planifican las tareas que se llevarán a cabo en
cada \textit{sprint}, de manera que sea posible cumplir con los objetivos del
proyecto en el tiempo establecido.

A diferencia de metodologías tradicionales o \emph{en cascada}, \textit{Scrum}
permite la adaptación a los cambios y la mejora continua del producto, ya que se
revisa y se adapta en cada \textit{sprint} según las necesidades del cliente y
del equipo de desarrollo. Por otro lado, \textit{Scrum} se diferencia de otras
metodologías ágiles como \textit{XP} en que no se centra tanto en las prácticas
de desarrollo, sino en la gestión del proyecto y en la entrega de valor al
cliente.

\begin{minipage}{\linewidth}
	\includegraphics[width=0.85\textwidth]{planif/scrum.png}
	\captionof{figure}{Diagrama de la metodología \textit{Scrum}}
\end{minipage}

\paragraph{Roles}
En \textit{Scrum} se distinguen tres roles principales:

\begin{itemize}
	\item \textbf{Product Owner:} es la persona responsable de definir los
		requisitos del producto y de priorizar las tareas del
		\textit{product backlog}. Es el enlace entre el equipo de desarrollo y
		el cliente, y es el responsable de garantizar que el producto cumple con
		las expectativas del cliente. En el caso de este proyecto, el
		\textit{Product Owner} es el director tecnológico de la empresa.
	\item \textbf{Scrum Master:} es la persona responsable de garantizar que el
		equipo de desarrollo sigue la metodología \textit{Scrum} y de eliminar
		los obstáculos que puedan surgir durante el desarrollo del proyecto. El
		\textit{Scrum Master} es el encargado de organizar las reuniones diarias
		y de asegurar que el equipo de desarrollo cumple con los plazos y los
		objetivos del proyecto. En este proyecto, el \textit{Scrum Master} son
		los tutores académicos del proyecto.
	\item \textbf{Equipo de desarrollo:} es el equipo encargado de llevar a cabo
		las tareas del \textit{product backlog} y de entregar el producto final.
		El equipo de desarrollo es autoorganizado y multidisciplinario, y se
		organiza en torno a las tareas que se van a realizar en cada
		\textit{sprint}. Para este proyecto, el ``equipo'' de desarrollo está
		constituido únicamente por el alumno, que se encarga de todas las tareas
		de desarrollo y documentación.
	\item \textbf{Stakeholders:} son las partes interesadas en el proyecto, como
		los clientes, los usuarios finales y los patrocinadores, que desconocen
		el proceso de desarrollo pero tienen un interés en el producto final y
		en su correcto funcionamiento.
\end{itemize}

\paragraph{Estimación}
En la metodología \textit{Scrum} se pueden utilizar diferentes técnicas de
estimación de tareas, como la estimación en puntos de historia, la estimación en
horas o la estimación en tallas de camiseta. En este proyecto se ha optado por
la estimación en tallas de camiseta, que consiste en asignar a cada tarea una
talla que representa su complejidad y su duración. Las tallas de camiseta se
suelen representar con letras (XS, S, M, L, XL), que se pueden traducir a puntos
de historia siguiendo la secuencia de Fibonacci, es decir, $XS = 1$, $S = 2$,
$M = 3$, $L = 5$, $XL = 8$.

La estimación en Scrum es esencial para la planificación de los \textit{sprints}
y para la asignación de tareas al equipo de desarrollo. La estimación en tallas
se considera óptima para este proyecto, ya que permite una estimación rápida y
sencilla de las tareas, al no necesitar una coordinación entre un equipo
completo de desarrollo.

Además de la estimación del tamaño de las tareas, también se realiza una
estimación sobre la \textit{prioridad} de las mismas, que se representa
siguiendo el equivalente de \textit{GitHub} al sistema de colores de semáforo,
donde el rojo (P0) es la máxima prioridad y el verde (P2) la mínima.

\newpage{}
\subsection{Visualización}\label{subsec:visual_planif}
Para la visualización de la planificación se ha utilizado la herramienta de
gestión de proyecto de \textit{GitHub}, que permite múltiples visualizaciones de
tareas e \textit{issues} en tableros separados.

\begin{itemize}
	\item Se utiliza un tablero de \textit{requisitos} al estilo \textit{Kanban}
		para visualizar los requisitos del proyecto y su estado, siguiendo con
		la metodología \textit{Scrum}. Un tablero \textit{Kanban} es una
		herramienta visual que permite gestionar el flujo de trabajo de un
		proyecto por ``sprints'', dividiendo las tareas en columnas y
		moviéndolas de una columna a otra según su estado.

		\begin{figure}[H]
			\centering
			\includegraphics[width=\textwidth]{planif/kanban3.png}
			\caption{Tablero \textit{Kanban} del proyecto}
			\label{fig:kanban}
		\end{figure}
	\item Adicionalmente, se utiliza un \textit{roadmap} de apartados de la
		memoria, separado del tablero de desarrollo normal, donde se visualiza
		su estado y sus fechas límite. Este \textit{roadmap} no está relacionado
		con la metodología \textit{Scrum}, sino que se ha creado para facilitar
		la visualización del progreso de cada sección y de la memoria en general.

		\begin{minipage}{\linewidth}
			\centering
			\includegraphics[width=0.9\textwidth]{planif/roadmap.png}
			\captionof{figure}{Roadmap de apartados de la memoria}
		\end{minipage}
\end{itemize}


\subsection{Comunicación}\label{subsec:comunicación}
La comunicación con los tutores y con el equipo de desarrollo se considera
fundamental para el correcto desarrollo del proyecto. Puesto que el trabajo se
desarrolla de manera presencial en la oficina de la empresa, la comunicación con
el equipo de desarrollo se realiza de manera frecuente y directa, mientras que
la comunicación con los tutores se realiza de manera remota pero igual de
frecuente, manteniendo el contacto mediante correo electrónico y Teams para
pedir revisiones e informar sobre el estado del trabajo en todo momento.


\subsection{Herramientas}\label{subsec:herr_planif}
Con el objetivo de facilitar las tareas de desarrollo y cumplimentar los
requisitos por parte de la empresa, se utilizan las siguientes plataformas y
herramientas de desarrollo para la fabricación del proyecto:

\begin{itemize}
	\item \textbf{GitHub:} plataforma de desarrollo colaborativo para el
		desarrollo del proyecto. Se utiliza para la gestión de tareas,
		seguimiento del desarrollo y la documentación del proyecto.
	\item \textbf{Suite de Atlassian (\emph{Jira, Bitbucket}):} Suite de
		herramientas de gestión de proyectos y desarrollo colaborativo. Se
		utiliza para el desarrollo y documentación del proyecto de parte de la
		empresa.
	\item \textbf{Suite de Microsoft (\emph{Teams, Outlook}):} se utilizan las
		plataformas de comunicación puestas a disposición por la universida.
\end{itemize}


\newpage{}
\section{Planificación inicial}\label{sec:planif_inicial}
Como se ha mencionado anteriormente, se utiliza la metodología \textit{Scrum}
para la planificación y desarrollo del proyecto. En la figura \ref{fig:backlog}
se puede ver el \textit{backlog} de tareas que se planifican en el proyecto.

\begin{figure}[H]
	\centering
	\includegraphics[width=\textwidth]{planif/backlog.png}
	\caption{Planificación inicial del proyecto}
	\label{fig:backlog}
\end{figure}

Las historias de usuario anteriores se clasifican y categorizan según su
prioridad y tamaño, haciendo uso de la estrategia de tallas de camiseta como
mencionado anteriormente. En el tablero \textit{Kanban}
(ver figura \ref{fig:kanban}) se puede ver en todo momento el estado de las HU,
su progreso y sus características. El listado inicial (ordenado según su prioridad)
es el siguiente:

\begin{table}[H]
	\centering
	\begin{tabular}{|p{0.7\linewidth}|c|c|}
		\hline
		\textbf{Nombre} & \textbf{Prioridad} & \textbf{Tamaño} \\
		\hline
		\hline
		Creación de la infraestructura base (técnica) & P0\cellcolor{red!50} & L\cellcolor{orange!50} \\
		\hline
		Como desarrollador de Okticket, quiero que la arquitectura se despliegue y orqueste de manera automática & P0\cellcolor{red!50} & XL\cellcolor{red!50} \\
		\hline
		Como desarrollador de Okticket, quiero que se ingesten de manera automática datos de la base de datos interna de MongoDB & P0\cellcolor{red!50} & M\cellcolor{yellow!50} \\
		\hline
		Como desarrollador de Okticket, quiero que se ingesten de manera automática datos de la base de datos interna de MySQL & P0\cellcolor{red!50} & M\cellcolor{yellow!50} \\
		\hline
		Como desarrollador de Okticket, quiero que los datos se limpien de manera automática & P0\cellcolor{red!50} & M\cellcolor{yellow!50} \\
		\hline
		Como trabajador de Okticket, quiero poder ver y consultar datos internos de la empresa & P1\cellcolor{orange!50} & L\cellcolor{orange!50} \\
		\hline
		Como desarrollador de Okticket, quiero que se ingesten de manera automática logs de balanceador de AWS & P1\cellcolor{orange!50} & S\cellcolor{green!25} \\
		\hline
		Como desarrollador de Okticket, quiero poder ver el estado general de la infraestructura & P1\cellcolor{orange!50} & L\cellcolor{orange!50} \\
		\hline
		Como desarrollador de Okticket, quiero que los datos contengan metadatos que faciliten su filtrado o búsqueda & P2\cellcolor{yellow!50} & S\cellcolor{green!25} \\
		\hline
		Como trabajador de Okticket, quiero poder ver y consultar datos de empresas cliente & P2\cellcolor{yellow!50} & M\cellcolor{yellow!50} \\
		\hline
		Como gestor de una empresa cliente, quiero poder ver información relevante sobre mi empresa que recoja Okticket & P2\cellcolor{yellow!50} & L\cellcolor{orange!50} \\
		\hline
		Como desarrollador de Okticket, quiero poder ingestar datos de APIs externas a la empresa & P2\cellcolor{yellow!50} & L\cellcolor{orange!50} \\
		\hline
		Como desarrollador de Okticket, quiero poder ingestar información de páginas web externas (\textit{scraping}) & P2\cellcolor{yellow!50} & XL\cellcolor{red!50} \\
		\hline
	\end{tabular}
	\caption{Historias de usuario iniciales}
	\label{tab:initial_tasks}
\end{table}

Siguiendo la tabla anterior, se pueden planear los \textit{sprints} y
asignar las tareas a cada uno de ellos.


\newpage{}
\section{Presupuesto}\label{sec:presupuesto}
Para poder llevar a cabo este proyecto, se realiza una estimación del coste
total neceario para su desarrollo, que se divide en dos partes: el coste del
material, que incluye el coste de los recursos necesarios para el desarrollo del
proyecto, y el coste del personal, que incluye el coste de las horas de trabajo
del desarrollador.

Se estima un horizonte de desarrollo de 3 meses, que es el tiempo estimado y
disponible para el desarrollo del proyecto.


\subsection{Presupuesto de material}\label{subsec:pres_material}
Puesto que el proyecto se desarrolla en la empresa, se dispone de todos los
recursos físicos necesarios para llevar a cabo el proyecto, es decir, que no se
incluirá el coste del ordenador o de la conexión a internet en el presupuesto.

Sin embargo, se incluirá el coste de las herramientas y servicios utilizados
durante el desarrollo del proyecto, como el coste de las licencias de software,
el coste de los servicios en la nube, el coste de las herramientas de
desarrollo, etc.

Es importante destacar de que los precios de los servicios en la nube son
aproximados y pueden variar en función de la región, el tipo de instancia, el
tipo de almacenamiento, etc. Por lo tanto, los precios presentados en este
presupuesto son orientativos y pueden variar en función de las necesidades del
proyecto. En este caso, se analizan los precios a junio de 2024 en la región
de Amazon Web Services (AWS) de \texttt{eu-north-1} (Estocolmo).

\begin{table}[H]
	\centering
	\small
	\begin{tabular}{|l|l|r|r|r|}
	\hline
	\textbf{Categoría} & \textbf{Ítem} & \textbf{Cantidad} & \textbf{Coste unitario} & \textbf{Coste total} \\
	\hline
	\hline
	AWS Compute & Fargate & 7168 vCPU-h/mes & 0,04928€/vCPU-h & 353,24€/mes \\
	 & & 18432 GB-h/mes & 0,00539€/GB-h & 99,35€/mes \\
	\hline
	AWS Storage & EFS & 512 GB/mes & 0,38€/GB-mes & 194,56€/mes \\
	 & S3 & 256 GB/mes & 0,0255€/GB-mes & 6,53€/mes \\
	\hline
	AWS Network & VPC & 4 NAT Gateways & 0,054€/h & 155,52€/mes \\
	 & ELB & 3 ALBs & 0,012€/h & 25,92€/mes \\
	 & & 1 NLB & 0,027€/h & 19,44€/mes \\
	 & Route 53 & 3 registros & 0,50€/registro-mes & 1,50€/mes \\
	\hline
	AWS Security & IAM & - & Sin cargo & 0,00€ \\
	 & KMS & 1 CMK & 1,00€/mes & 1,00€/mes \\
	 & Secret Manager & 12 secretos & 0,40€/secreto-mes & 4,80€/mes \\
	\hline
	Stack KELK & Kafka & - & Licencia gratuita & 0,00€ \\
	 & Elasticsearch & - & Licencia gratuita & 0,00€ \\
	 & Logstash & - & Licencia gratuita & 0,00€ \\
	 & Kibana & - & Licencia gratuita & 0,00€ \\
	\hline
	AWS Container & ECS & - & Sin cargo & 0,00€ \\
	\hline
	AWS Monitor & CloudWatch & 5 métricas & 0,30€/métrica-mes & 1,50€/mes \\
	 & X-Ray & 50,000 trazas/mes & 4,60€/1M trazas & 0,23€/mes \\
	\hline
	Despliegue & Terraform & - & Licencia gratuita & 0,00€ \\
	\hline
	\textbf{Subtotal} & \multicolumn{4}{r|}{863,59€/mes} \\
	\hline
	\hline
	Otros & Support & Plan Basic & Sin cargo & 0,00€ \\
	 & Optimización & - & 5\% del subtotal & 43,18€ \\
	 & Contingencia & - & 10\% del subtotal & 86,36€ \\
	\hline
	\textbf{Total} & \multicolumn{4}{r|}{993,13€/mes} \\
	\hline
	\end{tabular}
	\caption{Propuesta de presupuesto mensual de materiales (región eu-north-1)}
	\label{tab:presupuesto_material}
\end{table}

Asumiendo que el proyecto se desarrolla en la región de AWS de Estocolmo
(\texttt{eu-north-1}), el coste total del material asciende a 993,13€ (novecientos
noventa y tres euros con trece céntimos) al mes, que incluye
el coste de los servicios en la nube, el coste de las licencias de software y el
coste de las herramientas de desarrollo.

Suponiendo un horizonte de desarrollo de 3 meses, el coste total del material
durante el desarrollo del proyecto asciende a 2.979,39€ (dos mil novecientos
setenta y nueve euros con treinta y nueve céntimos).


\newpage{}
\subsection{Presupuesto de personal}\label{subsec:pres_personal}
A continuación, se presenta una propuesta de presupuesto de personal para el
desarrollo del proyecto, que incluye el coste de las horas de trabajo según
cada rol y el coste total del personal.

\begin{table}[H]
	\centering
	\small
	\begin{tabular}{|l|l|r|r|r|}
	\hline
	\textbf{Rol} & \textbf{Descripción} & \textbf{Horas/mes} & \textbf{CU (€/h)} & \textbf{Coste total} \\
	\hline
	\hline
	Arquitecto & Diseño de la arquitectura y supervisión & 40 & 60 & 2.400,00€/mes \\
	\hline
	Desarrollador & Desarrollo y mantenimiento & 160 & 45 & 7.200,00€/mes \\
	\hline
	Administrador & Gestión de sistemas y seguridad & 160 & 50 & 8.000,00€/mes \\
	\hline
	DevOps & Infraestructuras y monitorización & 80 & 55 & 4.400,00€/mes \\
	\hline
	\textbf{Subtotal} & \multicolumn{4}{r|}{22.000,00€/mes} \\
	\hline
	\hline
	Otros & \multicolumn{3}{|l|}{IVA (21\%)} & 4.620,00€/mes \\
	 & \multicolumn{3}{|l|}{Margen (5\%)} & 1.100,00€/mes \\
	\hline
	\textbf{Total} & \multicolumn{4}{r|}{27.720,00€/mes} \\
	\hline
	\end{tabular}
	\caption{Propuesta de presupuesto de personal}
	\label{tab:presupuesto_personal_aws}
\end{table}

El coste del personal es ficticio, pero se ha calculado en base a experiencias
previas de contratación y subcontratación de personal en la empresa, además de
tener en cuenta el coste medio de los roles en Asturias.

El coste total del personal asciende a 27.720,00€ (veintisiete mil setecientos
veinte euros) al mes, que incluye el coste de las horas de trabajo de cada rol,
el IVA y el margen de beneficio industrial.

Asumiendo un horizonte de desarrollo de 3 meses, el coste total del personal
durante el desarrollo del proyecto asciende a 83.160,00€ (ochenta y tres mil
ciento sesenta euros).


\newpage{}
\subsection{Presupuesto total}\label{subsec:pres_total}
Finalmente, se presenta el presupuesto total del proyecto, que incluye el coste
del material y el coste del personal, así como el coste total del proyecto,
durante el horizonte de desarrollo establecido de 3 meses.

\begin{table}[H]
	\centering
	\small
	\begin{tabular}{|l|r|}
	\hline
	\textbf{Concepto} & \textbf{Coste} \\
	\hline
	Presupuesto de materiales & 2.979,39€ \\
	\hline
	Presupuesto de personal & 83.160,00€ \\
	\hline
	\textbf{Subtotal} & \textbf{86.139,39€} \\
	\hline
	\hline
	Beneficio industrial (15\%) & 12.920,91€ \\
	\hline
	\textbf{Total} & \textbf{99.060,30€} \\
	\hline
	\end{tabular}
	\caption{Costes combinados de presupuesto y materiales con beneficio industrial}
	\label{tab:costes_combinados}
\end{table}

El presupuesto total del proyecto asciende a 99.060,30€ (noventa y nueve mil
sesenta euros con treinta céntimos), que incluye el coste del material, el coste
del personal y el margen de beneficio industrial.

\chapter{Diseño del sistema}\label{chap:diseño}
Previo al desarrollo y la implementación del proyecto, es necesario realizar un
diseño detallado del sistema que permita definir la arquitectura, los modelos de
datos y las tecnologías a utilizar. En este capítulo se estudiarán las
alternativas disponibles, se definirá la arquitectura del sistema en la nube y
se establecerán los modelos de datos necesarios para el desarrollo del proyecto.


\section{Estudio de alternativas}\label{sec:estudio}
En este apartado se explorarán las diferentes alternativas disponibles para el
diseño del sistema. Se analizarán las características, ventajas y desventajas de
cada opción, con el objetivo de proporcionar una visión clara y fundamentada que
permita seleccionar la alternativa más adecuada para el proyecto. Las áreas de
estudio incluirán tanto el despliegue de infraestructura como otros aspectos
críticos del diseño del sistema, asegurando una evaluación integral y detallada
de las posibles soluciones.

Los criterios de evaluación en líneas generales son los ya vistos en la sección
\fullref{sec:alternativas}, es decir: \textbf{coste}, \textbf{complejidad},
\textbf{rendimiento y escalabilidad} y \textbf{licencias} (o, más bien, la
ausencia de ellas).

En esta sección se estudiarán las alternativas referentes a:

\begin{itemize}
	\item \textbf{\nameref{subsec:alt_despliegue}}
	\item \textbf{\nameref{subsec:alt_ingesta}}
	\item \textbf{\nameref{subsec:alt_proveedor}}
	\item \textbf{\nameref{subsec:alt_servicios}}
	\item \textbf{\nameref{subsec:alt_maquinas}}
\end{itemize}

El orden de estudio de las alternativas no es casual, sino que sigue un orden
lógico \textit{top-down} en el desarrollo del proyecto, comenzando por la
herramienta de despliegue y los componentes para centrarse posteriormente en
proveedores, servicios, etc.


\newpage{}
\subsection{Despliegue de infraestructura}\label{subsec:alt_despliegue}
A la hora de desplegar la infraestructura de un proyecto, se consideran varias
herramientas populares que permiten automatizar este proceso. Entre todas ellas,
las más establecidas y atractivas son \textit{Terraform},
\textit{AWS CloudFormation} y \textit{Ansible}.

A continuación, se describen brevemente estas herramientas y se comparan sus
características.

\paragraph{Alternativas}
\subparagraph{Terraform} es una herramienta de código abierto desarrollada por
\textit{HashiCorp} que permite definir y desplegar infraestructura de forma
declarativa. \textit{Terraform} permite definir la infraestructura en un archivo
de configuración JSON, que describe los recursos que se desean crear y sus
dependencias. A partir de este archivo, \textit{Terraform} se encarga de
desplegar los recursos en el proveedor de nube especificado, que en el caso de
este proyecto es \textit{AWS}.

\begin{figure}[H]
	\centering
	\includegraphics[width=0.2\textwidth]{logos/terraform.png}
	\caption{Logo de Terraform~\textregistered}
	\label{fig:terraform}
\end{figure}

\subparagraph{AWS CloudFormation} es un servicio de \textit{Amazon Web Services}
similar a \textit{Terraform} que permite definir y desplegar infraestructura en
la nube de forma declarativa. \textit{AWS CloudFormation} permite definir la
infraestructura o bien mediante un archivo de configuración (en formato JSON o
YAML), o bien gráficamente mediante diagramas, un punto muy fuerte a favor de
esta alternativa.

\begin{figure}[H]
	\centering
	\includegraphics[width=0.2\textwidth]{logos/cloudformation.png}
	\caption{Logo de AWS CloudFormation~\textregistered}
	\label{fig:cloudformation}
\end{figure}

\subparagraph{Ansible} es una herramienta multi-propósito de automatización de tareas
entre las que se incluye el despliegue y orquestación de infraestructura. Se
trata de una herramienta desarrollada por \textit{Red Hat} que permite definir
la infraestructura mediante \textit{playbooks} escritos en YAML, que describen
las tareas a realizar y los servidores en los que se deben ejecutar.

\begin{figure}[H]
	\centering
	\includegraphics[width=0.2\textwidth]{logos/ansible.png}
	\caption{Logo de Ansible~\textregistered}
	\label{fig:ansible}
\end{figure}

\paragraph{Comparación}
Comenzando la comparativa por la facilidad de uso de cada herramienta,
\textit{Terraform} es la alternativa planteada más fácil de usar, ya que permite
definir la infraestructura en un archivo con sintaxis sencilla y desplegarla con
un solo comando. Por otro lado, \textit{AWS CloudFormation} es un poco más
complejo de usar, ya que requiere definir la infraestructura en un archivo de
configuración o en un diagrama, y luego desplegarla mediante la consola de
\textit{AWS}. \textit{Ansible} es más complejo de usar, ya que requiere definir
la infraestructura mediante \textit{playbooks} y ejecutarlos en los servidores,
pero es más flexible y potente que las otras dos herramientas.

Mientras que \textit{Terraform} y \textit{Ansible} son herramientas
\textit{multi-cloud}, lo que significa que funcionan con cualquier proveedor de
nube, \textit{AWS CloudFormation} es una herramienta específica de \textit{AWS}
y solo funciona con sus servicios, lo que puede suponer tanto una ventaja como
una desventaja, dependiendo de las necesidades del proyecto. En este caso, la
solución se va a desplegar en la nube de \textit{Amazon}, pero a la vez no se
requiere el uso de servicios específicos de \textit{AWS}, no se considera una
ventaja significativa.

\paragraph{Decisión}
Ninguna de las alternativas consideradas es claramente superior a las demás, ya
que se tratan de herramientas con características y funcionalidades similares y
una gran popularidad en la industria. Sin embargo, se decide utilizar
\textit{Terraform} para el despliegue de la infraestructura de este proyecto,
ya que es la herramienta más ``sencilla'', la que mejor se podría adaptar a las
necesidades del proyecto y la única que ya se ha usado en proyectos anteriores
dentro de la empresa.


\newpage{}
\subsection{Ingesta de datos}\label{subsec:alt_ingesta}
A partir del conjunto de tecnologías seleccionadas en la descripción detallada
del proyecto, se consdieran diversas tecnologías, como \textit{Redpanda},
\textit{AWS Glue} y \textit{Kafka} que permitan ingestar datos de todas las
fuentes que requieren ser procesadas.

\paragraph{Alternativas}
\subparagraph{Kafka} es una plataforma de transmisión de datos distribuida y de
código abierto que se utiliza para construir pipelines de datos en tiempo real y
aplicaciones de streaming. Desarrollada originalmente por LinkedIn y
posteriormente donada a la Apache Software Foundation, \textit{Kafka} se ha
convertido en una de las tecnologías más populares para la gestión de flujos de
datos en tiempo real.

\begin{figure}[H]
	\centering
	\includegraphics[width=0.2\textwidth]{logos/kafka.png}
	\caption{Logo de Kafka~\textregistered}
	\label{fig:kafka}
\end{figure}

Una de las principales ventajas de \textit{Kafka} es su capacidad para manejar
grandes volúmenes de datos con alta eficiencia y baja latencia. \textit{Kafka}
utiliza un modelo de publicación-suscripción, donde los productores publican
mensajes en temas y los consumidores se suscriben a estos temas para recibir
los mensajes. Esta arquitectura permite una alta escalabilidad y flexibilidad
en la gestión de datos.

\textit{Kafka} se compone de varios componentes clave:

\begin{itemize}
    \item \textbf{Tópico}: Un tópico es una categoría a la que se envían los
		mensajes y a la que los consumidores están \textit{suscritos}. Los
		consumidores pueden estar suscritos a uno o varios tópicos, y los
		productores pueden enviar mensajes a uno o varios tópicos. Los tópicos
		son la unidad básica de organización de los mensajes en cualquier
		sistema de mensajería de publicación/suscripción.
    \item \textbf{Productor}: El productor es el componente responsable de crear
		y enviar mensajes al cluster de Kafka. Está separado del resto de los
		componentes y produce mensajes de manera asíncrona y rápida.
    \item \textbf{Consumidor}: El consumidor es el componente responsable de
		leer los mensajes producidos por el productor. Está suscrito a un tópico
		a través del broker y consume los mensajes.
    \item \textbf{Broker}: El broker es el componente responsable de recibir los
		mensajes producidos por el productor y enviarlos a los consumidores. Es
		el intermediario entre los productores y los consumidores.
    \item \textbf{Zookeeper}: Zookeeper es un servicio separado de coordinación
		distribuida que se utiliza para gestionar y coordinar los brokers de
		Kafka. Se encarga de mantener la información de los brokers y de los
		tópicos. Actualmente, este servicio es una dependencia obligatoria de
		Kafka.~\footnote{
			Dejará de ser necesario en la versión 4.
			\url{https://x.com/coltmcnealy/status/1801987159534264641}
		}
\end{itemize}

A pesar de sus numerosas ventajas, \textit{Kafka} también presenta algunos
desafíos. La configuración y gestión de un clúster de \textit{Kafka} puede ser
compleja, especialmente en entornos de producción a gran escala. Además,
\textit{Kafka} depende de \textit{Zookeeper} para la coordinación, lo que añade
una capa adicional de complejidad en la administración del sistema.

En resumen, \textit{Kafka} es una solución robusta y escalable para la
transmisión de datos en tiempo real, ideal para aplicaciones que requieren alta
disponibilidad y procesamiento eficiente de grandes volúmenes de datos. Sin
embargo, su implementación y gestión requieren un conocimiento profundo de su
arquitectura y componentes.

\subparagraph{Redpanda} es una plataforma de transmisión de datos en tiempo real
que se destaca por su alto rendimiento y baja latencia. Diseñada como una
alternativa moderna a \textit{Kafka}, \textit{Redpanda} ofrece una arquitectura
simplificada que elimina la necesidad de dependencias externas como
\textit{Zookeeper}. Esto no solo reduce la complejidad operativa, sino que
también mejora la eficiencia y la escalabilidad del sistema. \textit{Redpanda}
es compatible con la API de \textit{Kafka}, lo que facilita la migración de
aplicaciones existentes sin necesidad de cambios significativos en el código.
Además, su diseño optimizado para hardware moderno permite un procesamiento más
rápido y un uso más eficiente de los recursos, lo que la convierte en una opción
ideal para aplicaciones que requieren una transmisión de datos rápida y
confiable.

Sin embargo, \textit{Redpanda} también presenta algunos puntos en contra. Al ser
una tecnología relativamente nueva, su ecosistema y comunidad de usuarios no son
tan amplios como los de \textit{Kafka}, lo que puede limitar el acceso a
recursos y soporte. Además, aunque la compatibilidad con la API de
\textit{Kafka} es una ventaja, puede haber ciertas características y extensiones
específicas de \textit{Kafka} que no estén completamente soportadas en
\textit{Redpanda}. Finalmente, la adopción de una nueva tecnología siempre
conlleva riesgos asociados con la estabilidad y el soporte a largo plazo,
aspectos que deben ser considerados cuidadosamente antes de su implementación.

\subparagraph{AWS Glue} es un servicio de integración de datos totalmente
administrado que facilita la preparación y carga de datos para análisis.
Diseñado para trabajar con grandes volúmenes de datos, este servicio
automatiza las tareas de descubrimiento, catalogación, limpieza, enriquecimiento
y movimiento de datos entre diferentes almacenes de datos.

Una de las principales ventajas de Glue es su capacidad para generar
automáticamente el código necesario para realizar las transformaciones de datos,
lo que reduce significativamente el tiempo y el esfuerzo requeridos. Además, es
altamente escalable y puede manejar tanto cargas de trabajo por lotes como en
tiempo real, lo que lo convierte en una opción muy versátil.

\textit{AWS Glue} es un servicio administrado, por lo que su uso puede implicar
costes adicionales en comparación con soluciones autogestionadas e introducir
\textit{vendor lock-in}. Además, aunque ofrece una gran flexibilidad y potencia,
su configuración y optimización pueden requerir un conocimiento profundo de los
servicios de la nube de Amazon y las correspondientes prácticas de integración
de datos.


\paragraph{Comparación y decisión}
Desde el primer momento, en la empresa se considera Kafka como la opción más
sólida junto con el \textit{stack ELK} para desarrollar el proyecto, al
tratarse de un estándar en la industria y una solución tanto rápida y escalable
como asequible a nivel económico. Por eso, y pese a que las otras alternativas
son atractivas para el desarrollo de este proyecto, se decide utilizar Kafka
como servicio de ingesta de datos, en consonancia con \textit{Logstash}.



\newpage{}
\subsection{Proveedor de nube}\label{subsec:alt_proveedor}
En cuanto a la elección del proveedor de nube, existen actualmente tres
grandes alternativas en el mercado: \textit{Amazon Web Services} (AWS),
\textit{Microsoft Azure} y \textit{Google Cloud Platform} (GCP). Cada uno de
estos proveedores ofrece una amplia gama de servicios y herramientas que
permiten a las empresas construir, desplegar y escalar aplicaciones en la nube
de forma rápida y eficiente.

\begin{itemize}
	\item \textbf{Amazon Web Services (AWS)} es el proveedor de nube más grande y
		popular del mundo, con una amplia gama de servicios y herramientas que
		permiten a las empresas construir, desplegar y escalar aplicaciones en
		la nube de forma rápida y eficiente. AWS ofrece una infraestructura
		global con centros de datos en todo el mundo, lo que garantiza una alta
		disponibilidad y rendimiento de los servicios. Además, AWS cuenta con
		una amplia comunidad de usuarios y desarrolladores, lo que facilita la
		integración y el soporte de las aplicaciones en la nube.
	\item \textbf{Microsoft Azure} es otro proveedor de nube líder en el mercado
		conocido por su integración con las herramientas y servicios de
		Microsoft. El hecho de formar parte de las soluciones de Microsoft puede
		ser una ventaja para las empresas que ya utilizan sus productos, como es
		el caso de la Universidad de Oviedo.
	\item \textbf{Google Cloud Platform (GCP)} es el proveedor de nube de Google
		y, al igual que AWS y Azure, ofrece una amplia gama de servicios y
		herramientas para construir, desplegar y escalar aplicaciones en la nube.
		GCP es conocido por su enfoque en la innovación y la tecnología de
		vanguardia, lo que puede ser atractivo para empresas que buscan
		soluciones avanzadas y de alto rendimiento. Sin embargo, es la opción
		menos utilizada de las tres y ha tenido escándalos recientes de
		preservación de la información.\footnote{\href
			{https://www.business-standard.com/world-news/google-cloud-accidentally-deletes-125-billion-australian-pension-fund-124051800606_1.html}
			{Google Cloud accidentally deletes \$1.25 billion Australian pension fund}
		}
\end{itemize}

Pese a que las tres alternativas son válidas y ofrecen una amplia gama de
servicios y herramientas, la empresa ya utiliza la nube de Amazon para todas sus
aplicaciones y servicios, por lo que es la opción más lógica y coherente para
este proyecto. Además, AWS cuenta con servicios específicos referentes a
\textit{contenedores} que facilitarán el despliegue de la infraestructura y la
gestión de los servicios en la nube.



\newpage{}
\subsection{Sistemas de ejecución y servicios}\label{subsec:alt_servicios}
Okticket y el equipo de desarrollo utiliza \textit{Amazon Web Services} (AWS)
como proveedor de nube preferido para desplegar sus servicios y aplicaciones.
AWS ofrece una amplia gama de servicios y herramientas que permiten a las
empresas construir, desplegar y escalar aplicaciones en la nube de forma rápida
y eficiente. Dentro de la plataforma de AWS, existen varios servicios que pueden
ser utilizados para implementar las funcionalidades requeridas por el proyecto.

\paragraph{Amazon EC2} es el servicio de computación tradicional de AWS, que
permite lanzar máquinas virtuales con arquitecturas comunes de manera rápida.
Al tratarse de un sistema normal de máquinas virtuales, EC2 no es totalmente
compatible con el modelo de microservicios y contenedores, lo que puede limitar
su escalabilidad y flexibilidad en entornos de producción a gran escala.

\paragraph{Amazon ECS} es un servicio de orquestación de contenedores que
permite ejecutar y escalar contenedores de Docker en la nube de AWS. ECS es
totalmente compatible con Docker y proporciona una interfaz sencilla para
gestionar contenedores en entornos de producción. Sin embargo, ECS puede ser
complicado de configurar y gestionar, especialmente en entornos de gran escala.

\paragraph{Amazon EKS} es un servicio de orquestación de contenedores basado
en Kubernetes que permite ejecutar y escalar contenedores de Docker en la nube
de AWS. EKS es totalmente compatible con Kubernetes y proporciona una interfaz
sencilla para gestionar clústeres de Kubernetes en entornos de producción. EKS
es una opción popular para empresas que ya utilizan Kubernetes y desean
aprovechar las ventajas de la nube de AWS. Sin embargo, esto supondría plantear
todo el desarrollo desde cero en Kubernetes, un sistema que no se ha utilizado
en la empresa hasta la fecha y que requeriría una curva de aprendizaje
significativa.

Para este proyecto, se valoran positivamente las opciones más ``novedosas'' pero
también se ha de tener en cuenta las limitaciones de tiempo y recursos, por lo
que se decide utilizar \textit{Amazon ECS} como servicio de orquestación de
contenedores, ya que es el servicio más sencillo y fácil de configurar de los
tres, y el que mejor se adapta a las necesidades del proyecto.


\newpage{}
\subsection{Tipos de máquinas}\label{subsec:alt_maquinas}
Una vez seleccionado el proveedor de nube y el servicio de orquestación, en
este caso \textit{Amazon ECS}, existen dos tipos de máquinas virtuales que se
pueden utilizar para desplegar los contenedores: \textit{EC2} y \textit{Fargate}.

\paragraph{Amazon EC2}, como ya se ha comentado, es el servicio de computación
tradicional de AWS que permite lanzar máquinas virtuales con arquitecturas
comunes de manera rápida. Al escoger esta opción, se tendría que gestionar el
\textit{tipo de instancia}\footnote{\url{
	https://aws.amazon.com/es/ec2/instance-types/
}} y la \textit{capacidad} de las máquinas, lo que puede acarrear más tiempo de
análisis y configuración.

\begin{figure}[H]
	\centering
	\includegraphics[width=0.1\textwidth]{logos/ec2.png}
	\caption{Logo de Amazon EC2~\textregistered}
	\label{fig:ec2}
\end{figure}

\paragraph{AWS Fargate}\footnote{\url{https://aws.amazon.com/es/fargate/}} es un
servicio de contenedores de AWS con un enfoque más moderno que EC2, ya que
permite ejecutar contenedores sin necesidad de gestionar las máquinas
subyacentes. Fargate se encarga de aprovisionar y escalar la infraestructura
necesaria para ejecutar los contenedores en base a los recursos definidos en la
configuración. Aunque Fargate es más sencillo de usar y configurar que EC2,
también puede ser más costoso y menos flexible, ya que no es posible acceder
directamente a las máquinas subyacentes de manera sencilla o personalizar la
configuración de las instancias.

\begin{figure}[H]
	\centering
	\includegraphics[width=0.1\textwidth]{logos/fargate.png}
	\caption{Logo de AWS Fargate~\textregistered}
	\label{fig:fargate}
\end{figure}

\paragraph{Decisión}
Dado que el proyecto no requiere una configuración específica de las máquinas
subyacentes y se busca una solución sencilla y rápida de desplegar, se decide
utilizar \textit{AWS Fargate} como servicio de contenedores para ejecutar los
servicios del sistema. Fargate es una opción más moderna y sencilla que EC2, y
permite centrarse en el desarrollo de las aplicaciones sin tener que preocuparse
por la gestión de la infraestructura subyacente.


\newpage{}
\section{Arquitectura del sistema}\label{sec:arquitectura}
Tras la definción de los requisitos y la valoración de las alternativas
disponibles, en este apartado se plantea la arquitectura completa del sistema
en la nube, tomando como proveedor a \textit{Amazon Web Services} (AWS), ya que
es el proveedor de nube preferido por la empresa.

La arquitectura definida a continuación deberá ser definida y desplegada de
manera automatizada mediante \textit{Terraform} contando con la mínima
intervención posible por parte de los operadores del sistema, y hará uso de
\textit{Amazon ECS} como servicio de orquestación de contenedores.

Además de \textit{ECS}, se utilizarán otros servicios de AWS necesarios para el
planteamiento de una arquitectura completa y funcional, como \textit{VPC},
\textit{IAM}, \textit{EFS}, \textit{S3}, \textit{SG}, entre otros. A
continuación, se detallan los servicios y componentes que formarán parte de la
arquitectura del sistema.

\begin{itemize}
	\item \textbf{IAM:} \textit{Identity and Access Management} es un servicio
		que permite gestionar el acceso a los recursos de AWS de forma segura.
		En este caso, se crearán roles y políticas de IAM para controlar el
		acceso a los servicios del sistema y garantizar la seguridad de los
		datos.
	\item \textbf{ALB/NLB:} Los \textit{Application} o \textit{Network Load
		Balancers} son servicios de balanceo de carga que permiten distribuir
		el tráfico entre los contenedores del sistema. En este caso, se
		configurará un ALB o NLB para equilibrar la carga entre los contenedores
		del sistema y garantizar la disponibilidad y la escalabilidad de los
		servicios. \begin{itemize}
			\item Los balanceadores de carga cuentan a su vez con varios
				componentes como \textit{Target Groups} y \textit{Listeners} que
				permiten configurar las reglas de enrutamiento y el tráfico de
				red.
			\item La diferencia entre ALB y NLB\footnote{\url{
					https://aws.amazon.com/es/compare/the-difference-between-the-difference-between-application-network-and-gateway-load-balancing/?nc1=h_ls
				}} radica en el nivel de la capa de red en la que operan, siendo
				ALB adecuado para aplicaciones web y NLB para aplicaciones de
				red de capa 4. Puesto que se usan servicios que operan
				mediante el protocolo TCP y no HTTP, se utilizará un NLB para
				garantizar la conectividad de dichos servicios. Sin embargo, el
				uso de un NLB conlleva mayor complejidad de configuración y
				mayores costes, además de limitar la capacidad de integración
				con otros servicios de AWS como el sistema de logs.
		\end{itemize}
	\item \textbf{Componentes de red:} Se configurarán varios componentes de
		red, como \textit{VPC}\footnote{\url{
			https://docs.aws.amazon.com/es_es/vpc/latest/userguide/what-is-amazon-vpc.html
		}}, \textit{Subnets}, \textit{Route Tables},
		\textit{Internet Gateways}, \textit{NAT Gateways}\ldots, para garantizar
		la conectividad y la seguridad de los servicios del sistema.
		% TODO: Añadir más detalles sobre los componentes de red.
	\item \textbf{EFS:} \textit{Elastic File System} es un servicio de
		almacenamiento de archivos que permite compartir archivos entre los
		contenedores del sistema. EFS es vital para almacenar los datos y
		configuraciones de manera compartida entre los contenedores.
	\item \textbf{S3:} \textit{Simple Storage Service} es un servicio de
		almacenamiento de objetos que permite almacenar y recuperar grandes
		volúmenes de datos de forma segura y escalable. Se usará S3 para
		almacenar los datos de los servicios del sistema.
	\item \textbf{CloudWatch:} \textit{CloudWatch} es un servicio de
		monitorización y gestión de logs que permite supervisar y analizar los
		recursos de AWS en tiempo real. Este servicio se utilizará durante el
		periodo de desarrollo de la infraestructura, cuando la ingesta de datos
		no esté completamente implementada.
	\item \textbf{SG:} Los \textit{Security Groups} son reglas de seguridad que
		definen qué tráfico está permitido o denegado en los recursos de AWS.
		Se tendrán que configurar grupos de seguridad para controlar el tráfico
		desde y hacia los servicios del sistema.
	\item \textbf{ECS:} Dentro de \textit{Elastic Container Service}, se
		configurarán los \underline{clústeres}, \underline{servicios},
		\underline{tareas} y \underline{contenedores} necesarios para
		ejecutar los servicios del sistema.
	\item \textbf{Route 53:} \textit{Route 53} es un servicio de DNS que permite
		rastrear y redirigir el tráfico de red a los recursos de AWS. En este
		caso, se utilizará Route 53 para gestionar los nombres de dominio de la
		empresa y redirigir el tráfico a los servicios del sistema.
	\item \textbf{Secret Manager:} \textit{Secrets Manager} es un servicio de
		gestión de secretos que permite almacenar y recuperar información
		sensible de forma segura. Se utilizará Secret Manager para
		gestionar las credenciales y claves de acceso de los servicios del
		sistema.
	\item \textbf{ACM:} \textit{Certificate Manager} es un servicio de gestión
		de certificados SSL/TLS que permite proteger las conexiones seguras
		entre los servicios del sistema. ACM gestionará los certificados SSL/TLS
		de los recursos de AWS.
\end{itemize}

A continuación, se presentan los diagramas de la arquitectura del sistema en AWS
para cada una de las áreas de estudio: infraestructura, seguridad y redes. En
todos los diagramas se resaltan en morado (con líneas intermitentes) la región y
en verde la nube virtual privada (\textit{VPC}) en la que se desplegarán los
servicios del sistema.


\newpage{}
\subsection{Infraestructura}\label{subsec:infra}
Para la infraestructura del sistema, se utilizará un clúster de \textit{ECS} con
cuatro servicios en total: uno dedicado a \textit{Elasticsearch}, otro para
\textit{Kibana}, un tercero para \textit{Logstash} y por último un servicio que
recoja \textit{Kafka} y, su dependencia, \textit{Zookeeper}. Cada uno de estos
servicios estará compuesto por tantas tareas como se requieran para garantizar
la disponibilidad y escalabilidad de los servicios, aunque inicialmente solo se
desplegará una tarea por servicio. Dentro de cada tarea se podrán encontrar los
correspondientes contenedores - imágenes de Docker que contienen el código y las
dependencias necesarias para ejecutar los servicios.

Cada uno de los servicios estará detrás de un \textit{ALB} o \textit{NLB} que se
encargará de distribuir el tráfico entre las tareas, garantizando la
disponibilidad y escalabilidad de los servicios.

Los \textit{ALB} están conectados directamente a un \textit{bucket} S3 que
almacena los logs de los servicios (\textit{NLB} no soporta esta funcionalidad).
Además, los servicios que necesiten almacenar datos de forma persistente (todos
menos Kafka) estarán conectados a un sistema de archivos \textit{EFS} que
permitirá compartir datos y configuraciones entre los contenedores del sistema.

\begin{figure}[H]
	\centerline{\includegraphics[width=1.3\textwidth]{infra/aws_infra.png}}
	\caption{Diagrama inicial de la infraestructura en AWS}

	\label{fig:aws_infra}
\end{figure}


\subsection{Seguridad}\label{subsec:seguridad}
A nivel de seguridad, se debe garantizar la protección de los datos y la
confidencialidad de la información. Para ello, se utilizarán varios servicios de
AWS, como \textit{IAM} para la gestión de roles y políticas de seguridad, o
\textit{grupos de seguridad} que permiten controlar el tráfico de red entre los
contenedores del sistema.

Cada uno de los componentes del \textit{clúster} de \textit{ECS} tendrá su
propio grupo de seguridad que permitirá controlar el tráfico de red según los
puertos y protocolos permitidos. Además, se deberán utilizar tan solo las
políticas y roles necesarios para garantizar la seguridad de los datos.

Además de los componentes de seguridad de AWS, se utilizarán protocolos seguros
como \textit{HTTPS} para la comunicación entre los servicios y se cifrará la
información haciendo uso de \textit{Secret Manager} para cargar y guardar las
claves necesarias.


\begin{figure}[H]
	\centerline{\includegraphics[width=1.3\textwidth]{infra/aws_seguridad.png}}
	\caption{Diagrama incial de la seguridad en AWS}
	\label{fig:aws_seguridad}
\end{figure}

En el diagrama anterior, se pueden observar los grupos de seguridad (resaltados
en azul) que engloban a los componentes, mientras que todos ellos se encuentran
dentro de una \textit{VPC}.

\subsection{Redes}\label{subsec:redes}
A nivel de configuración de redes, se establecen tres subredes: dos públicas y
otra privada, estando las públicas en zonas de disponibilidad diferentes. La
subred pública estará conectada a internet a través de un
\textit{Internet Gateway}, mientras que la subred privada estará conectada a
internet a través de un \textit{NAT Gateway}. Ambas subredes estarán conectadas
a una \textit{VPC} que permitirá la comunicación entre los servicios del
sistema.

\begin{figure}[H]
	\centerline{\includegraphics[width=1.3\textwidth]{infra/aws_redes.png}}
	\caption{Diagrama incial de las redes en AWS}
	\label{fig:aws_redes}
\end{figure}

En el diagrama se pueden observar las subredes (resaltadas en naranja con líneas
intermitentes), siendo las dos inferiores subredes públicas en diferentes zonas
de disponibilidad conectadas a un \textit{Internet Gateway} y la subred superior
una privada donde se encuentran los servicios junto con el \textit{NAT Gateway}
necesario para la conexión a Internet.

Además de los componentes planteados en el esquema anterior, se configurarán
\textit{rutas} y \textit{tablas de rutas} para garantizar la conectividad y la
seguridad de los servicios del sistema.


\newpage{}
\section{Modelos de datos}\label{sec:modelo}
Los modelos de datos son una representación estructurada de la información
almacenada que se utiliza para almacenar, recuperar y manipular los datos de
forma eficiente. En este caso, se definirán los modelos de datos que se
ingestarán en el sistema, así como las relaciones entre ellos y las
características de cada uno.

\emph{NOTA: debido a cuestiones de confidencialidad, no se mostrarán los modelos
de datos internos de Okticket.}

\subsection{Logs de balanceadores (AWS)}
\begin{figure}[H]
	\centering
	\includegraphics[width=0.65\textwidth]{uml/logs_elb.png}
	\caption{Modelo de datos de los logs de un balanceador de carga de AWS}
	\label{fig:logs_elb}
\end{figure}


\newpage{}
\subsection{Logs de bases de datos SQL}
\begin{figure}[H]
	\centering
	\includegraphics[width=0.4\textwidth]{uml/logs_sql.png}
	\caption{Modelo de datos de los logs de una base de datos SQL}
	\label{fig:logs_sql}
\end{figure}


\newpage{}
\subsection{Logs de bases de datos MongoDB}
\begin{figure}[H]
	\centering
	\includegraphics[width=0.4\textwidth]{uml/logs_mongo.png}
	\caption{Modelo de datos de los logs de una base de datos MongoDB}
	\label{fig:logs_mongo}
\end{figure}


\newpage{}
\subsection{Logs de Laravel (API)}
\begin{figure}[H]
	\centering
	\includegraphics[width=0.5\textwidth]{uml/logs_laravel.png}
	\caption{Modelo de datos de los logs de una aplicación Laravel}
	\label{fig:logs_laravel}
\end{figure}


\chapter{Implementación}
Durante la implementación, se ha seguido la planificación y metodologías
anteriormente descritas, dividiendo el proyecto en tareas más pequeñas y
manejables, para que se puedan realizar en un periodo de tiempo razonable.

Al seguir la prioridad de las tareas, se realizan primero las tareas más
críticas, como la creación de la infraestructura y la ingesta de las fuentes
esenciales, y se dejan para más adelante tareas como la visualización para
clientes externos o fuentes menos críticas y más complejas, como las APIs de
terceros o el \textit{web scraping}.


\section{Despliegue local}\label{sec:impl_local}
Para arrancar el desarrollo del proyecto, se decide construir una versión
local del sistema, que permita trabajar en un entorno controlado y sin
dependencias externas para comprobar su correcto funcionamiento y establecer
las bases de configuración y del posterior despliegue en la nube.

Puesto que se ha decidido utilizar Docker para la gestión de contenedores, se
ha creado un archivo \texttt{docker-compose.yml} que define los servicios
necesarios para el proyecto, es decir, \textit{Kafka}, \textit{Zookeeper},
\textit{Elasticsearch}, \textit{Kibana} y \textit{Logstash}, ignorando por
el momento la ingesta de datos y la escalabilidad del sistema.

Para la configuración de los servicios, se ha creado un archivo \texttt{.env}
que define las variables de entorno necesarias para el correcto funcionamiento
de los servicios, como las contraseñas o la versión del \textit{stack}.

Durante el arranque de los contenedores, se ejecuta un script de inicialización
que se encarga de crear las credenciales y los usuarios necesarios para el
funcionamiento de los servicios. Estas credenciales son necesarias ya que los
contenedores funcionan mediante tráfico HTTPS.

Los contenedores cuentan con comprobaciones de salud (o \textit{health-checks})
básicas para asegurar que los servicios se han arrancado correctamente y están
funcionando.

Esta sección de la documentación documenta el desarrollo de la historia de
usuario inicial, de acuerdo con lo establecio en la sección \ref{sec:planif_inicial}:

\begin{table}[H]
	\centering
	\begin{tabular}{|p{0.7\linewidth}|c|c|}
		\hline
		\textbf{Nombre} & \textbf{Prioridad} & \textbf{Tamaño} \\
		\hline
		\hline
		Creación de la infraestructura base (técnica) & P0\cellcolor{red!50} & L\cellcolor{orange!50} \\
		\hline
  	\end{tabular}
  	\caption{Lista de HUs cumplimentadas con el despliegue local}
  	\label{tab:impl_local}
\end{table}




\newpage{}
\subsection{Explicación del código}
El código de despliegue local se encuentra en \fullref{anexo:local}.

A nivel de configuración, se definen variables de entorno para cada contenedor
mediante el uso de la palabra clave \texttt{environment} y las claves definidas
por cada imagen. Para cada contenedor, se define además información adicional
dependiendo de las características del servicio, como la dependencia en otras
imágenes, los límites de recursos o los puertos de escucha.

Para evitar el ruido excesivo por consola una vez arrancado los servicios, se
reduce el nivel de \textit{logging} a \textit{WARN} en los servicios que lo
soporten.

Al comienzo del archivo \texttt{docker-compose.yml}, se definen los volúmenes y
las redes necesarias para el correcto funcionamiento de los servicios.

\begin{lstlisting}[style=yaml, caption={Definición de volúmenes y redes en Docker Compose}]
volumes:
	es01data:
	kibanadata:
	elasticdata:
	logstashdata:
	kafkadata:
	certs:

networks:
	default:
		driver: bridge
\end{lstlisting}

A continuación, se definen los servicios necesarios para el proyecto, comenzando
por el contenedor de preparación de credenciales y usuarios. Se utiliza una
imagen de Elasticsearch para la creación de las credenciales, y se monta un
volumen para la persistencia de las mismas.

\begin{lstlisting}[style=yaml, caption={Definición del servicio de preparación}]
setup:
	image: docker.elastic.co/elasticsearch/elasticsearch:${STACK_VERSION}
	volumes:
		- certs:/usr/share/elasticsearch/config/certs
	user: root
	container_name: setup
	command: >
		bash -c '
			if [ x${ELASTIC_PASSWORD} == x ]; then
			echo "Set the ELASTIC_PASSWORD environment variable in the .env file";
			exit 1;
			elif [ x${KIBANA_PASSWORD} == x ]; then
			echo "Set the KIBANA_PASSWORD environment variable in the .env file";
			exit 1;
			fi;
			if [ ! -f config/certs/ca.zip ]; then
			echo "Creating CA";
			bin/elasticsearch-certutil ca --silent --pem -out config/certs/ca.zip;
			unzip config/certs/ca.zip -d config/certs;
			fi;
			if [ ! -f config/certs/certs.zip ]; then
			echo "Creating certs";
			echo -ne \
			"instances:\n"\
			"  - name: es01\n"\
			"    dns:\n"\
			"      - es01\n"\
			"      - localhost\n"\
			"    ip:\n"\
			"      - 127.0.0.1\n"\
			"  - name: kibana\n"\
			"    dns:\n"\
			"      - kibana\n"\
			"      - localhost\n"\
			"    ip:\n"\
			"      - 127.0.0.1\n"\
			> config/certs/instances.yml;
			bin/elasticsearch-certutil cert --silent --pem -out config/certs/certs.zip --in config/certs/instances.yml --ca-cert config/certs/ca/ca.crt --ca-key config/certs/ca/ca.key;
			unzip config/certs/certs.zip -d config/certs;
			fi;
			echo "Setting file permissions"
			chown -R root:root config/certs;
			find . -type d -exec chmod 750 \{\} \;;
			find . -type f -exec chmod 640 \{\} \;;
			echo "Waiting for Elasticsearch availability";
			until curl -s --cacert config/certs/ca/ca.crt https://es01:9200 | grep -q "missing authentication credentials"; do sleep 1; done;
			echo "Setting kibana_system password";
			until curl -s -X POST --cacert config/certs/ca/ca.crt -u "elastic:${ELASTIC_PASSWORD}" -H "Content-Type: application/json" https://es01:9200/_security/user/kibana_system/_password -d "{\"password\":\"${KIBANA_PASSWORD}\"}" | grep -q "^{}"; do sleep 10; done;
			echo "All done!";
		''

	healthcheck:
		test: [ "CMD-SHELL", "[ -f config/certs/es01/es01.crt ]" ]
		interval: 5s
		timeout: 10s
		retries: 10
\end{lstlisting}

El servicio de preparación necesita tener una comprobación de salud, puesto que
el resto de contenedores lo tienen marcado como dependencia para su arranque. En
este caso, la comprobación consiste en la existencia de un archivo de
certificado.

Una vez definido el servicio de preparación, se definen los servicios de
\textit{Kafka} y \textit{Zookeeper}, que se basan en imágenes oficiales de
\textit{Confluent}.

\begin{lstlisting}[style=yaml, caption={Definición de los servicios de Kafka}]
zookeeper:
	container_name: zookeeper
	image: confluentinc/cp-zookeeper:latest
	environment:
	ZOOKEEPER_CLIENT_PORT: 2181
	ZOOKEEPER_TICK_TIME: 2000
	ZOO_LOG4J_PROP: WARN,CONSOLE
	ports:
	- 2181:2181

kafka:
	container_name: kafka
	image: confluentinc/cp-kafka:latest
	depends_on:
		- zookeeper
		- es01
	ports:
		- 9092:9092
		- 29092:29092
	environment:
		KAFKA_BROKER_ID: 1
		KAFKA_ZOOKEEPER_CONNECT: zookeeper:2181
		KAFKA_ADVERTISED_LISTENERS: LISTENER_DOCKER_INTERNAL://kafka:29092,LISTENER_DOCKER_EXTERNAL://localhost:9092
		KAFKA_LISTENER_SECURITY_PROTOCOL_MAP: LISTENER_DOCKER_INTERNAL:PLAINTEXT,LISTENER_DOCKER_EXTERNAL:PLAINTEXT
		KAFKA_INTER_BROKER_LISTENER_NAME: LISTENER_DOCKER_INTERNAL
		KAFKA_OFFSETS_TOPIC_REPLICATION_FACTOR: 1
		KAFKA_LOG4J_ROOT_LOGLEVEL: WARN
		KAFKA_TOOLS_LOG4J_LOGLEVEL: ERROR
		KAFKA_LOG4J_LOGGERS: 'kafka=WARN,kafka.controller=WARN,kafka.log.LogCleaner=WARN,state.change.logger=WARN,kafka.producer.async.DefaultEventHandler=WARN'
\end{lstlisting}

Ambos servicios cuentan con una serie de variables de entorno que definen su
configuración, como el puerto de escucha, el \textit{broker ID} o la dirección
de \textit{Zookeeper}.

Una vez definidos los servicios de Kafka, se define el servicio más crítico, el
contenedor de Elasticsearch, del que depende el funcionamiento de todo el
sistema.

\begin{lstlisting}[style=yaml, caption={Definición del servicio de Elasticsearch}]
es01:
    image: docker.elastic.co/elasticsearch/elasticsearch:${STACK_VERSION}
    container_name: es01
    restart: unless-stopped
    depends_on:
      - setup
    environment:
      - node.name=es01
      - cluster.name=${CLUSTER_NAME}
      - discovery.type=single-node
      - bootstrap.memory_lock=true
      - logger.level=WARN
      - ELASTIC_PASSWORD=${ELASTIC_PASSWORD}
      - xpack.security.enabled=true
      - xpack.security.http.ssl.enabled=true
      - xpack.security.http.ssl.key=certs/es01/es01.key
      - xpack.security.http.ssl.certificate=certs/es01/es01.crt
      - xpack.security.http.ssl.certificate_authorities=certs/ca/ca.crt
      - xpack.security.transport.ssl.enabled=true
      - xpack.security.transport.ssl.key=certs/es01/es01.key
      - xpack.security.transport.ssl.certificate=certs/es01/es01.crt
      - xpack.security.transport.ssl.certificate_authorities=certs/ca/ca.crt
      - xpack.security.transport.ssl.verification_mode=certificate
    ulimits:
      memlock:
        soft: -1
        hard: -1
      nofile:
        soft: 65536
        hard: 65536
    cap_add:
      - IPC_LOCK
    labels:
      co.elastic.logs/module: elasticsearch
      co.elastic.metrics/module: elasticsearch
    volumes:
      - es01data:/usr/share/elasticsearch/data
      - certs:/usr/share/elasticsearch/config/certs
    ports:
      - 9200:9200
    healthcheck:
      test:
        [
          "CMD-SHELL",
          "curl -s --cacert config/certs/ca/ca.crt https://localhost:9200 | grep -q 'missing authentication credentials'"
        ]
      interval: 10s
      timeout: 10s
      retries: 10
\end{lstlisting}

Debido a que se trata del servicio más importante y grande, se requieren muchas
opciones de configuración, como la limitación de recursos, la persistencia de
datos o la configuración de seguridad. Como en el resto de servicios, se define
una comprobación de salud que se encarga de comprobar que el servicio está
disponible.

Para simplificar lo máximo posible la arquitectura de este prototipo, tan solo
se define un nodo de Elasticsearch, aunque la configuración de escalabilidad
sería sencilla gracias al diseño de Docker.

Por último, se definen los servicios de Kibana y Logstash, que dependen de
Elastic.

\begin{lstlisting}[style=yaml, caption={Definición de los servicios de Kibana}]
kibana:
    container_name: kibana
    image: docker.elastic.co/kibana/kibana:${STACK_VERSION}
    restart: unless-stopped
    volumes:
      - kibanadata:/usr/share/kibana/data
      - certs:/usr/share/kibana/config/certs
    environment:
      SERVER_NAME: kibana
      SERVER_PORT: 5601
      SERVER_HOST: 0.0.0.0
      ELASTICSEARCH_HOSTS: https://es01:9200
      ELASTICSEARCH_USERNAME: kibana_system
      ELASTICSEARCH_PASSWORD: ${KIBANA_PASSWORD}
      ELASTICSEARCH_SSL_CERTIFICATEAUTHORITIES: config/certs/ca/ca.crt
      LOGGING_ROOT_LEVEL: warn
      XPACK_ENCRYPTEDSAVEDOBJECTS_ENCRYPTIONKEY: ${KEY}
      XPACK_REPORTING_ENCRYPTIONKEY: ${KEY}
      XPACK_SECURITY_ENCRYPTIONKEY: ${KEY}
    links:
      - es01
    depends_on:
      - setup
      - es01
    labels:
      co.elastic.logs/module: kibana
      co.elastic.metrics/module: kibana
    ports:
      - 5601:5601
    healthcheck:
      test:
        [
          "CMD-SHELL",
          "curl -s -I http://localhost:5601 | grep -q 'HTTP/1.1 302 Found'"
        ]
      interval: 10s
      timeout: 10s
      retries: 10
\end{lstlisting}

\begin{lstlisting}[style=yaml, caption={Definición de los servicios de Logstash}]
logstash:
    container_name: logstash
    image: docker.elastic.co/logstash/logstash:${STACK_VERSION}
    restart: unless-stopped
    user: root
    volumes:
      - logstashdata:/usr/share/logstash/data
      - certs:/usr/share/logstash/certs
    environment:
      - ELASTIC_HOSTS="https://es01:9200"
      - ELASTIC_USER="elastic"
      - ELASTIC_PASSWORD=${ELASTIC_PASSWORD}
      - log.level=warn
      - xpack.monitoring.enabled=false
    command: >
      bash -c "echo 'input { kafka { bootstrap_servers => \"kafka:29092\" topics => [\"laravel-logs\"] } } output { elasticsearch { hosts => [\"https://es01:9200\"] user => \"elastic\" password => \"${ELASTIC_PASSWORD}\" ssl => true cacert => \"certs/ca/ca.crt\" } }' > /usr/share/logstash/pipeline/logstash.conf; bin/logstash -f /usr/share/logstash/pipeline/logstash.conf"
    ports:
      - 5000:5000
    depends_on:
      - es01
      - kafka
      - setup
    labels:
      co.elastic.logs/module: logstash
      co.elastic.metrics/module: logstash
    links:
      - es01
      - kibana
\end{lstlisting}

Se hace uso de la red interna de Docker para facilitar la comunicación entre los
contenedores, y se definen comprobaciones de salud para asegurar que los
servicios se han arrancado correctamente. Para Logstash, se define un script de
lanzamiento que genera el archivo de configuración y ejecuta el servicio, aunque
podría montarse un volumen con el archivo de configuración ya generado.

\newpage{}
\subsection{Uso del sistema}
Una vez arrancados los contenedores mediante el comando
\texttt{docker compose up}, se puede acceder a los servicios a través de la
dirección local \texttt{localhost}. En el caso de Kibana, se puede acceder a
la interfaz de usuario mediante la dirección \texttt{localhost:5601}. En el
caso de Elasticsearch, se pueden hacer peticiones HTTPS a través de la dirección
\texttt{localhost:9200}. Para Kafka, Zookeeper y Logstash, se pueden hacer
peticiones a través de las direcciones \texttt{localhost:9092},
\texttt{localhost:2181} y \texttt{localhost:9600}, respectivamente.

\begin{figure}[H]
	\centering
	\includegraphics[width=\textwidth]{impl/local1.png}
	\caption{Inicio de sesión en Kibana}
	\label{fig:kibana_login}
\end{figure}

Una vez en la pantalla de inicio de sesión, se puede acceder con las credenciales
definidas en el archivo \texttt{.env} para el usuario \texttt{elastic}.

\begin{figure}[H]
	\centering
	\includegraphics[width=\textwidth]{impl/local2.png}
	\caption{Página de inicio de Kibana}
	\label{fig:kibana_start}
\end{figure}

Una vez aquí, se puede hacer click en la opción \texttt{Try sample data} para
cargar un conjunto de datos de ejemplo y probar la funcionalidad de Kibana con
Elasticsearch. Por supuesto, también se pueden probar la ingesta de datos a
través de Logstash y Kafka o, si así se desea, a través de los \textit{Beats}
especializados de ingesta directa de Elastic.

El manual de uso del sistema se encuentra en \fullref{sec:manual_usuario}.


\newpage{}
\subsection{Proceso de desarrollo}\label{subsec:impl_local_desarrollo}
Para el desarrollo del sistema, se ha seguido un proceso iterativo, comenzando
a partir del ejemplo oficial de Elastic para \textit{Docker Compose}\footnote{
  \url{https://www.elastic.co/blog/getting-started-with-the-elastic-stack-and-docker-compose}
}. A partir de este ejemplo, se añaden progresivamente configuraciones y
servicios, y se prueban las funcionalidades de cada uno de ellos, actualizando
con regularidad el código en el repositorio privado establecido.

\begin{figure}[H]
  \centering
  \includegraphics[width=\textwidth]{impl/commits.png}
  \caption{Ejemplo de commits en el repositorio privado}
  \label{fig:commits}
\end{figure}

Inicialmente, se prueban los servicios mínimos (Kibana y Elasticsearch) sin
HTTPS para comprobar su correcto funcionamiento. Una vez se ha comprobado que
los servicios funcionan correctamente, se añade la configuración de seguridad
y se prueban los servicios con HTTPS, ajustando más configuraciones como el
nivel de registro de logs o las comprobaciones de salud de los contenedores.

Una vez se ha comprobado que los servicios funcionan correctamente, se añaden
los servicios de Kafka, Zookeeper y Logstash, y se prueban las conexiones entre
los servicios, ajustando las configuraciones necesarias para que los servicios
se comuniquen correctamente.

\newpage{}
\section{Despliegue \textit{cloud}}\label{sec:impl_cloud}
El desarrollo principal del despliegue en la nube se concentra en la creación
de los scripts de \textit{Terraform} necesarios para la implementación de la
infraestructura planteada en el apartado \fullref{sec:arquitectura}. Para ello,
se divide el proyecto en scripts separados de manera que se puedan gestionar
los recursos y los servicios de manera independiente.

El diseño de una infraestructura base y el desarrollo de un prototipo de manera
local permiten tener una idea clara de los recursos necesarios y de las
características específicas de cada servicio, facilitándo la tarea de
desarrollo.

Para el desarollo, se hace uso de un repositorio privado en \textit{Bitbucket}
para el control de versiones y facilitar a la empresa la revisión y uso del
código. El código completo se encuentra en \fullref{anexo:cloud}.

\subsection{Proceso de desarrollo}\label{subsec:impl_cloud_desarrollo}
El proceso de desarrollo de los scripts de \textit{Terraform} parte de la
implementación original de la infraestructura en local, y se va adaptando a
las necesidades de la infraestructura en la nube, puesto que ambos comparten
similaridades (como la mayoría de la configuración de los servicios, la
estructura general de los mismos, las imágenes y versiones utilizadas, etc.).

Al igual que con el desarrollo local, se sigue un proceso iterativo, comenzando
por la creación de un solo servicio, en este caso Kafka, y continuando con el
resto de la arquitectura. Los primeros despliegues son tan solo pruebas de
concepto, con el objetivo de adaptarse a la infraestructura de la nube, el
funcionamiento de Terraform y la configuración de AWS.

Pese a que Terraform suele encargarse de la creación, modificación y destrucción
de los recursos de manera automática, existen casos en los que es necesaria la
intervención manual, como en la destrucción de los contenedores de
\textit{Secret Manager} o en la actualización de algunas configuraciones de los
recursos. Estos casos ocurrirán solo durante la fase de desarrollo, puesto que
se espera que, en la fase de producción, no sea necesario la reconfiguración de
los recursos y servicios.

La definición de las tareas de ECS durante el desarrollo queda registrado en la
sección correspondiente de AWS, cuyo código y configuraciones se puede consultar
si así se desea.

\begin{figure}[H]
	\centering
	\includegraphics[width=\textwidth]{impl/definitions.png}
	\caption{Ejemplo de definciones de tareas de ECS en AWS}
	\label{fig:definitions}
\end{figure}

\begin{figure}[H]
	\centering
	\includegraphics[width=\textwidth]{impl/ejemplo_definition.png}
	\caption{Ejemplo de definción de tarea (Kafka)}
	\label{fig:definition}
\end{figure}


\newpage{}
\subsection{Despliegue de la infraestructura}\label{subsec:impl_cloud_despliegue}
% TODO: desarrollar
% Aquí podría estar bien poner un diagrama de despliegue. Pero que se vea en el
% modelo que ese diagrama de despliegue no es el despliegue de tu proyecto.
% Es decir, el proyecto es crear un proceso de despliegue que hace un despliegue.
% Entonces que se vea que hay ese proceso y luego el diagrama de despliegue de lo
% que el proyecto permite desplegar


\newpage{}
\subsection{Explicación del código}\label{sec:impl_configuracion}
Los scripts de Terraform se dividen en varios archivos, cada uno de ellos con
una función específica, con el objetivo de facilitar la gestión y configuración
de los recursos y servicios. A continuación, se detallan dichos archivos y su
función en el proyecto.


\newpage{}
\section{Ingesta de datos}\label{sec:impl_ingesta}
Tras la creación y el despliegue de la infraestructura base, se procede a la
creación de los scripts de ingesta de datos, de manera escalonada y siguiendo
la prioridad de las fuentes de datos.

Puesto que se ha decidido utilizar Kafka como sistema de mensajería, se
desarrollan los scripts de ingesta de datos para que los datos se envíen a
Kafka y se procesen mediante Logstash y Elasticsearch.

Para la ingesta de datos, se han desarrollado scripts de Python que se encargan
de la lectura de los datos de las fuentes, su transformación y su envío a Kafka.
Estos scripts se ejecutan mediante un \textit{cron} que se encarga de la
ejecución periódica de los mismos.


\newpage{}
\section{Visualización de datos}\label{sec:impl_visualizacion}
Una vez se cuentan con datos en Elasticsearch, se puede comenzar el desarrollo
de la visualización de los mismos mediante Kibana. Para ello, se han desarrollado
paneles de visualización que permiten la monitorización de los datos en tiempo
real y la creación de informes y \textit{dashboards} personalizados para cada
modelo de datos contemplado.

\chapter{Manuales}\label{chap:manual}

\section{Manual de usuario}\label{sec:manual_usuario}

\section{Manual de instalación}\label{sec:manual_instalacion}

\chapter{Resultados y trabajo futuro}
El propósito de este capítulo es presentar las conclusiones obtenidas a partir
del desarrollo del proyecto, recopilar las dificultades encontradas y proponer
líneas de trabajo futuro en vista a la amplicación y mejora del sistema.

\section{Resultados}
El resultado del proyecto es un sistema de monitorización y análisis de datos
funcional y escalable, que permite a los usuarios obtener insights valiosos a
partir de la información recopilada.

Se ha logrado implementar una arquitectura robusta y flexible, basada en
tecnologías modernas y en la nube, que facilita la ingesta, procesamiento y
visualización de datos de manera eficiente y sencilla.

El sistema es capaz de ingestar datos de diversas fuentes, como bases de datos
internas, logs de aplicaciones y servicios externos, y de presentarlos de forma
clara y útil a través de Kibana.

La infraestructura se despliega y orquesta de manera automática en la nube,
permitiendo una gestión sencilla y eficiente del sistema para los
administradores.

Pese a que no se han completado todas las historias de usuario planificadas,
se han logrado los objetivos principales del proyecto, sentando las bases para
futuras iteraciones y logrando así entregar un producto mínimo viable (MVP) de
calidad.


\newpage{}
\section{Trabajo futuro}
El proyecto ha sentado las bases para un sistema de monitorización y análisis de
datos robusto y escalable. Sin embargo, existen áreas de mejora y ampliación que
podrían ser abordadas en futuras iteraciones.


\subsection{Historias de usuario restantes}
Lo primero de todo sería completar las historias de usuario menos críticas que
han quedado pendientes, como la ingesta de datos de APIs externas o el
\textit{web scraping}. Estas funcionalidades permitirían enriquecer los datos
disponibles y ampliar las fuentes de información.

\begin{table}[H]
	\centering
	\begin{tabular}{|p{0.7\linewidth}|c|c|}
		\hline
		\textbf{Nombre} & \textbf{Prioridad} & \textbf{Tamaño} \\
		\hline
		\hline
		Como desarrollador de Okticket, quiero que los datos contengan metadatos que faciliten su filtrado o búsqueda & P2\cellcolor{yellow!50} & S\cellcolor{green!25} \\
		\hline
		Como trabajador de Okticket, quiero poder ver y consultar datos de empresas cliente & P2\cellcolor{yellow!50} & M\cellcolor{yellow!50} \\
		\hline
		Como gestor de una empresa cliente, quiero poder ver información relevante sobre mi empresa que recoja Okticket & P2\cellcolor{yellow!50} & L\cellcolor{orange!50} \\
		\hline
		Como desarrollador de Okticket, quiero poder ingestar datos de APIs externas a la empresa & P2\cellcolor{yellow!50} & L\cellcolor{orange!50} \\
		\hline
		Como desarrollador de Okticket, quiero poder ingestar información de páginas web externas (\textit{scraping}) & P2\cellcolor{yellow!50} & XL\cellcolor{red!50} \\
		\hline
	\end{tabular}
	\caption{Historias de usuario restantes para futuras iteraciones}
	\label{tab:remaining_tasks}
\end{table}


\newpage{}
\subsection{Integración de lenguaje natural para búsqueda (DSL)}
Sería interesante explorar la posibilidad de integrar el sistema con
un sistema de búsqueda y filtrado de datos a través de lenguaje natural, como
\textit{Natural Language Processing} (NLP)\footnote{
	\url{https://www.elastic.co/guide/en/elasticsearch/reference/current/query-dsl-query-string-query.html}
}

Esto permitiría a los usuarios realizar consultas de manera más intuitiva y
eficiente, sin necesidad de conocer la sintaxis de Kibana Query Language (KQL).


\subsection{Aplicación de modelos de Lenguaje de Gran Escala (LLM)}
Desde la empresa, se ha propuesto la posibilidad de aplicar modelos de LLM
para la generación de texto automática, presumiblemente de manera agnóstica al
proveedor (OpenAI, Anthropic, Google\ldots).

Esto permitiría la generación de informes y análisis de manera automática a
partir de los datos recopilados, facilitando la toma de decisiones y la
comunicación de insights a los usuarios.

\subsection{Más perspectivas futuras}
Gracias a la flexibilidad del stack ELK, se podrían añadir nuevas fuentes de
datos y visualizaciones, como logs de otras partes de la aplicación (gestor web,
otros servicios internos como Hubspot o Holded, aplicación móvil\ldots) o
visualizaciones más avanzadas y personalizadas.

Lo bueno de haber diseñado la arquitectura de la manera que se ha hecho
es que se pueden añadir nuevas funcionalidades sin necesidad de modificar
la infraestructura existente, simplemente añadiendo nuevos servicios y
configuraciones.

La escalabilidad del sistema también es un punto a tener en cuenta. En caso de
necesitar más capacidad de procesamiento o almacenamiento, se podría establecer
un sistema de autoescalabilidad en base a las definiciones ya realizadas.

Por último, se podría explotar la funcionalidad del stack de generar alertas
en base a la información ingestada y procesada, para notificar a los usuarios
de eventos críticos o anomalías detectadas en los datos.


\newpage{}
\section{Conclusiones y retrospectiva}


%% Anexos
% \addcontentsline{toc}{chapter}{Anexos}
\appendix % Inicia los anexos
\chapter{Código de despliegue local}\label{anexo:local}
\lstinputlisting[style=yaml]{input/docker-compose.yml}
\lstinputlisting{input/.env.example}

\chapter{Script de creación de certificados}\label{anexo:certificados}
\lstinputlisting[language=bash, caption={Script de creación de credenciales}]{input/certs.sh}
% TODO: fix estilo bash no funcional


%% Esto incluirá la bibliografía correctamente en nuestro trabajo
\newpage % En una nueva página
\addcontentsline{toc}{chapter}{Bibliografía} % Añade la referencia al índice de contenido

\bibliographystyle{ieeetr} % Define el estilo de la bibliografía
\bibliography{biblio} % Indica el archivo que contiene la colección de citas

\nocite{template}
\nocite{mier2024anomalias}

\end{document}
