\documentclass[12pt, oneside, openany]{book}

\usepackage{mathptmx} % Contiene una fuente similar a Times New Roman

\usepackage[spanish, es-tabla]{babel} % Permite escritura en castellano
\usepackage[utf8]{inputenc} % Permite utilizar caracteres UTF8

\usepackage{graphicx} % Para la inclusión de gráficos e imágenes
\graphicspath{ {images/} } % Ruta para buscar las imágenes
\usepackage[a4paper,top=30mm,left=30mm,right=25mm,bottom=25mm,headheight=20mm]{geometry} % Configuración de los margenes de la página

% Paquetes para que funcione el formato.
\usepackage{titlesec}
\usepackage{setspace}
\usepackage{ragged2e}
\usepackage{fancyhdr}
\usepackage{lastpage}
\usepackage{stackengine}
\usepackage{array}
\usepackage[hidelinks]{hyperref}
\usepackage{enumitem}
\usepackage{float}
\usepackage{hypcap}
\usepackage{caption}
\usepackage{fancyvrb}

\usepackage{hyperref} % Paquete para que las referencias funcionen, y permite introducir links
\usepackage{xcolor} % Paquete para trabajar con colores (fondo de celdas, color del texto...)

\pgfplotsset{compat=1.18}
\usepgfplotslibrary{dateplot} % Se define un color gris desde su código RGB
\definecolor{gris}{RGB}{220,220,220}

\setcounter{secnumdepth}{3} % Para permitir numerar las sub-subsecciones

% Modifica el nombre de los índices al castellano
\addto\captionsspanish{
  \renewcommand{\contentsname}{Índice de contenido}
  \renewcommand{\listfigurename}{Índice de figuras}
  \renewcommand{\listtablename}{Índice de tablas}
  \renewcommand{\lstlistingname}{Listado}
  \renewcommand{\lstlistlistingname}{Índice de listados}
}

% Formateo de los nombres de los apartados:
\titleformat{\chapter}[block]
{\normalfont\Huge\bfseries\singlespacing}{\thechapter.}{1em}{\Huge}
\titlespacing*{\chapter}{0pt}{-62pt}{0pt}

\titleformat{\section}[block]
{\normalfont\Large\bfseries}{\thesection.}{4pt}{\Large}
\titlespacing*{\section}{0pt}{\baselineskip}{0pt}

\titleformat{\subsection}[block]
{\normalfont\large\bfseries}{\thesubsection.}{4pt}{\normalsize\large}
\titlespacing*{\subsection}{0pt}{0pt}{0pt}

\titleformat{\subsubsection}[block]
{\normalfont\normalsize\bfseries}{\thesubsubsection.}{4pt}{\normalsize}
\titlespacing*{\subsubsection}{0pt}{0pt}{0pt}

\def\tablename{Tabla}

%% Variables para portada y cabeceras
%% Cambiar los valores para cada documento!!!
\def\title{Explotación, integración y visualización de múltiples fuentes de datos mediante un Data Lake}
\def\shortTitle{ETL de datos heterogéneos mediante un Data Lake}
\def\subject{Lenguajes y Sistemas Informáticos}
\def\author{Mier Montoto, Juan Francisco}
\def\date{julio 2024}
\def\org{Escuela Politécnica de Ingeniería de Gijón}
\def\area{Grado en Ingeniería Informática en Tecnologías de la Información}
\def\tutorOne{D. Augusto Alonso, Cristian}
\def\tutorTwo{D. Morán Barbón, Jesús}
\def\tutorThree{D. Vázquez Faes, Eduardo}
\def\tfgId{???}
\def\dni{71777658V}

\def\ORG{\expandafter\MakeUppercase\expandafter{\org}}
\def\AREA{\expandafter\MakeUppercase\expandafter{\area}}
\def\SUBJECT{\expandafter\MakeUppercase\expandafter{\subject}}


\captionsetup{justification=centering}
\setlength{\headheight}{65pt}

\fancyhf{}
\fancyhead[L]{\includegraphics[height=8mm]{style/square.png}
  \hspace{3em} \Longstack[l] {
    \textbf{\SUBJECT} \newline
    \textbf{\shortTitle}}
  \newline \leftmark{}
}
\fancyhead[R]{\bfseries{Hoja \hyperlink{toc}{\thepage}~de~\pageref{LastPage}}}
\fancyfoot[C]{\author}
\renewcommand{\headrulewidth}{0pt} % default is 0pt
\renewcommand{\footrulewidth}{0.4pt} % default is 0

\fancypagestyle{plain}{%
  \fancyhf{}
  \fancyhead[L]{\includegraphics[height=8mm]{style/square.png}
    \hspace{3em} \Longstack[l]{
      \textbf{\SUBJECT} \newline
      \textbf{\shortTitle}}}
  \fancyhead[R]{\bfseries{Hoja \hyperlink{toc}{\thepage}~de~\pageref{LastPage}}}
  \fancyfoot[C]{\author}
  \renewcommand{\headrulewidth}{0pt} % default is 0pt
  \renewcommand{\footrulewidth}{0.4pt} % default is 0pt
}

\newcommand*{\fullref}[1]{\textit{\hyperref[{#1}]{\ref*{#1} \nameref*{#1}}}}
\newcommand*{\halfref}[2]{\texttt{\hyperref[{#1}]{#2}}}
\newcommand{\blankpage}{
  \newpage
  \thispagestyle{empty}
  \mbox{}
  \newpage
}

\pagestyle{fancy}

\restylefloat{table}
\definecolor{backcolour}{rgb}{0.95,0.95,0.92}
\definecolor{darkgreen}{rgb}{0.0, 0.5, 0.0}

\lstdefinestyle{default}{
  basicstyle=\ttfamily\footnotesize,
  breakatwhitespace=false,
  breaklines=true,
  captionpos=b,
  keepspaces=true,
  numbers=left,
  numbersep=5pt,
  numberstyle=\tiny\color{black},
  showspaces=false,
  showstringspaces=false,
  showtabs=false,
  tabsize=2,
  backgroundcolor=\color{backcolour},
  postbreak=\mbox{\textcolor{red}{$\hookrightarrow$}\space},
}

\lstset{style=default}

\lstdefinestyle{yaml}{
  basicstyle=\ttfamily\footnotesize\color{darkgreen}\bfseries,
  rulecolor=\color{black},
  string=[s]{'}{'},
  stringstyle=\color{darkgreen},
  comment=[l]{:},
  commentstyle=\color{blue},
  morecomment=[l]{-},
  numbers=left,
  numbersep=5pt,
  numberstyle=\tiny\color{black},
  backgroundcolor=\color{backcolour},
  breaklines=true,
  captionpos=b,
  keepspaces=true,
  showspaces=false,
  showstringspaces=false,
  showtabs=false,
  tabsize=2,
  breakatwhitespace=true,
  postbreak=\mbox{\textcolor{red}{$\hookrightarrow$}\space},
}

\pgfplotsset{compat=1.18}
\usepgfplotslibrary{dateplot}



\begin{document}

\rmfamily % Fuente tipo Romana

% Portada de la memoria
\begin{titlepage}
    \centering
    \bfseries {
        \null{}
        \vspace{0cm}
        \begin{table}[h]
            \centering
            \begin{tabular}{m{10cm} m{1cm} m{3cm}}
                \vspace{0.2cm}
                \includegraphics[width=70mm]{style/full.png} &
                \vspace{1.5mm} \includegraphics[width=25mm]{style/square.png} \\
            \end{tabular}
        \end{table}

        \vspace{3\baselineskip}

        \Large{\ORG{} \\ \vspace{3\baselineskip}}
        \large {
            \AREA{} \\ \vspace{3\baselineskip}
            \subject{} \\ \vspace{2\baselineskip}

            TRABAJO FIN DE GRADO/MÁSTER Nº \tfgId{} \vspace{\baselineskip} \\
            \title{} \\ \vspace{1\baselineskip}

            \author{} \\ \vspace{1\baselineskip}
            TUTORES:\\
            \tutorOne{} \\
            \tutorTwo{} \\
            \tutorThree{} \\ \vspace{\baselineskip}

            \vspace{2\baselineskip}
            FECHA:\@ \date{}
        }
    }
\end{titlepage}


% Índice de contenido
\addcontentsline{toc}{chapter}{Índice de contenido} % Añade la referencia al índice de contenido
\hypertarget{toc}{}
\tableofcontents
\newpage

% Índice de figuras
\addcontentsline{toc}{chapter}{Índice de figuras}  % Añade la referencia al índice de contenido
\hypertarget{lof}{}
\listoffigures

\justify{} % Texto justificado
\setlength{\parskip}{\baselineskip} % Separación entre párrafos de 1 linea
\onehalfspacing{}

%% El contenido de la memoria, dividido en capítulos:
\chapter{Introducción}\label{chap:intro}
\section{Antecedentes}\label{sec:antecedentes}
\subsection{Big data y evolución}\label{subsec:bigdata}
En la actualidad, la cantidad de datos que se generan y almacenan es cada vez mayor
\footnote{\url{https://www.statista.com/statistics/871513/worldwide-data-created/}}, una tendencia
que por supuesto se traduce a las empresas. Estos datos provienen de múltiples fuentes y en múltiples
formatos, lo que dificulta su análisis y explotación. A esta característica de la información se le
conoce como \textit{heterogeneidad}\footnote{\url{https://www.sciencedirect.com/topics/computer-science/data-heterogeneity}}.

A su vez, el progreso tecnológico ha permitido la creación de nuevas herramientas y técnicas
que facilitan la recogida, almacenamiento y análisis de estos datos. Una de estas técnicas son
los \textit{data lakes} (ver~\fullref{sec:datalake}),
que permiten almacenar grandes cantidades de datos de diferentes tipos y formatos, para poder
analizarlos y explotarlos de forma más eficiente.

\subsection{Visualización de datos y DIKW}\label{subsec:visual}
La visualización de datos es una técnica que permite representar la información de manera visual,
para facilitar su análisis y comprensión. La visualización de datos es una parte importante del
proceso de análisis de datos, ya que permite identificar patrones, tendencias y anomalías en los
datos de forma más rápida y sencilla.

La evolución de la visualización de datos ha ido de la mano de la evolución de la tecnología, y
actualmente existen múltiples herramientas y técnicas que permiten visualizar datos de forma
más eficiente y efectiva. Una de estas técnicas es el modelo DIKW (ver \fullref{sec:dikw}),
que describe el proceso de transformación de los datos en información, la información en
conocimiento y el conocimiento en sabiduría.


\newpage{}
\section{Motivación}\label{sec:motivacion}
El proyecto surge de la necesidad de la empresa (ver \fullref{sec:empresa}) de recoger y
analizar datos heterogéneos de todas las fuentes de las que se disponen, tanto internas
(e.g.~bases de datos, archivos de registros, APIs, entre otros), como externas
(e.g. APIs o datos de webs de terceros, datos de fuentes públicas\ldots).

En la actualidad, la empresa dispone de una gran cantidad de datos que se encuentran en
diferentes formatos y en diferentes ubicaciones, lo que dificulta su análisis y explotación.
Por otra parte, se depende de la consulta manual o de servicios de terceros (como dashboards
en NewRelic o AWS CloudWatch) para poder analizar estos datos, lo que supone un coste adicional.

Además del uso interno, la empresa también quiere ofrecer a sus clientes la posibilidad de
consultar estos datos de forma visual y sencilla, para que puedan analizarlos y explotarlos de
forma autónoma, lo que supondría un valor añadido para los mismos. Este tipo de dashboards
son diferentes a los dashboards de monitorización antes mencionados, ya que permiten al usuario
final la consulta de datos de negocio, y no de infraestructura.

\section{Finalidad del proyecto}\label{sec:finalidad}
El objetivo de este sistema es centralizar y unificar las fuentes de datos heterogéneas
cuya consulta se realiza de manera manual, con la finalidad de analizar los datos de forma
más eficiente.

El cumplimiento de este objetivo permitirá a la empresa obtener una serie de beneficios:
\begin{itemize}
	\item una eliminación del tiempo invertido en la consulta manual de los datos.
	\item una reducción de los costes de las plataformas de terceros.
	\item una mejora en la toma de decisiones, al poder analizar los datos de forma más eficiente.
	\item una mejora en la calidad de los servicios ofrecidos a los clientes, al poder ofrecerles
	      la posibilidad de consultar los datos de forma visual y sencilla.
	\item el beneficio económico que supondría la venta de este servicio a los clientes.
\end{itemize}

Además de la mejora de los procesos ya existentes, la explotación mediante esta herramienta
abrirá la puerta a nuevas posibilidades de análisis y explotación de los datos, como la detección
de anomalías en la infraestructura o la predicción de patrones y eventos futuros.

\section{La empresa}\label{sec:empresa}
Okticket es una startup nacida en Gijón en 2017 cuyo producto principal es un servicio software
que reduce los costes y el tiempo que invierten las empresas en contabilizar y manejar los gastos
de viaje de los profesionales mediante el escaneo automático de tickets y notas de gastos.

La empresa tienen su suede principal  en el Parque Tecnológico de Gijón, aunque cuenta con un número
de sedes creciente en varios países, como Francia, Portugal o, más recientemente, México. En esta
oficina principal se encuentran los departamentos de ventas y marketing, así como el equipo de
desarrollo y soporte.

Okticket es una de las empresas que más crecen tanto del sector como del propio Parque
Tecnológico. Debido a este rápido crecimiento, el equipo está en constante desarrollo y
cambio, tanto aquí en España como en el resto de sedes. Este crecimiento se refleja
en la recepción de un gran número de galardones y reconocimientos
\footnote{\href{https://www.linkedin.com/posts/okticket_okticket-en-el-especial-startups-de-forbes-activity-7140622980618903552-UGWK}{Okticket en el especial startups 2023 de Forbes (LinkedIn)}}
\footnote{\href{https://www.elcomercio.es/economia/arcelor-okticket-premios-20230222002438-ntvo.html}{Arcelor y Okticket, premios nacional de Ingeniería Informática (EL COMERCIO)}}
\footnote{\href{https://www.okticket.es/blog/empresa-pyme-innovadora}{Okticket recibe el sello Pyme Innovadora (okticket.es)}}
\footnote{\href{https://www.okticket.es/blog/okticket-empresa-emergente-certificada}{Okticket, empresa emergente certificada (okticket.es)}}

La parte principal del negocio es el núcleo del software como servicio (Software as a
Service en inglés, en adelante \textit{SaaS}), es decir, la aplicación completa tanto
para administradores como para empleados. Este SaaS se oferta a empresas de cualquier
tamaño, cuyo precio final varía en función del número de usuarios, las características
e integraciones que requiera la empresa cliente y el soporte que se ofrezca.

Recientemente se han añadido nuevas propuestas a la cartera de servicios ofertada por
Okticket, como la OKTCard {-} una tarjeta inteligente que gestiona automáticamente los gastos,
así como la inclusión de nuevos ``módulos'' de gestión de gastos y viajes.

\chapter{Resultados y trabajo futuro}
El propósito de este capítulo es presentar las conclusiones obtenidas a partir
del desarrollo del proyecto, recopilar las dificultades encontradas y proponer
líneas de trabajo futuro en vista a la amplicación y mejora del sistema.

\section{Resultados}
El resultado del proyecto es un sistema de monitorización y análisis de datos
funcional y escalable, que permite a los usuarios obtener insights valiosos a
partir de la información recopilada.

Se ha logrado implementar una arquitectura robusta y flexible, basada en
tecnologías modernas y en la nube, que facilita la ingesta, procesamiento y
visualización de datos de manera eficiente y sencilla.

El sistema es capaz de ingestar datos de diversas fuentes, como bases de datos
internas, logs de aplicaciones y servicios externos, y de presentarlos de forma
clara y útil a través de Kibana.

La infraestructura se despliega y orquesta de manera automática en la nube,
permitiendo una gestión sencilla y eficiente del sistema para los
administradores.

Pese a que no se han completado todas las historias de usuario planificadas,
se han logrado los objetivos principales del proyecto, sentando las bases para
futuras iteraciones y logrando así entregar un producto mínimo viable (MVP) de
calidad.


\newpage{}
\section{Trabajo futuro}
El proyecto ha sentado las bases para un sistema de monitorización y análisis de
datos robusto y escalable. Sin embargo, existen áreas de mejora y ampliación que
podrían ser abordadas en futuras iteraciones.


\subsection{Historias de usuario restantes}
Lo primero de todo sería completar las historias de usuario menos críticas que
han quedado pendientes, como la ingesta de datos de APIs externas o el
\textit{web scraping}. Estas funcionalidades permitirían enriquecer los datos
disponibles y ampliar las fuentes de información.

\begin{table}[H]
	\centering
	\begin{tabular}{|p{0.7\linewidth}|c|c|}
		\hline
		\textbf{Nombre} & \textbf{Prioridad} & \textbf{Tamaño} \\
		\hline
		\hline
		Como desarrollador de Okticket, quiero que los datos contengan metadatos que faciliten su filtrado o búsqueda & P2\cellcolor{yellow!50} & S\cellcolor{green!25} \\
		\hline
		Como trabajador de Okticket, quiero poder ver y consultar datos de empresas cliente & P2\cellcolor{yellow!50} & M\cellcolor{yellow!50} \\
		\hline
		Como gestor de una empresa cliente, quiero poder ver información relevante sobre mi empresa que recoja Okticket & P2\cellcolor{yellow!50} & L\cellcolor{orange!50} \\
		\hline
		Como desarrollador de Okticket, quiero poder ingestar datos de APIs externas a la empresa & P2\cellcolor{yellow!50} & L\cellcolor{orange!50} \\
		\hline
		Como desarrollador de Okticket, quiero poder ingestar información de páginas web externas (\textit{scraping}) & P2\cellcolor{yellow!50} & XL\cellcolor{red!50} \\
		\hline
	\end{tabular}
	\caption{Historias de usuario restantes para futuras iteraciones}
	\label{tab:remaining_tasks}
\end{table}


\newpage{}
\subsection{Integración de lenguaje natural para búsqueda (DSL)}
Sería interesante explorar la posibilidad de integrar el sistema con
un sistema de búsqueda y filtrado de datos a través de lenguaje natural, como
\textit{Natural Language Processing} (NLP)\footnote{
	\url{https://www.elastic.co/guide/en/elasticsearch/reference/current/query-dsl-query-string-query.html}
}

Esto permitiría a los usuarios realizar consultas de manera más intuitiva y
eficiente, sin necesidad de conocer la sintaxis de Kibana Query Language (KQL).


\subsection{Aplicación de modelos de Lenguaje de Gran Escala (LLM)}
Desde la empresa, se ha propuesto la posibilidad de aplicar modelos de LLM
para la generación de texto automática, presumiblemente de manera agnóstica al
proveedor (OpenAI, Anthropic, Google\ldots).

Esto permitiría la generación de informes y análisis de manera automática a
partir de los datos recopilados, facilitando la toma de decisiones y la
comunicación de insights a los usuarios.

\subsection{Más perspectivas futuras}
Gracias a la flexibilidad del stack ELK, se podrían añadir nuevas fuentes de
datos y visualizaciones, como logs de otras partes de la aplicación (gestor web,
otros servicios internos como Hubspot o Holded, aplicación móvil\ldots) o
visualizaciones más avanzadas y personalizadas.

Lo bueno de haber diseñado la arquitectura de la manera que se ha hecho
es que se pueden añadir nuevas funcionalidades sin necesidad de modificar
la infraestructura existente, simplemente añadiendo nuevos servicios y
configuraciones.

La escalabilidad del sistema también es un punto a tener en cuenta. En caso de
necesitar más capacidad de procesamiento o almacenamiento, se podría establecer
un sistema de autoescalabilidad en base a las definiciones ya realizadas.

Por último, se podría explotar la funcionalidad del stack de generar alertas
en base a la información ingestada y procesada, para notificar a los usuarios
de eventos críticos o anomalías detectadas en los datos.


\newpage{}
\section{Conclusiones y retrospectiva}


%% Esto incluirá la bibliografía correctamente en nuestro trabajo
\newpage % En una nueva página
\addcontentsline{toc}{chapter}{Bibliografía} % Añade la referencia al índice de contenido

\bibliographystyle{ieeetr} % Define el estilo de la bibliografía
\bibliography{biblio} % Indica el archivo que contiene la colección de citas

\nocite{template}

\end{document}
