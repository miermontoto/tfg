\chapter{Fundamento teórico}\label{chap:teoria}
\section{Definición de Data Lake}\label{sec:datalake}
Los \textit{data lakes}\footnote{\url{https://aws.amazon.com/es/what-is/data-lake/}}
son almacenes de datos que almacenan grandes cantidades de datos de manera no
estructurada~\cite{mier2023dashboards}. A diferencia de los \textit{data warehouses},
los \textit{data lakes} no tienen un esquema definido, lo que permite
almacenar datos de cualquier tipo y formato. Esto permite almacenar grandes
cantidades de datos sin tener que definir un esquema de antemano, lo que puede
ser útil en algunos casos. Sin embargo, esto también puede ser un inconveniente, ya
que no se puede realizar un análisis de negocio de los datos almacenados en un
data lake sin antes transformarlos en un formato estructurado.

\section{Procesos ETL}\label{sec:etl}
Los procesos ETL~\cite{mier2023dashboards} son procesos que combinan datos de múltiples
fuentes en un único destino, transformando los datos en un formato común. Estos procesos
se utilizan para extraer datos de diferentes fuentes, transformarlos en un formato común
y cargarlos en un dashboard para que se muestren de manera visual.

Los proceoss ETL deben de adaptarse a la estructura de los datos de la fuente de origen,
ya que cada sistema procesa diferentes tipos de datos, de diferentes fuentes y en diferentes formatos.

Una de las características clave de los procesos ETL es que sean escalables, ya que los datos
que se muestran en los dashboards suelen ser datos que se generan de manera continua, y por
lo tanto los procesos ETL deben ser capaces de procesar grandes cantidades de datos de manera
eficiente. En ocasiones, los procesos ETL se pueden realizar en \textit{streaming}, lo que
significa que los datos se procesan en tiempo real a medida que se generan.

\section{Dashboards}\label{sec:dashboards}
La palabra \textit{dashboard}, que traducido de manera literal significa \textit{cuadro de mandos},
es un término que se utiliza para referirse a cualquier interfaz gráfica que muestre información
relevante de manera visual sobre un proceso o negocio. Aunque el término se utiliza en
muchos ámbitos (puede incluir indicadores comerciales, de producción, de marketing, de
calidad, de recursos humanos\ldots)

En el ámbito de deste proyecto, los dashboards reflejan en tiempo real el rendimiento de
actividades o procesos de negocio, y se utilizan para tomar decisiones informadas sobre los
mismos. En el caso de una empresa, un dashboard puede mostrar desde el rendimiento de la
plataforma en tiempo real hasta un reflejo del de las ventas, y permitir a los directivos tomar
decisiones informadas sobre el futuro de la empresa.

\section{Componentes del stack}\label{sec:componentes}
En el \textit{stack} tecnológico escogido para el proyecto se manejan
diferentes términos y conceptos que son necesarios desarrollar para
entender el funcionamiento del sistema.

\subsection{Tópico}\label{subsec:topico}
Un tópico es una categoría a la que se envían los mensajes. Al proceso de enviar mensajes a
un tópico se le llama \emph{publicar}, y al proceso de leer mensajes de un tópico se le llama
\emph{consumir}.

\subsection{Productor}\label{subsec:productor}
El productor es el componente responsable de crear y enviar mensajes al cluster de Kafka.
Está separado del resto de los componentes y produce mensajes de manera asíncrona y rápida.

\subsection{Consumidor}\label{subsec:consumidor}
El consumidor es el componente responsable de leer los mensajes producidos por el productor.
Está suscrito a un \nameref{subsec:topico} a través del broker y consume los mensajes.

\subsection{Broker}\label{subsec:broker}
El broker es el componente responsable de recibir los mensajes producidos por el productor y
enviarlos a los consumidores. Es el intermediario entre los productores y los consumidores.

\subsection{Zookeeper}\label{subsec:zookeeper}
Zookeeper es un servicio de coordinación distribuida que se utiliza para gestionar y coordinar
los brokers de Kafka. Se encarga de mantener la información de los brokers y de los tópicos.
