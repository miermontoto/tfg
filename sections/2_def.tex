\chapter{Fundamento teórico}\label{chap:teoria}
\section{Definición de Data Lake}\label{sec:datalake}

\section{Procesos ETL}\label{sec:etl}
Los procesos ETL~\cite{mier2023dashboards} son procesos que combinan datos de múltiples
fuentes en un único destino, transformando los datos en un formato común. Estos procesos
se utilizan para extraer datos de diferentes fuentes, transformarlos en un formato común
y cargarlos en un dashboard para que se muestren de manera visual.

Estos procesos deben estar “personalizados” para cada caso, ya que cada sistema
procesa diferentes tipos de datos, de diferentes fuentes y en diferentes formatos. Además,
estos procesos deben ser escalables, ya que los datos que se muestran en los dashboards
suelen ser datos que se generan de manera continua, y por lo tanto los procesos ETL deben
ser capaces de procesar grandes cantidades de datos de manera eficiente.

\section{Dashboards}\label{sec:dashboards}
La palabra \textit{dashboard}, que traducido de manera literal significa \textit{cuadro de mandos},
es un término que se utiliza para referirse a cualquier interfaz gráfica que muestre información
relevante de manera visual sobre un proceso o negocio. Aunque el término se utiliza en
muchos ámbitos (puede incluir indicadores comerciales, de producción, de marketing, de
calidad, de recursos humanos\ldots)

En el ámbito de deste proyecto, los dashboards reflejan en tiempo real el rendimiento de
actividades o procesos de negocio, y se utilizan para tomar decisiones informadas sobre los
mismos. En el caso de una empresa, un dashboard puede mostrar desde el rendimiento de la
plataforma en tiempo real hasta un reflejo del de las ventas, y permitir a los directivos tomar decisiones informadas sobre el futuro de la empresa.

\section{Componentes del stack}\label{sec:componentes}
En el \textit{stack} tecnológico escogido para el proyecto se manejan
diferentes términos y conceptos que son necesarios desarrollar para
entender el funcionamiento del sistema.

\subsection{Productor}\label{subsec:productor}


\subsection{Consumidor}\label{subsec:consumidor}


\subsection{Tópico}\label{subsec:topico}


\subsection{Broker}\label{subsec:broker}


\subsection{Zookeeper}\label{subsec:zookeeper}
