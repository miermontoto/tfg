\chapter{Fundamento teórico}\label{chap:teoria}
\section{Data lake}\label{sec:datalake}
Los \textit{data lakes}\footnote{\url{https://aws.amazon.com/es/what-is/data-lake/}}
son almacenes de datos que almacenan grandes cantidades de datos de manera no
estructurada~\cite{mier2023dashboards}.

A diferencia de los \textit{data warehouses}, los \textit{data lakes} no tienen un
esquema definido, lo que permite almacenar datos \textit{heterogéneos}. Esto permite
almacenar grandes cantidades de datos sin tener que definir un esquema de antemano,
lo que puede ser útil en aquellos casos en los que no se conoce la estructura de los
datos que se van a almacenar.


\section{Modelo DIKW}\label{sec:dikw}
La pirámide DIKW\footnote{\url{https://en.wikipedia.org/wiki/DIKW_pyramid}} es un modelo que
describe la relación entre los datos, la información, el conocimiento y la sabiduría. Según este
modelo, los datos son la materia prima de la información, que a su vez es la materia prima del
conocimiento, que a su vez es la materia prima de la sabiduría.

La visualización de datos es una técnica que permite representar los datos de forma visual, para
facilitar su análisis y explotación. Los dashboards (ver~\fullref{sec:dashboards}) son una
herramienta que permite visualizar los datos de forma sencilla y eficiente, para poder tomar
decisiones informadas sobre los mismos.

\section{Procesos ETL}\label{sec:etl}
Los procesos ETL~\cite{mier2023dashboards} son procesos que combinan datos de múltiples
fuentes en un único destino, transformando los datos en un formato común. Estos procesos
se utilizan para extraer datos de diferentes fuentes, transformarlos en un formato común
y cargarlos en un destino común.

Los procesos ETL deben de adaptarse a la estructura de los datos de la fuente de origen,
ya que dichas fuentes pueden tener diferentes estructuras y tener tipos de datos diferentes,
(la \textit{heterogeneidad} que ya se ha mencionado).

Una de las características clave de los procesos ETL es que sean escalables, ya que los datos
que se muestran en los dashboards suelen ser datos que se generan de manera continua, y por
lo tanto los procesos ETL deben ser capaces de procesar grandes cantidades de datos de manera
eficiente. En ocasiones, los procesos ETL se pueden realizar en \textit{streaming}, lo que
significa que los datos se procesan en tiempo real a medida que se generan.

\section{Dashboards}\label{sec:dashboards}
La palabra \textit{dashboard}, que traducido de manera literal significa \textit{cuadro de mandos},
es un término que se utiliza para referirse a cualquier interfaz gráfica que muestre información
relevante de manera visual sobre un proceso o negocio. Aunque el término se utiliza en
muchos ámbitos (puede incluir indicadores comerciales, de producción, de marketing, de
calidad, de recursos humanos\ldots)

En el ámbito de deste proyecto, los dashboards reflejan en tiempo real el rendimiento de
actividades o procesos de negocio, y se utilizan para tomar decisiones informadas sobre los
mismos. En el caso de una empresa, un dashboard puede mostrar desde el rendimiento de la
plataforma en tiempo real hasta un reflejo del de las ventas, y permitir a los directivos tomar
decisiones informadas sobre el futuro de la empresa.

Para el sistema que se describe, se plantean dos tipos de dashboards diferentes:
\begin{itemize}
	\item \textbf{Dashboards internos}: que reflejan el rendimiento de la plataforma en tiempo real.
	\item \textbf{Dashboards externos}: que reflejan el rendimiento de las ventas y permiten a los
	      clientes tomar decisiones informadas sobre su negocio.
\end{itemize}
\newpage{}
\section{Componentes del sistema}\label{sec:componentes}
En el \textit{stack} tecnológico escogido para el proyecto se manejan
diferentes términos y conceptos que son necesarios desarrollar para
entender el funcionamiento del sistema.

\subsection{Tópico}\label{subsec:topico}
Un tópico es una categoría a la que se envían los mensajes a la que los consumidores están
\textit{suscritos}. Los consumidores pueden estar suscritos a uno o varios tópicos, y los
productores pueden enviar mensajes a uno o varios tópicos. Los tópicos son la unidad básica
de organización de los mensajes en cualquier sistema de mensajería de publicación/suscripción.

\subsection{Productor}\label{subsec:productor}
El productor es el componente responsable de crear y enviar mensajes al cluster de Kafka.
Está separado del resto de los componentes y produce mensajes de manera asíncrona y rápida.

\subsection{Consumidor}\label{subsec:consumidor}
El consumidor es el componente responsable de leer los mensajes producidos por el productor.
Está suscrito a un \nameref{subsec:topico} a través del broker y consume los mensajes.

\subsection{Broker}\label{subsec:broker}
El broker es el componente responsable de recibir los mensajes producidos por el productor y
enviarlos a los consumidores. Es el intermediario entre los productores y los consumidores.

\subsection{Zookeeper}\label{subsec:zookeeper}
Zookeeper es un servicio de coordinación distribuida que se utiliza para gestionar y coordinar
los brokers de Kafka. Se encarga de mantener la información de los brokers y de los tópicos.
