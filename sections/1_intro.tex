\chapter{Introducción}\label{chap:intro}
\section{Antecedentes}\label{sec:antecedentes}
En la actualidad, la cantidad de datos que se generan y almacenan es cada vez mayor
\footnote{\url{https://www.statista.com/statistics/871513/worldwide-data-created/}}, una tendencia
que por supuesto se traduce a las empresas. Estos datos provienen de múltiples fuentes y en múltiples
formatos, lo que dificulta su análisis y explotación. A esta característica de la información se le
conoce como \textit{heterogeneidad}\footnote{\url{https://www.sciencedirect.com/topics/computer-science/data-heterogeneity}}.

A su vez, el progreso tecnológico ha permitido la creación de nuevas herramientas y técnicas
que facilitan la recogida, almacenamiento y análisis de estos datos. Una de estas técnicas son
los \textit{data lakes} (ver~\fullref{sec:datalake}),
que permiten almacenar grandes cantidades de datos de diferentes tipos y formatos, para poder
analizarlos y explotarlos de forma más eficiente.

De manera paralela, el progreso en la \textit{ciencia de datos} ha supuesto una mejora en la
visualización de estos, lo que permite un mejor análisis. La visualización de datos suele
significar la existencia de uno o varios \textit{dashboards}~\cite{mier2023dashboards}, que
permiten al usuario final la consulta rápida y sencilla de los datos sin necesidad de conocimientos técnicos del sistema.

\section{Motivación}\label{sec:motivacion}
El proyecto surge de la necesidad de la empresa (ver \fullref{sec:empresa}) de recoger y
analizar datos heterogéneos de todas las fuentes de las que se disponen, tanto internas
(e.g.~bases de datos, archivos de registros, APIs, entre otros), como externas
(e.g. APIs o datos de webs de terceros, datos de fuentes públicas\ldots).

En la actualidad, la empresa dispone de una gran cantidad de datos que se encuentran en
diferentes formatos y en diferentes ubicaciones, lo que dificulta su análisis y explotación.
Por otra parte, se depende de la consulta manual o de servicios de terceros (como dashboards
en NewRelic o AWS CloudWatch) para poder analizar estos datos, lo que supone un coste adicional.

Además del uso interno, la empresa también quiere ofrecer a sus clientes la posibilidad de
consultar estos datos de forma visual y sencilla, para que puedan analizarlos y explotarlos de
forma autónoma, lo que supondría un valor añadido para los mismos. Este tipo de dashboards
son diferentes a los dashboards de monitorización antes mencionados, ya que permiten al usuario
final la consulta de datos de negocio, y no de infraestructura.

\section{Finalidad del proyecto}\label{sec:finalidad}
El objetivo de este sistema es centralizar y unificar las fuentes de datos heterogéneas
cuya consulta se realiza de manera manual, con la finalidad de analizar los datos de forma
más eficiente.

El cumplimiento de este objetivo permitirá a la empresa obtener una serie de beneficios:
\begin{itemize}
	\item una eliminación del tiempo invertido en la consulta manual de los datos.
	\item una reducción de los costes de las plataformas de terceros.
	\item una mejora en la toma de decisiones, al poder analizar los datos de forma más eficiente.
	\item una mejora en la calidad de los servicios ofrecidos a los clientes, al poder ofrecerles
	      la posibilidad de consultar los datos de forma visual y sencilla.
	\item el beneficio económico que supondría la venta de este servicio a los clientes.
\end{itemize}

Además de la mejora de los procesos ya existentes, la explotación mediante esta herramienta
abrirá la puerta a nuevas posibilidades de análisis y explotación de los datos, como la detección
de anomalías en la infraestructura o la predicción de patrones y eventos futuros.

\section{La empresa}\label{sec:empresa}
Okticket es una startup nacida en Gijón en 2017 cuyo producto principal es un servicio software
que reduce los costes y el tiempo que invierten las empresas en contabilizar y manejar los gastos
de viaje de los profesionales mediante el escaneo automático de tickets y notas de gastos.

Sus oficinas principales (incluyendo la zona de desarrollo) se encuentran en el Parque
Tecnológico de Gijón, aunque cuenta con un número de sedes creciente en varios países, como
Francia, Portugal o, más recientemente, México.

Okticket es una de las empresas que más crecen tanto del sector como del propio Parque
Tecnológico. Debido a este rápido crecimiento, el equipo está en constante desarrollo y
cambio, tanto aquí en España como en el resto de sedes. Este crecimiento se refleja
en la recepción de un gran número de galardones y reconocimientos
\footnote{\href{https://www.linkedin.com/posts/okticket_okticket-en-el-especial-startups-de-forbes-activity-7140622980618903552-UGWK}{Okticket en el especial startups 2023 de Forbes (LinkedIn)}}
\footnote{\href{https://www.elcomercio.es/economia/arcelor-okticket-premios-20230222002438-ntvo.html}{Arcelor y Okticket, premios nacional de Ingeniería Informática (EL COMERCIO)}}
\footnote{\href{https://www.okticket.es/blog/empresa-pyme-innovadora}{Okticket recibe el sello Pyme Innovadora (okticket.es)}}
\footnote{\href{https://www.okticket.es/blog/okticket-empresa-emergente-certificada}{Okticket, empresa emergente certificada (okticket.es)}}

La parte principal del negocio es el núcleo del software como servicio (Software as a
Service en inglés, en adelante \textit{SaaS}), es decir, la aplicación completa tanto
para administradores como para empleados. Este SaaS se oferta a empresas de cualquier
tamaño, cuyo precio final varía en función del número de usuarios, las características
e integraciones que requiera la empresa cliente y el soporte que se ofrezca.

Recientemente se han añadido nuevas propuestas a la cartera de servicios ofertada por
Okticket, como la OKTCard {-} una tarjeta inteligente que gestiona automáticamente los gastos,
así como la inclusión de nuevos ``módulos'' de gestión de gastos y viajes.
