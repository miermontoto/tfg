\chapter{Introducción}\label{chap:intro}
\section{Antecedentes}\label{sec:antecedentes}
En la actualidad, la cantidad de datos que se generan y almacenan en las empresas es cada vez
mayor. Estos datos provienen de múltiples fuentes y en múltiples formatos, lo que dificulta su
análisis y explotación.

A su vez, el progreso tecnológico ha permitido la creación de nuevas herramientas y técnicas
que facilitan la recogida, almacenamiento y análisis de estos datos. Una de estas técnicas son
los \textit{data lakes}, que permiten almacenar grandes cantidades de datos de diferentes
tipos y formatos, para poder analizarlos y explotarlos de forma más eficiente.

De manera paralela, el progreso en la \textit{ciencia de datos} ha supuesto una mejora en la
visualización de estos, lo que permite un mejor análisis. La visualización de datos suele
significar la existencia de uno o varios \textit{dashboards}, que permiten al usuario final
la consulta rápida y sencilla de los datos sin necesidad de conocimientos técnicos del sistema.

\section{Motivación}\label{sec:motivacion}
El proyecto surge de la necesidad de \fullref{sec:empresa} de recoger y analizar
datos de múltiples características de todas las fuentes de las que se disponen, tanto internas
(como bases de datos, logs, la propia API, etc.) como externas (como APIs de otras empresas,
datos web de terceros, etc.).

En la actualidad, la empresa dispone de una gran cantidad de datos que se encuentran en
diferentes formatos y en diferentes ubicaciones, lo que dificulta su análisis y explotación.
Por otra parte, se depende de la consulta manual o de servicios de terceros (como dashboards
en NewRelic o AWS CloudWatch) para poder analizar estos datos, lo que supone un coste adicional.

Además del uso interno, la empresa también quiere ofrecer a sus clientes la posibilidad de
consultar estos datos de forma visual y sencilla, para que puedan analizarlos y explotarlos de
forma autónoma, lo que supondría un valor añadido para los mismos.

\section{Finalidad del proyecto}\label{sec:finalidad}
El objetivo de este sistema es centralizar y unificar todas estas fuentes que se encuentran
dispersas, en diferentes formatos y que en la actualidad se tienen que consultar manualmente,
para poder realizar análisis de los datos de forma más eficiente.

Como se comenta anteriormente, el cumplimiento de este objetivo permitirá:
\begin{itemize}
	\item una eliminación del tiempo invertido en la consulta manual de los datos.
	\item una reducción de los costes de las plataformas de terceros.
	\item una mejora en la toma de decisiones, al poder analizar los datos de forma más eficiente.
	\item una mejora en la calidad de los servicios ofrecidos a los clientes, al poder ofrecerles
	      la posibilidad de consultar los datos de forma visual y sencilla.
	\item el beneficio económico que supondría la venta de este servicio a los clientes.
\end{itemize}

Además de la mejora de los procesos ya existentes, la explotación de esta herramienta abrirá la
puerta a nuevas posibilidades de análisis y explotación de los datos, como la detección de
anomalías en la infraestructura o la predicción de patrones y eventos futuros.

\section{La empresa}\label{sec:empresa}
Okticket es una startup nacida en Gijón en 2017 cuyo producto principal es un servicio software
que reduce los costes y el tiempo que invierten las empresas en contabilizar y manejar los gastos
de viaje de los profesionales mediante el escaneo automático de tickets y notas de gastos.

Sus oficinas principales (incluyendo la zona de desarrollo) se encuentran en el Parque
Tecnológico de Gijón, aunque cuenta con un número de sedes creciente en varios países:
Francia, Portugal y, más recientemente, México.

Okticket es una de las empresas que más crecen tanto del sector como del propio Parque
Tecnológico. Debido a este rápido crecimiento, el equipo está en constante desarrollo y
cambio, tanto aquí en España como en el resto de sedes.

Pese a que la parte principal del negocio es el SaaS (Software as a Service en inglés),
es decir, la aplicación completa tanto para administradores como para empleados,
recientemente se han añadido nuevas propuestas como la OKTCard, una tarjeta inteligente
que gestiona automáticamente los gastos, entre otros proyectos.
