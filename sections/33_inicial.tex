\section{Descripción del proyecto}\label{sec:descripcion}
% TODO: describir el proyecto en términos generales.
% 		debe quedar claro que el usuario puede hacer muchos despliegues y qué es
% 		lo que se despliega (aunque no se sepa qué clústeres, nodos, etc.) ni
% 		las tecnologías de despliegue ni análisis. Se tiene que describir qué
%		desplegar. También tiene que verse qué usuarios hay y su rol. Quizás
%		haya un ingeniero de datos que desliega y un analista de datos que lo
%		utiliza.
%
%		Esto suele quedar bien en forma de gráfico con una serie de procesos
%		(ej. desplegar, ...) o como un stakck Proveedor en la nube, por encima
%		un sistema BigData... Puede que incluso sirva de inspiración el apartado
%		\ref{sec:alternativas}.

El proyecto consiste en el desarrollo de una arquitectura de análisis de datos
que permita obtener insights valiosos a partir de la información recopilada a
partir de la extracción, transformación y carga de datos de diversas fuentes.

Además, el despliegue de la infraestructura se debe realizar de manera
automática en la nube, de manera que se facilite la gestión sencilla y eficiente
del sistema para los administradores.

El sistema es capaz de ingestar datos de diversas fuentes, como bases de datos
internas, logs de aplicaciones y servicios externos, y de presentarlos de forma
clara y útil a través de dashboards personalizados.

Los usuarios finales podrán acceder a los dashboards a través de un servicio web
y visualizar los datos en tiempo real, así como filtrar la información mostrada
según diferentes campos.


\subsection{Roles y usuarios}
En el proyecto se distinguen los siguientes roles:

\begin{itemize}
	\item \textbf{Arquitecto:} responsable de diseñar la arquitectura de la
		plataforma y definir las tecnologías a utilizar.
	\item \textbf{Administrador/Desarrollador (DevOps):} responsable de
		desplegar y orquestar la infraestructura de manera automática.
	\item \textbf{Analista:} responsable de analizar los datos y sacar
		conclusiones aplicables al negocio a partir de los mismos.
\end{itemize}

A parte de estos roles, también se distinguen los siguientes usuarios:

\begin{itemize}
	\item \textbf{Usuario interno:} empleado de la empresa que utiliza los
		dashboards para monitorizar el rendimiento de la plataforma.
	\item \textbf{Usuario externo:} cliente de la empresa que utiliza los
		dashboards para tomar decisiones informadas sobre su negocio.
\end{itemize}


\newpage{}
\subsection{Dashboards planteados}
Para el sistema que se describe, se plantean dos tipos de dashboards diferentes:

\begin{itemize}
	\item \textbf{Dashboards internos:} que reflejan el rendimiento de la
		plataforma en tiempo real. Estos dashboards están destinados al uso
		interno de la empresa, y permiten a los empleados monitorizar el
		rendimiento de la plataforma y tomar decisiones informadas sobre su
		mantenimiento y evolución.
	\item \textbf{Dashboards externos:} que reflejan el rendimiento de las
		ventas y permiten a los clientes tomar decisiones informadas sobre su
		negocio. Estos dashboards están enfocados a los clientes de la empresa,
		y permite a los mismos obtener información relevante sobre su negocio
		que tenga Okticket.
\end{itemize}


\subsection{Proceso de despliegue}
El proceso de despliegue de la infraestructura se realizará de manera automática
en la nube, utilizando herramientas de orquestación como Terraform y Ansible.
Puesto que se trata de un \textit{stack ELK}, se deberán desplegar los servicios,
bien de manera separada mediante contenedores o clústeres, o bien de manera
conjunta mediante una solución tradicional de máquinas virtuales.

El sistema se ha de poder desplegar rápida y sencillamente, de manera que se
facilite la gestión y mantenimiento del mismo para los administradores en caso
de necesitar actualizar, configurar o escalar los servicios.

Por requisitos de la empresa, la solución deberá estar desplegada en la nube,
por lo que se deberá elegir un proveedor de servicios en la nube y desplegar la
infraestructura en el mismo.


\newpage{}
\subsection{Características del sistema}
El sistema debe cumplir con las siguientes características:

\begin{itemize}
	\item \textbf{Escalabilidad:} el sistema debe ser capaz de escalar
		horizontalmente para soportar un gran volumen de datos.
	\item \textbf{Flexibilidad:} el sistema debe ser flexible y permitir la
		ingesta de datos de diversas fuentes.
	\item \textbf{Robustez:} el sistema debe ser robusto y tolerante a fallos,
		para garantizar la disponibilidad de los datos en todo momento.
	\item \textbf{Seguridad:} el sistema debe ser seguro y proteger la
		información sensible de los usuarios, al tratar información sensible de
		clientes de Okticket.
\end{itemize}
