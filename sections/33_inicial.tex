\section{Descripción del proyecto}\label{sec:descripcion}
% TODO: describir el proyecto en términos generales.
% 		debe quedar claro que el usuario puede hacer muchos despliegues y qué es
% 		lo que se despliega (aunque no se sepa qué clústeres, nodos, etc.) ni
% 		las tecnologías de despliegue ni análisis. Se tiene que describir qué
%		desplegar. También tiene que verse qué usuarios hay y su rol. Quizás
%		haya un ingeniero de datos que desliega y un analista de datos que lo
%		utiliza.
%
%		Esto suele quedar bien en forma de gráfico con una serie de procesos
%		(ej. desplegar, ...) o como un stakck Proveedor en la nube, por encima
%		un sistema BigData... Puede que incluso sirva de inspiración el apartado
%		\ref{sec:alternativas}.

\subsection{Dashboards planteados}
Para el sistema que se describe, se plantean dos tipos de dashboards diferentes:

\begin{itemize}
	\item \textbf{Dashboards internos:} que reflejan el rendimiento de la
		plataforma en tiempo real. Estos dashboards están destinados al uso
		interno de la empresa, y permiten a los empleados monitorizar el
		rendimiento de la plataforma y tomar decisiones informadas sobre su
		mantenimiento y evolución.
	\item \textbf{Dashboards externos:} que reflejan el rendimiento de las
		ventas y permiten a los clientes tomar decisiones informadas sobre su
		negocio. Estos dashboards están enfocados a los clientes de la empresa,
		y permite a los mismos obtener información relevante sobre su negocio
		que tenga Okticket.
\end{itemize}
