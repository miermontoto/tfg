\chapter{Análisis del sistema}
En este capítulo se presenta un análisis detallado del sistema a desarrollar,
desglosando las funcionalidades principales en epics y historias de usuario.
Este enfoque permite una visión estructurada del proyecto, facilitando su
planificación y desarrollo.

\section{Epics}
Los \textit{epics} representan las grandes áreas funcionales del proyecto.
Se han identificado las siguientes historias épicas:

\begin{enumerate}
    \item Infraestructura y despliegue
    \item Ingesta de datos
    \item Procesamiento y almacenamiento de datos
    \item Visualización y análisis
\end{enumerate}

Cada uno de estos epics engloba un conjunto de funcionalidades relacionadas
que, en conjunto, conforman el proyecto planteado.

\newpage{}
\section{Historias de usuario}
Las historias de usuario describen las funcionalidades específicas desde la
perspectiva del usuario final. A continuación, se detallan las historias de
usuario para cada epic, incluyendo sus criterios de aceptación:


\subsection{Epic 1: Infraestructura y despliegue}
Este epic se centra en la creación y gestión de la infraestructura necesaria
para el sistema.

\begin{itemize}
    \item \textbf{HU1.1:} Como desarrollador de Okticket, quiero poder
    desplegar un prototipo del sistema en mi entorno local para facilitar el
    desarrollo y las pruebas iniciales.
    \begin{itemize}
        \item El sistema se desplegará correctamente en un entorno local.
        \item Todos los servicios estarán funcionando.
        \item Se podrá acceder a las interfaces web de los servicios.
    \end{itemize}

    \item \textbf{HU1.2:} Como desarrollador de Okticket, quiero que la
    arquitectura se despliegue y orqueste de manera automática en la nube para
    facilitar la gestión y el paso a producción del sistema.
    \begin{itemize}
        \item El sistema se desplegará automáticamente con un solo comando.
        \item Todos los recursos en la nube se crearán correctamente.
        \item Los servicios serán accesibles a través de URLs públicas.
    \end{itemize}

    \item \textbf{HU1.3:} Como administrador del sistema, quiero que la
    infraestructura sea capaz de escalar automáticamente en función de la
    demanda para optimizar el rendimiento y los costos.
    \begin{itemize}
        \item Se habrán configurado políticas de auto-escalado para los
        servicios críticos.
        \item El sistema aumentará los recursos cuando la carga aumente.
        \item El sistema reducirá los recursos cuando la demanda disminuya.
    \end{itemize}
\end{itemize}


\newpage{}
\subsection{Epic 2: Ingesta de datos}
La ingesta de datos es fundamental para el funcionamiento del sistema,
abarcando diversas fuentes de información.

\begin{itemize}
    \item \textbf{HU2.1:} Como desarrollador de Okticket, quiero que se
    ingesten de manera automática datos de la base de datos interna de MongoDB
    para centralizar la información.
    \begin{itemize}
        \item Los datos de MongoDB se ingestarán correctamente en el sistema.
        \item La ingesta se realizará de forma periódica y automática.
        \item Se podrá verificar la integridad de los datos ingestados.
    \end{itemize}

    \item \textbf{HU2.2:} Como desarrollador de Okticket, quiero que se
    ingesten de manera automática datos de la base de datos interna de MySQL
    para tener una visión completa de los datos.
    \begin{itemize}
        \item Los datos de MySQL se ingestarán correctamente en el sistema.
        \item La ingesta se realizará de forma periódica y automática.
        \item Se podrá verificar la integridad de los datos ingestados.
    \end{itemize}

    \item \textbf{HU2.3:} Como desarrollador de Okticket, quiero que se
    ingesten de manera automática logs de balanceador de AWS para monitorear
    el rendimiento de la infraestructura.
    \begin{itemize}
        \item Los logs del balanceador de AWS se ingestarán correctamente.
        \item La ingesta se realizará en tiempo real o con un retraso mínimo.
        \item Los logs ingestados incluirán toda la información relevante.
    \end{itemize}

    \item \textbf{HU2.4:} Como desarrollador de Okticket, quiero poder ingestar
    datos de APIs externas a la empresa para enriquecer nuestros análisis.
    \begin{itemize}
        \item Se podrán configurar conexiones a múltiples APIs externas.
        \item Los datos de las APIs se ingestarán correctamente en el sistema.
        \item La ingesta de APIs se realizará de forma programada o bajo demanda.
    \end{itemize}

    \item \textbf{HU2.5:} Como desarrollador de Okticket, quiero poder ingestar
    información de páginas web externas (scraping) para obtener datos
    adicionales relevantes.
    \begin{itemize}
        \item Se podrán configurar tareas de scraping para múltiples sitios web.
        \item Los datos obtenidos por scraping se almacenarán correctamente.
        \item El proceso de scraping respetará las políticas de los sitios web.
    \end{itemize}
\end{itemize}


\newpage{}
\subsection{Epic 3: Procesamiento y almacenamiento de datos}
Este epic se enfoca en la manipulación y organización eficiente de los datos
ingresados.

\begin{itemize}
    \item \textbf{HU3.1:} Como desarrollador de Okticket, quiero que los datos
    se limpien de manera automática para garantizar la calidad de la
    información.
    \begin{itemize}
        \item Se implementarán procesos de limpieza de datos para cada fuente.
        \item Los datos limpiados no contendrán valores nulos o incorrectos.
        \item Se mantendrá un registro de las transformaciones aplicadas.
    \end{itemize}

    \item \textbf{HU3.2:} Como desarrollador de Okticket, quiero que los datos
    contengan metadatos que faciliten su filtrado o búsqueda para mejorar la
    eficiencia en el análisis.
    \begin{itemize}
        \item Cada registro de datos incluirá metadatos relevantes.
        \item Los metadatos permitirán filtrar y buscar eficientemente.
        \item Se podrán realizar búsquedas complejas utilizando los metadatos.
    \end{itemize}
\end{itemize}


\newpage{}
\subsection{Epic 4: Visualización y análisis}
La visualización y análisis de datos es crucial para extraer valor de la
información recopilada.

\begin{itemize}
    \item \textbf{HU4.1:} Como trabajador de Okticket, quiero poder ver y
    consultar datos internos de la empresa para tomar decisiones informadas.
    \begin{itemize}
        \item Existirán paneles de control que mostrarán datos internos relevantes.
        \item Los paneles se actualizarán en tiempo real o con una frecuencia adecuada.
        \item Los usuarios podrán personalizar las visualizaciones según sus necesidades.
    \end{itemize}

    \item \textbf{HU4.2:} Como desarrollador de Okticket, quiero poder ver el
    estado general de la infraestructura para monitorear su salud y rendimiento.
    \begin{itemize}
        \item Existirá un panel que mostrará el estado de todos los servicios del sistema.
        \item Se visualizarán métricas clave como CPU, memoria y uso de red.
        \item El panel incluirá alertas visuales para problemas críticos.
    \end{itemize}

    \item \textbf{HU4.3:} Como trabajador de Okticket, quiero poder ver y
    consultar datos de empresas cliente para ofrecer un mejor servicio y
    soporte.
    \begin{itemize}
        \item Se podrán visualizar datos específicos de cada empresa cliente.
        \item La información se presentará de forma clara y organizada.
        \item Se podrán generar informes personalizados para cada cliente.
    \end{itemize}

    \item \textbf{HU4.4:} Como gestor de una empresa cliente, quiero poder ver
    información relevante sobre mi empresa que recoja Okticket para optimizar
    mis procesos y tomar decisiones estratégicas.
    \begin{itemize}
        \item Existirá un panel de control específico para cada empresa cliente.
        \item Los datos se presentarán de forma comprensible para usuarios no técnicos.
        \item Se incluirán métricas y KPIs relevantes para la gestión empresarial.
    \end{itemize}
\end{itemize}

\newpage{}
\section{Story Mapping}
El Story Mapping proporciona una visión de cómo las historias de usuario se
traducen en tareas concretas de desarrollo. Esta estrategia permite una
planificación más precisa y un seguimiento efectivo del progreso del proyecto.

\begin{figure}[h]
	\centering
	\includegraphics[width=\textwidth]{storymapping.png}
	\caption{Diagrama \textit{Story Mapping} del proyecto}
	\label{fig:story_mapping}
\end{figure}

El diagrama anterior muestra cómo las historias de usuario se organizan en
epics y se desglosan en tareas específicas. Esta representación visual ayuda a
comprender la estructura del proyecto y las dependencias entre las diferentes
historias y tareas.

Este \textit{story mapping} establece una hoja de ruta clara para el desarrollo
del proyecto, asegurando que todas las historias de usuario se aborden de manera
sistemática y eficiente.
