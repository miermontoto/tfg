\chapter{Diseño del sistema}\label{chap:diseño}
\section{Estudio de alternativas}\label{sec:estudio}
\subsection{Herramientas de despliegue de infraestructura}\label{subsec:herdesinf}
A la hora de desplegar la infraestructura de un proyecto, se consideran varias
herramientas populares que permiten automatizar este proceso. Entre ellas, se
encuentran \textit{Terraform}, \textit{AWS CloudFormation} y \textit{Ansible}.

A continuación, se describen brevemente estas herramientas y se comparan sus
características.

\subsubsection{Alternativas}
\paragraph{Terraform}
\textit{Terraform} es una herramienta de código abierto desarrollada por
\textit{HashiCorp} que permite definir y desplegar infraestructura de forma
declarativa. \textit{Terraform} permite definir la infraestructura en un archivo
de configuración JSON, que describe los recursos que se desean crear y sus
dependencias. A partir de este archivo, \textit{Terraform} se encarga de
desplegar los recursos en el proveedor de nube especificado, que en el caso de
este proyecto es \textit{AWS}.

\paragraph{AWS CloudFormation}
\textit{AWS CloudFormation} es un servicio de \textit{Amazon Web Services} similar
a \textit{Terraform} que permite definir y desplegar infraestructura en la nube
de forma declarativa. \textit{AWS CloudFormation} permite definir la infraestructura
o bien mediante un archivo de configuración (en formato JSON o YAML), o bien gráficamente
mediante diagramas.

\paragraph{Ansible}
\textit{Ansible} es una herramienta multi-propósito de automatización de tareas entre
las que se incluye el despliegue y orquestación de infraestructura. Se trata de una
herramienta desarrollada por \textit{Red Hat} que permite definir la infraestructura
mediante \textit{playbooks} escritos en YAML, que describen las tareas a realizar y
los servidores en los que se deben ejecutar.

\subsubsection{Comparación}
Comenzando la comparativa por la facilidad de uso de cada herramienta, \textit{Terraform}
es la alternativa planteada más fácil de usar, ya que permite definir la infraestructura
en un archivo con sintaxis sencilla y desplegarla con un solo comando. Por otro lado,
\textit{AWS CloudFormation} es un poco más complejo de usar, ya que requiere definir
la infraestructura en un archivo de configuración o en un diagrama, y luego desplegarla
mediante la consola de \textit{AWS}. \textit{Ansible} es más complejo de usar, ya qu
requiere definir la infraestructura mediante \textit{playbooks} y ejecutarlos en los
servidores, pero es más flexible y potente que las otras dos herramientas.

Mientras que \textit{Terraform} y \textit{Ansible} son herramientas \textit{multi-cloud},
lo que significa que funcionan con cualquier proveedor de nube, \textit{AWS CloudFormation}
es una herramienta específica de \textit{AWS} y solo funciona con sus servicios, lo que
puede suponer tanto una ventaja como una desventaja, dependiendo de las necesidades del
proyecto. Puesto que este proyecto se despliega en \textit{AWS} sin intención de migrar a
otra plataforma, pero a la vez no se requiere el uso de servicios específicos de \textit{AWS},
no se considera una ventaja significativa.

\subsubsection{Decisión}
Ninguna de las alternativas consideradas es claramente superior a las demás, ya que
se tratan de herramientas con características y funcionalidades similares y una
amplia acogida en la industria. Sin embargo, se ha decidido utilizar \textit{Terraform}
para el despliegue de la infraestructura de este proyecto, ya que es la herramienta
más ``sencilla'' de utilizar, la que mejor se podría adaptar a las necesidades del
proyecto y la única que ya se ha utilizado en proyectos anteriores dentro de la empresa.

\subsection{Ingesta de datos}\label{subsec:ingesta}
A partir del conjunto de tecnologías seleccionadas en la descripción detallada del proyecto,
se consdieran diversas tecnologías que permitan ingestar datos de todas las fuentes que
requieren ser procesadas, como \textit{Apache NiFi}, \textit{AWS Glue} y \textit{Kafka}.

\subsubsection{Alternativas}
\paragraph{Apache NiFi}

\paragraph{AWS Glue}

\paragraph{Kafka}


\subsubsection{Comparación}


\subsubsection{Decisión}

\newpage{}
\section{Arquitectura del sistema}\label{sec:arquitectura}
Tras toda la definción de los requisitos y la valoración de las alternativas disponibles,
se plantea la siguiente arquitectura:



\newpage{}
\section{Modelo de datos}\label{sec:modelo}
