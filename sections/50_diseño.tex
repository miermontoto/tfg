\chapter{Diseño del sistema}\label{chap:diseño}
\section{Estudio de alternativas}\label{sec:estudio}
En este apartado se explorarán las diferentes alternativas disponibles para el
diseño del sistema. Se analizarán las características, ventajas y desventajas de
cada opción, con el objetivo de proporcionar una visión clara y fundamentada que
permita seleccionar la alternativa más adecuada para el proyecto. Las áreas de
estudio incluirán tanto el despliegue de infraestructura como otros aspectos
críticos del diseño del sistema, asegurando una evaluación integral y detallada
de las posibles soluciones.


\subsection{Despliegue de infraestructura}\label{subsec:herdesinf}
A la hora de desplegar la infraestructura de un proyecto, se consideran varias
herramientas populares que permiten automatizar este proceso. Entre todas ellas,
las más establecidas y atractivas son \textit{Terraform},
\textit{AWS CloudFormation} y \textit{Ansible}.

A continuación, se describen brevemente estas herramientas y se comparan sus
características.

\subsubsection{Alternativas}
\paragraph{Terraform}
\textit{Terraform} es una herramienta de código abierto desarrollada por
\textit{HashiCorp} que permite definir y desplegar infraestructura de forma
declarativa. \textit{Terraform} permite definir la infraestructura en un archivo
de configuración JSON, que describe los recursos que se desean crear y sus
dependencias. A partir de este archivo, \textit{Terraform} se encarga de
desplegar los recursos en el proveedor de nube especificado, que en el caso de
este proyecto es \textit{AWS}.

\begin{figure}[H]
	\centering
	\includegraphics[width=0.2\textwidth]{logos/terraform.png}
	\caption{Logo de Terraform~\textregistered}
	\label{fig:terraform}
\end{figure}

\paragraph{AWS CloudFormation}
\textit{AWS CloudFormation} es un servicio de \textit{Amazon Web Services}
similar a \textit{Terraform} que permite definir y desplegar infraestructura en
la nube de forma declarativa. \textit{AWS CloudFormation} permite definir la
infraestructura o bien mediante un archivo de configuración (en formato JSON o
YAML), o bien gráficamente mediante diagramas, un punto muy fuerte a favor de
esta alternativa.

\begin{figure}[H]
	\centering
	\includegraphics[width=0.2\textwidth]{logos/cloudformation.png}
	\caption{Logo de AWS CloudFormation~\textregistered}
	\label{fig:cloudformation}
\end{figure}

\paragraph{Ansible}
\textit{Ansible} es una herramienta multi-propósito de automatización de tareas
entre las que se incluye el despliegue y orquestación de infraestructura. Se
trata de una herramienta desarrollada por \textit{Red Hat} que permite definir
la infraestructura mediante \textit{playbooks} escritos en YAML, que describen
las tareas a realizar y los servidores en los que se deben ejecutar.

\begin{figure}[H]
	\centering
	\includegraphics[width=0.2\textwidth]{logos/ansible.png}
	\caption{Logo de Ansible~\textregistered}
	\label{fig:ansible}
\end{figure}

\subsubsection{Comparación}
Comenzando la comparativa por la facilidad de uso de cada herramienta,
\textit{Terraform} es la alternativa planteada más fácil de usar, ya que permite
definir la infraestructura en un archivo con sintaxis sencilla y desplegarla con
un solo comando. Por otro lado, \textit{AWS CloudFormation} es un poco más
complejo de usar, ya que requiere definir la infraestructura en un archivo de
configuración o en un diagrama, y luego desplegarla mediante la consola de
\textit{AWS}. \textit{Ansible} es más complejo de usar, ya que requiere definir
la infraestructura mediante \textit{playbooks} y ejecutarlos en los servidores,
pero es más flexible y potente que las otras dos herramientas.

Mientras que \textit{Terraform} y \textit{Ansible} son herramientas
\textit{multi-cloud}, lo que significa que funcionan con cualquier proveedor de
nube, \textit{AWS CloudFormation} es una herramienta específica de \textit{AWS}
y solo funciona con sus servicios, lo que puede suponer tanto una ventaja como
una desventaja, dependiendo de las necesidades del proyecto. En este caso, la
solución se va a desplegar en la nube de \textit{Amazon}, pero a la vez no se
requiere el uso de servicios específicos de \textit{AWS}, no se considera una
ventaja significativa.

\subsubsection{Decisión}
Ninguna de las alternativas consideradas es claramente superior a las demás, ya
que se tratan de herramientas con características y funcionalidades similares y
una gran popularidad en la industria. Sin embargo, se decide utilizar
\textit{Terraform} para el despliegue de la infraestructura de este proyecto,
ya que es la herramienta más ``sencilla'', la que mejor se podría adaptar a las
necesidades del proyecto y la única que ya se ha usado en proyectos anteriores
dentro de la empresa.

\subsection{Ingesta de datos}\label{subsec:ingesta}
A partir del conjunto de tecnologías seleccionadas en la descripción detallada
del proyecto, se consdieran diversas tecnologías, como \textit{Redpanda},
\textit{AWS Glue} y \textit{Kafka} que permitan ingestar datos de todas las
fuentes que requieren ser procesadas.

\subsubsection{Alternativas}
\paragraph{Kafka}
\textit{Kafka} es una plataforma de transmisión de datos distribuida y de código
abierto que se utiliza para construir pipelines de datos en tiempo real y
aplicaciones de streaming. Desarrollada originalmente por LinkedIn y
posteriormente donada a la Apache Software Foundation, \textit{Kafka} se ha
convertido en una de las tecnologías más populares para la gestión de flujos de
datos en tiempo real.

\begin{figure}[H]
	\centering
	\includegraphics[width=0.2\textwidth]{logos/kafka.png}
	\caption{Logo de Kafka~\textregistered}
	\label{fig:kafka}
\end{figure}

Una de las principales ventajas de \textit{Kafka} es su capacidad para manejar
grandes volúmenes de datos con alta eficiencia y baja latencia. \textit{Kafka}
utiliza un modelo de publicación-suscripción, donde los productores publican
mensajes en temas y los consumidores se suscriben a estos temas para recibir
los mensajes. Esta arquitectura permite una alta escalabilidad y flexibilidad
en la gestión de datos.

\textit{Kafka} se compone de varios componentes clave:

\begin{itemize}
    \item \textbf{Tópico}: Un tópico es una categoría a la que se envían los
		mensajes y a la que los consumidores están \textit{suscritos}. Los
		consumidores pueden estar suscritos a uno o varios tópicos, y los
		productores pueden enviar mensajes a uno o varios tópicos. Los tópicos
		son la unidad básica de organización de los mensajes en cualquier
		sistema de mensajería de publicación/suscripción.
    \item \textbf{Productor}: El productor es el componente responsable de crear
		y enviar mensajes al cluster de Kafka. Está separado del resto de los
		componentes y produce mensajes de manera asíncrona y rápida.
    \item \textbf{Consumidor}: El consumidor es el componente responsable de
		leer los mensajes producidos por el productor. Está suscrito a un tópico
		a través del broker y consume los mensajes.
    \item \textbf{Broker}: El broker es el componente responsable de recibir los
		mensajes producidos por el productor y enviarlos a los consumidores. Es
		el intermediario entre los productores y los consumidores.
    \item \textbf{Zookeeper}: Zookeeper es un servicio separado de coordinación
		distribuida que se utiliza para gestionar y coordinar los brokers de
		Kafka. Se encarga de mantener la información de los brokers y de los
		tópicos. Actualmente, este servicio es una dependencia obligatoria de
		Kafka.~\footnote{
			Dejará de ser necesario en la versión 4.
			\url{https://x.com/coltmcnealy/status/1801987159534264641}
		}
\end{itemize}

A pesar de sus numerosas ventajas, \textit{Kafka} también presenta algunos
desafíos. La configuración y gestión de un clúster de \textit{Kafka} puede ser
compleja, especialmente en entornos de producción a gran escala. Además,
\textit{Kafka} depende de \textit{Zookeeper} para la coordinación, lo que añade
una capa adicional de complejidad en la administración del sistema.

En resumen, \textit{Kafka} es una solución robusta y escalable para la
transmisión de datos en tiempo real, ideal para aplicaciones que requieren alta
disponibilidad y procesamiento eficiente de grandes volúmenes de datos. Sin
embargo, su implementación y gestión requieren un conocimiento profundo de su
arquitectura y componentes.


\paragraph{Redpanda}
\textit{Redpanda} es una plataforma de transmisión de datos en tiempo real que
se destaca por su alto rendimiento y baja latencia. Diseñada como una
alternativa moderna a \textit{Kafka}, \textit{Redpanda} ofrece una arquitectura
simplificada que elimina la necesidad de dependencias externas como
\textit{Zookeeper}. Esto no solo reduce la complejidad operativa, sino que
también mejora la eficiencia y la escalabilidad del sistema. \textit{Redpanda}
es compatible con la API de \textit{Kafka}, lo que facilita la migración de
aplicaciones existentes sin necesidad de cambios significativos en el código.
Además, su diseño optimizado para hardware moderno permite un procesamiento más
rápido y un uso más eficiente de los recursos, lo que la convierte en una opción
ideal para aplicaciones que requieren una transmisión de datos rápida y
confiable.

Sin embargo, \textit{Redpanda} también presenta algunos puntos en contra. Al ser
una tecnología relativamente nueva, su ecosistema y comunidad de usuarios no son
tan amplios como los de \textit{Kafka}, lo que puede limitar el acceso a
recursos y soporte. Además, aunque la compatibilidad con la API de
\textit{Kafka} es una ventaja, puede haber ciertas características y extensiones
específicas de \textit{Kafka} que no estén completamente soportadas en
\textit{Redpanda}. Finalmente, la adopción de una nueva tecnología siempre
conlleva riesgos asociados con la estabilidad y el soporte a largo plazo,
aspectos que deben ser considerados cuidadosamente antes de su implementación.


\paragraph{AWS Glue}
\textit{AWS Glue} es un servicio de integración de datos totalmente administrado
que facilita la preparación y carga de datos para análisis. Diseñado para
trabajar con grandes volúmenes de datos, \textit{AWS Glue} automatiza las tareas
de descubrimiento, catalogación, limpieza, enriquecimiento y movimiento de datos
entre diferentes almacenes de datos. Una de las principales ventajas de
\textit{AWS Glue} es su capacidad para generar automáticamente el código
necesario para realizar las transformaciones de datos, lo que reduce
significativamente el tiempo y el esfuerzo requeridos para preparar los datos
para el análisis. Además, \textit{AWS Glue} es altamente escalable y puede
manejar tanto cargas de trabajo por lotes como en tiempo real, lo que lo
convierte en una opción versátil para diversas necesidades de integración de
datos.

\textit{AWS Glue} es un servicio administrado, por lo que su uso puede implicar
costes adicionales en comparación con soluciones autogestionadas. Además, aunque
\textit{AWS Glue} ofrece una gran flexibilidad y potencia, su configuración y
optimización pueden requerir un conocimiento profundo de los servicios de
\textit{AWS} y de las mejores prácticas de integración de datos. Por último, la
dependencia de \textit{AWS Glue} puede limitar la portabilidad de las soluciones
de integración de datos a otros proveedores de nube, lo que podría ser un
inconveniente en caso de querer migrar el proyecto a otro proveedor.


\subsubsection{Comparación y decisión}
Desde el primer momento, en la empresa se considera Kafka como la opción más
sólida junto con el \textit{stack ELK} para desarrollar el proyecto, al
tratarse de un estándar en la industria y una solución tanto rápida y escalable
como asequible a nivel económico. Por eso, y pese a que las otras alternativas
son atractivas para el desarrollo de este proyecto, se decide utilizar Kafka
como servicio de ingesta de datos, en consonancia con \textit{Logstash}.


\newpage{}
\section{Arquitectura del sistema}\label{sec:arquitectura}
Tras toda la definción de los requisitos y la valoración de las alternativas
disponibles, en este apartado se plantea la arquitectura completa del sistema.



\newpage{}
\section{Modelo de datos}\label{sec:modelo}
