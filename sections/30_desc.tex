\chapter{Descripción general del proyecto}
Esta sección describe el proyecto en términos generales, incluyendo una descripción de los
problemas que se pretenden resolver, las partes interesadas en el proyecto y una valoración de
las alternativas consideradas.


\section{Partes interesadas (stakeholders)}\label{sec:stakeholders}
Las partes interesadas en el proyecto son aquellas personas o entidades que tienen un interés
en el mismo, ya sea porque se ven afectadas por el resultado del proyecto, o porque tienen
algún tipo de interés en el mismo. Las partes interesadas en este proyecto son las siguientes:

\begin{enumerate}
	\item \textbf{Okticket}: la empresa es la principal parte interesada en el proyecto, ya que
		es la que se beneficiará directamente de los resultados del mismo, así como de las
		oportunidades de negocio que se abren con la explotación de los datos. Dentro de la empresa,
		se pueden identificar dos entidades:
		\begin{itemize}
			\item \textbf{Equipo de desarrollo de la empresa}: el equipo de desarrollo es otra parte
				interesada en el proyecto, ya que son los encargados de llevar a cabo la implementación
				del sistema y de garantizar su correcto funcionamiento, además de gestionar el soporte
				de servicio a nivel técnico.
			\item \textbf{Equipo de soporte de la empresa}: el sistema planteado ahorraría tiempo al equipo de
				soporte, ya que les permitiría analizar los datos de forma más eficiente e identificar
				problemas antes de que tener que resolver las peticiones de los clientes afectados a
				nivel básico.
		\end{itemize}
	\item \textbf{Clientes}: los clientes de la empresa también son partes interesadas, puesto
		que se beneficiarán de los nuevos servicios que se ofrecen, como los dashboards de
		negocio que se han descrito anteriormente. Estos clientes no son necesariamente los
		usuarios finales, sino los administradores y gestores de las empresas que utilizan
		Okticket como herramienta de gestión de gastos.
	\item \textbf{Investigador y desarrollador (\emph{\author}):} el desarrollador del
		proyecto tiene la oportunidad de aplicar los conocimientos adquiridos en el desarrollo de un
		proyecto real, y de adquirir nuevos conocimientos en el proceso.
\end{enumerate}

\section{Valoración de alternativas}\label{sec:alternativas}
\subsection{Criterios de evaluación}\label{subsec:criterios}

\subsection{Alternativas consideradas}\label{subsec:alternativas}

\subsection{Resultados}\label{subsec:resultados}

\newpage{}
\section{Descripción del proyecto}\label{sec:descripcion}
