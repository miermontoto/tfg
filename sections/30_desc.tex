\chapter{Descripción general del proyecto}\label{chap:desc}
Esta sección describe el proyecto en términos generales, incluyendo una
descripción de los problemas que se pretenden resolver, las partes interesadas
en el proyecto y una valoración de las alternativas consideradas.


\section{Partes interesadas (stakeholders)}\label{sec:stakeholders}
Las partes interesadas en el proyecto son aquellas personas o entidades que
tienen un interés en el mismo, ya sea porque se ven afectadas por el resultado
del proyecto, o porque tienen algún tipo de interés en el mismo. Las partes
interesadas en este proyecto son las siguientes:

\begin{enumerate}
	\item \textbf{Okticket}: la empresa es la principal parte interesada en el
		proyecto, ya que es la que se beneficiará directamente de los resultados
		del mismo, así como de las oportunidades de negocio que se abren con la
		explotación de los datos. Dentro de la empresa, se pueden identificar
		dos entidades:
		\begin{itemize}
			\item \textbf{Equipo de desarrollo de la empresa}: el equipo de
				desarrollo es otra parte interesada en el proyecto, ya que son
				los encargados de llevar a cabo la implementación del sistema y
				de garantizar su correcto funcionamiento, además de gestionar el
				soporte de servicio a nivel técnico.
			\item \textbf{Equipo de soporte de la empresa}: el sistema planteado
				ahorraría tiempo al equipo de soporte, ya que les permitiría
				analizar los datos de forma más eficiente e identificar
				problemas antes de que tener que resolver las peticiones de los
				clientes afectados a nivel básico.
		\end{itemize}
	\item \textbf{Clientes}: los clientes de la empresa también son partes
		interesadas, puesto que se beneficiarán de los nuevos servicios que se
		ofrecen, como los dashboards de negocio que se han descrito
		anteriormente. Estos clientes no son necesariamente los usuarios
		finales, sino los administradores y gestores de las empresas que
		utilizan Okticket como herramienta de gestión de gastos.
	\item \textbf{Investigador y desarrollador (\emph{\author}):} el
		desarrollador del proyecto tiene la oportunidad de aplicar los
		conocimientos adquiridos en el desarrollo de un proyecto real, y de
		adquirir nuevos conocimientos en el proceso.
\end{enumerate}

\section{Alternativas existentes}\label{sec:alternativas}
Antes de comenzar la planificación y el desarrollo del proyecto, se consideran
varias opciones ya existentes en el mercado que podrían resolver o adaptarse a
las necesidades planteadas.

Los puntos claves a considerar de cada alternativa son los siguientes:
\begin{itemize}
	\item \textbf{Coste:} el coste es un factor clave para evaluar las
		alternativas, prefiriendo aquellas que sean gratuitas o cuya
		implementación requiera la mínima inversión posible.
	\item \textbf{Complejidad:} al tratarse de un proyecto con poco conocimiento
		previo y sin mucha urgencia, se prefiere una solución que sea sencilla y
		que no requiera demasiado desarrollo.
	\item \textbf{Rendimiento y escalabilidad:} obviamente, a la hora de manejar
		los volúmenes de datos con los que se están tratando, el sistema debe
		responder positivamente tanto a la ingesta, tratamiento y visualización
		como a la escalabilidad del mismo.
\end{itemize}

Como punto añadido, se valoran de manera positiva las alternativas
\textit{open-source}, ya que son más flexibles y permiten personalizar el
sistema a los necesidades del proyecto.

A continuación, se describen brevemente estas alternativas y se comparan sus
características.

\subsection{Loggly, Splunk}
Ambas herramientas se centran en la gestión y análisis de logs basada en la nube,
permitiendo centralizar, monitorizar y analizar datos de logs en tiempo real.
Diseñadas para simplificar la gestión de logs, las dos alternativas ofrecen una
interfaz intuitiva y potentes capacidades de búsqueda y visualización que
facilitan la identificación y resolución de problemas en los sistemas y
aplicaciones.

Una de las principales ventajas de este tipo de herramientas es su capacidad
para integrarse con una amplia variedad de servicios y plataformas, lo que
permite a las empresas consolidar sus datos de logs en un único lugar. Además,
suelen ofrecer algún sistema de alertas en tiempo real y paneles de control
personalizables, lo que encaja muy bien con algunos de los requisitos de este
proyecto.

Sin embargo, también presentan algunas desventajas que entran en conflicto con
los intereses descritos anteriormente. En primer lugar, al tratarse de
servicios enfocados en el tratamiento exclusivo de logs, podrían no acomodar
algunos de los requisitos esenciales, como la ingesta de las bases de datos
propias de la empresa.

Otra posible limitación es su sistema de precios; estas herramientas cuentan con
una jerarquía de suscripciones que limita mucho su escalabilidad y aumentaría
rápidamente los costes de su uso en caso de llegar a depender de ellas.

En resumen, aunque tanto Loggly como Splunk como cualquier otro SaaS por el
son soluciones robustas y eficientes para la gestión de logs, sus posibles
costes adicionales, junto con sus limitaciones en el análisis de datos de
inteligencia de negocio, las convierten en opciones menos atractivas para el
desarollo de este proyecto.

\subsection{Loki}
Loki es una herramienta de gestión de logs desarrollada por Grafana Labs,
diseñada específicamente para ser altamente eficiente y escalable. A diferencia
de otras soluciones de gestión de datos, Loki no indexa el contenido completo de
estos, sino que se centra en los metadatos, lo que reduce significativamente
los requisitos de almacenamiento y mejora el rendimiento.

\subsubsection{Ventajas}
\begin{itemize}
    \item \textbf{Integración con Grafana:} Una de las principales ventajas de
		Loki es su integración nativa con Grafana, una popular plataforma de
			visualización de datos. Esto permite a los usuarios crear paneles de
			control y alertas basados en los datos de logs de manera sencilla y
			eficiente.
    \item \textbf{Eficiencia en el almacenamiento:} Al no indexar el contenido
		completo de los logs, Loki reduce significativamente los requisitos de
		almacenamiento. Esto lo hace una opción más económica y eficiente en
		comparación con otras soluciones.
    \item \textbf{Escalabilidad:} Loki está diseñado para ser altamente
		escalable, lo que permite manejar grandes volúmenes de datos de logs sin
		comprometer el rendimiento.
    \item \textbf{Simplicidad en la configuración:} La configuración de Loki es
		relativamente sencilla, especialmente para aquellos que ya están
		familiarizados con Grafana y Prometheus. Esto facilita su adopción y
		despliegue en entornos de producción.
\end{itemize}

\subsubsection{Desventajas}
\begin{itemize}
    \item \textbf{Limitaciones en la búsqueda:} Al no indexar el contenido
		completo de los logs, las capacidades de búsqueda de Loki son más
		limitadas en comparación con otras herramientas como Elasticsearch.
		Esto puede ser una desventaja en caso de necesitar realizar búsquedas
		complejas y detalladas.
    \item \textbf{Ecosistema en desarrollo:} Aunque Loki ha ganado popularidad
		rápidamente, su ecosistema y comunidad de usuarios aún están en
		desarrollo. Esto puede limitar el acceso a recursos y soporte en
		comparación con soluciones más maduras.
    \item \textbf{Dependencia de Grafana:} Si bien la integración con Grafana es
		una ventaja, es un factor limitante para aquellos que no utilicen
		Grafana, como es el caso de Okticket, al tener que adaptarse a un
		ecosistema diferente.
\end{itemize}

En resumen, Loki es una solución eficiente y escalable para la gestión de datos,
especialmente adecuada para aquellos que ya utilizan Grafana. Sin embargo, sus
limitaciones en la búsqueda y su ecosistema en desarrollo son factores a
considerar antes de su implementación.


\subsection{Graylog + Prometheus}
Esta combinación de herramientas es una alternativa interesante para la gestión
de logs y métricas en entornos de producción. Graylog es una plataforma de
gestión de logs de código abierto que permite centralizar, monitorizar y
analizar logs de diferentes fuentes. Por otro lado, Prometheus es un sistema de
monitorización y alertas de código abierto que se centra en la recopilación de
métricas de sistemas y aplicaciones.

En este documento se tratan de manera combinada porque, aunque son herramientas
independientes, su integración es una solución completa que se está volviendo
popular en el sector. Graylog se encarga de la gestión de logs, mientras que
Prometheus se encarga de la recopilación de métricas y alertas.

La combinación de Graylog y Prometheus ofrece varios beneficios, como la
centralización de logs y métricas que facilita un análisis más integral y
eficiente de los datos. Además, la mejora en la monitorización a través de las
capacidades avanzadas de Prometheus, integradas con Graylog, mejora la detección
de problemas y la respuesta a incidentes en tiempo real. Esta integración
también proporciona flexibilidad y escalabilidad, adaptándose a las necesidades
específicas de cada entorno de producción. Sin embargo, existen desventajas como
la complejidad de configuración al integrar dos sistemas independientes, lo que
puede resultar en un mayor esfuerzo en mantenimiento y gestión. Además, al
tratarse de herramientas novedosas, supondría una mayor curva de aprendizaje
para el equipo de desarrollo a la hora de tratar con las herramientas.


\newpage{}
\section{Descripción del proyecto}\label{sec:descripcion}
