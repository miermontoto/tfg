\chapter{Descripción general del proyecto}
Esta sección describe el proyecto en términos generales, incluyendo una
descripción de los problemas que se pretenden resolver, las partes interesadas
en el proyecto y una valoración de las alternativas consideradas.


\section{Partes interesadas (stakeholders)}\label{sec:stakeholders}
Las partes interesadas en el proyecto son aquellas personas o entidades que
tienen un interés en el mismo, ya sea porque se ven afectadas por el resultado
del proyecto, o porque tienen algún tipo de interés en el mismo. Las partes
interesadas en este proyecto son las siguientes:

\begin{enumerate}
	\item \textbf{Okticket}: la empresa es la principal parte interesada en el
		proyecto, ya que es la que se beneficiará directamente de los resultados
		del mismo, así como de las oportunidades de negocio que se abren con la
		explotación de los datos. Dentro de la empresa, se pueden identificar
		dos entidades:
		\begin{itemize}
			\item \textbf{Equipo de desarrollo de la empresa}: el equipo de
				desarrollo es otra parte interesada en el proyecto, ya que son
				los encargados de llevar a cabo la implementación del sistema y
				de garantizar su correcto funcionamiento, además de gestionar el
				soporte de servicio a nivel técnico.
			\item \textbf{Equipo de soporte de la empresa}: el sistema planteado
				ahorraría tiempo al equipo de soporte, ya que les permitiría
				analizar los datos de forma más eficiente e identificar
				problemas antes de que tener que resolver las peticiones de los
				clientes afectados a nivel básico.
		\end{itemize}
	\item \textbf{Clientes}: los clientes de la empresa también son partes
		interesadas, puesto que se beneficiarán de los nuevos servicios que se
		ofrecen, como los dashboards de negocio que se han descrito
		anteriormente. Estos clientes no son necesariamente los usuarios
		finales, sino los administradores y gestores de las empresas que
		utilizan Okticket como herramienta de gestión de gastos.
	\item \textbf{Investigador y desarrollador (\emph{\author}):} el
		desarrollador del proyecto tiene la oportunidad de aplicar los
		conocimientos adquiridos en el desarrollo de un proyecto real, y de
		adquirir nuevos conocimientos en el proceso.
\end{enumerate}

\section{Alternativas existentes}\label{sec:alternativas}
Antes de comenzar la planificación y el desarrollo del proyecto, se consideran
varias opciones ya existentes en el mercado que podrían resolver o adaptarse a
las necesidades planteadas.

Los puntos claves a considerar de cada alternativa son los siguientes:
\begin{itemize}
	\item \textbf{Coste:} el coste es un factor clave para evaluar las
		alternativas, prefiriendo aquellas que sean gratuitas o cuya
		implementación requiera la mínima inversión posible.
	\item \textbf{Complejidad:} al tratarse de un proyecto con poco conocimiento
		previo y sin mucha urgencia, se prefiere una solución que sea sencilla y
		que no requiera demasiado desarrollo.
	\item \textbf{Rendimiento y escalabilidad:} obviamente, a la hora de manejar
		los volúmenes de datos con los que se están tratando, el sistema debe
		responder positivamente tanto a la ingesta, tratamiento y visualización
		como a la escalabilidad del mismo.
\end{itemize}

Como punto añadido, se valoran de manera positiva las alternativas
\textit{open-source}, ya que son más flexibles y permiten personalizar el
sistema a los necesidades del proyecto.

A continuación, se describen brevemente estas alternativas y se comparan sus
características.

\subsection{Splunk}
Splunk es un producto de \textit{big data} que permite la recolección, el
almacenamiento y la análisis de datos en tiempo real. Pese a que se centra en la
ingesta de ``macrodatos'', no está enfocada a la ingesta de datos de
inteligencia de negocio, sino a la ingesta de datos de \textit{logs} y
\textit{eventos} en tiempo real con el objetivo de mejorar la respuesta frente a
ataques y caídas de disponibilidad. Además, aunque hay versiones gratuitas, está
enfocado principalmente a la gestión como \textit{SaaS} por precios muy
elevados.



\newpage{}
\section{Descripción del proyecto}\label{sec:descripcion}
