\section{Presupuesto}\label{sec:presupuesto}
Para poder llevar a cabo este proyecto, se realiza una estimación del coste
total neceario para su desarrollo, que se divide en dos partes: el coste del
material, que incluye el coste de los recursos necesarios para el desarrollo del
proyecto, y el coste del personal, que incluye el coste de las horas de trabajo
del desarrollador.

Se estima un horizonte de desarrollo de 3 meses, que es el tiempo estimado y
disponible para el desarrollo del proyecto.


\subsection{Presupuesto de material}\label{subsec:pres_material}
Puesto que el proyecto se desarrolla en la empresa, se dispone de todos los
recursos físicos necesarios para llevar a cabo el proyecto, es decir, que no se
incluirá el coste del ordenador o de la conexión a internet en el presupuesto.

Sin embargo, se incluirá el coste de las herramientas y servicios utilizados
durante el desarrollo del proyecto, como el coste de las licencias de software,
el coste de los servicios en la nube, el coste de las herramientas de
desarrollo, etc.

Es importante destacar de que los precios de los servicios en la nube son
aproximados y pueden variar en función de la región, el tipo de instancia, el
tipo de almacenamiento, etc. Por lo tanto, los precios presentados en este
presupuesto son orientativos y pueden variar en función de las necesidades del
proyecto. En este caso, se analizan los precios a junio de 2024 en la región
de Amazon Web Services (AWS) de \texttt{eu-north-1} (Estocolmo).

\begin{table}[H]
	\centering
	\small
	\begin{tabular}{|l|l|r|r|r|}
	\hline
	\textbf{Categoría} & \textbf{Ítem} & \textbf{Cantidad} & \textbf{Coste unitario} & \textbf{Coste total} \\
	\hline
	\hline
	AWS Compute & Fargate & 7168 vCPU-h/mes & 0,04928€/vCPU-h & 353,24€/mes \\
	 & & 18432 GB-h/mes & 0,00539€/GB-h & 99,35€/mes \\
	\hline
	AWS Storage & EFS & 512 GB/mes & 0,38€/GB-mes & 194,56€/mes \\
	 & S3 & 256 GB/mes & 0,0255€/GB-mes & 6,53€/mes \\
	\hline
	AWS Network & VPC & 4 NAT Gateways & 0,054€/h & 155,52€/mes \\
	 & ELB & 3 ALBs & 0,012€/h & 25,92€/mes \\
	 & & 1 NLB & 0,027€/h & 19,44€/mes \\
	 & Route 53 & 3 registros & 0,50€/registro-mes & 1,50€/mes \\
	\hline
	AWS Security & IAM & - & Sin cargo & 0,00€ \\
	 & KMS & 1 CMK & 1,00€/mes & 1,00€/mes \\
	 & Secret Manager & 12 secretos & 0,40€/secreto-mes & 4,80€/mes \\
	\hline
	Stack KELK & Kafka & - & Licencia gratuita & 0,00€ \\
	 & Elasticsearch & - & Licencia gratuita & 0,00€ \\
	 & Logstash & - & Licencia gratuita & 0,00€ \\
	 & Kibana & - & Licencia gratuita & 0,00€ \\
	\hline
	AWS Container & ECS & - & Sin cargo & 0,00€ \\
	\hline
	AWS Monitor & CloudWatch & 5 métricas & 0,30€/métrica-mes & 1,50€/mes \\
	 & X-Ray & 50,000 trazas/mes & 4,60€/1M trazas & 0,23€/mes \\
	\hline
	Despliegue & Terraform & - & Licencia gratuita & 0,00€ \\
	\hline
	\textbf{Subtotal} & \multicolumn{4}{r|}{863,59€/mes} \\
	\hline
	\hline
	Otros & Support & Plan Basic & Sin cargo & 0,00€ \\
	 & Optimización & - & 5\% del subtotal & 43,18€ \\
	 & Contingencia & - & 10\% del subtotal & 86,36€ \\
	\hline
	\textbf{Total} & \multicolumn{4}{r|}{993,13€/mes} \\
	\hline
	\end{tabular}
	\caption{Propuesta de presupuesto mensual de materiales (región eu-north-1)}
	\label{tab:presupuesto_material}
\end{table}

Asumiendo que el proyecto se desarrolla en la región de AWS de Estocolmo
(\texttt{eu-north-1}), el coste total del material asciende a 993,13€ (novecientos
noventa y tres euros con trece céntimos) al mes, que incluye
el coste de los servicios en la nube, el coste de las licencias de software y el
coste de las herramientas de desarrollo.

Suponiendo un horizonte de desarrollo de 3 meses, el coste total del material
durante el desarrollo del proyecto asciende a 2.979,39€ (dos mil novecientos
setenta y nueve euros con treinta y nueve céntimos).


\newpage{}
\subsection{Presupuesto de personal}\label{subsec:pres_personal}
A continuación, se presenta una propuesta de presupuesto de personal para el
desarrollo del proyecto, que incluye el coste de las horas de trabajo según
cada rol y el coste total del personal.

\begin{table}[H]
	\centering
	\small
	\begin{tabular}{|l|l|r|r|r|}
	\hline
	\textbf{Rol} & \textbf{Descripción} & \textbf{Horas/mes} & \textbf{CU (€/h)} & \textbf{Coste total} \\
	\hline
	\hline
	Arquitecto & Diseño de la arquitectura y supervisión & 40 & 60 & 2.400,00€/mes \\
	\hline
	Desarrollador & Desarrollo y mantenimiento & 160 & 45 & 7.200,00€/mes \\
	\hline
	Administrador & Gestión de sistemas y seguridad & 160 & 50 & 8.000,00€/mes \\
	\hline
	DevOps & Infraestructuras y monitorización & 80 & 55 & 4.400,00€/mes \\
	\hline
	\textbf{Subtotal} & \multicolumn{4}{r|}{22.000,00€/mes} \\
	\hline
	\hline
	Otros & \multicolumn{3}{|l|}{IVA (21\%)} & 4.620,00€/mes \\
	 & \multicolumn{3}{|l|}{Margen (5\%)} & 1.100,00€/mes \\
	\hline
	\textbf{Total} & \multicolumn{4}{r|}{27.720,00€/mes} \\
	\hline
	\end{tabular}
	\caption{Propuesta de presupuesto de personal}
	\label{tab:presupuesto_personal_aws}
\end{table}

El coste del personal es ficticio, pero se ha calculado en base a experiencias
previas de contratación y subcontratación de personal en la empresa, además de
tener en cuenta el coste medio de los roles en Asturias.

El coste total del personal asciende a 27.720,00€ (veintisiete mil setecientos
veinte euros) al mes, que incluye el coste de las horas de trabajo de cada rol,
el IVA y el margen de beneficio industrial.

Asumiendo un horizonte de desarrollo de 3 meses, el coste total del personal
durante el desarrollo del proyecto asciende a 83.160,00€ (ochenta y tres mil
ciento sesenta euros).


\newpage{}
\subsection{Presupuesto total}\label{subsec:pres_total}
Finalmente, se presenta el presupuesto total del proyecto, que incluye el coste
del material y el coste del personal, así como el coste total del proyecto,
durante el horizonte de desarrollo establecido de 3 meses.

\begin{table}[H]
	\centering
	\small
	\begin{tabular}{|l|r|}
	\hline
	\textbf{Concepto} & \textbf{Coste} \\
	\hline
	Presupuesto de materiales & 2.979,39€ \\
	\hline
	Presupuesto de personal & 83.160,00€ \\
	\hline
	\textbf{Subtotal} & \textbf{86.139,39€} \\
	\hline
	\hline
	Beneficio industrial (15\%) & 12.920,91€ \\
	\hline
	\textbf{Total} & \textbf{99.060,30€} \\
	\hline
	\end{tabular}
	\caption{Costes combinados de presupuesto y materiales con beneficio industrial}
	\label{tab:costes_combinados}
\end{table}

El presupuesto total del proyecto asciende a 99.060,30€ (noventa y nueve mil
sesenta euros con treinta céntimos), que incluye el coste del material, el coste
del personal y el margen de beneficio industrial.
