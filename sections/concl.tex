\chapter{Resultados y trabajo futuro}
El propósito de este capítulo es presentar las conclusiones obtenidas a partir
del desarrollo del proyecto, recopilar las dificultades encontradas y proponer
líneas de trabajo futuro en vista a la amplicación y mejora del sistema.

\section{Resultados}


\newpage{}
\section{Trabajo futuro}

\begin{table}[H]
	\centering
	\begin{tabular}{|p{0.7\linewidth}|c|c|}
		\hline
		\textbf{Nombre} & \textbf{Prioridad} & \textbf{Tamaño} \\
		\hline
		\hline
		Como desarrollador de Okticket, quiero que los datos contengan metadatos que faciliten su filtrado o búsqueda & P2\cellcolor{yellow!50} & S\cellcolor{green!25} \\
		\hline
		Como trabajador de Okticket, quiero poder ver y consultar datos de empresas cliente & P2\cellcolor{yellow!50} & M\cellcolor{yellow!50} \\
		\hline
		Como gestor de una empresa cliente, quiero poder ver información relevante sobre mi empresa que recoja Okticket & P2\cellcolor{yellow!50} & L\cellcolor{orange!50} \\
		\hline
		Como desarrollador de Okticket, quiero poder ingestar datos de APIs externas a la empresa & P2\cellcolor{yellow!50} & L\cellcolor{orange!50} \\
		\hline
		Como desarrollador de Okticket, quiero poder ingestar información de páginas web externas (\textit{scraping}) & P2\cellcolor{yellow!50} & XL\cellcolor{red!50} \\
		\hline
	\end{tabular}
	\caption{Historias de usuario restantes para futuras iteraciones}
	\label{tab:remaining_tasks}
\end{table}


\newpage{}
\section{Conclusiones y retrospectiva}
