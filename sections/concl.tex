\chapter{Resultados y trabajo futuro}
El propósito de este capítulo es presentar las conclusiones obtenidas a partir
del desarrollo del proyecto, recopilar las dificultades encontradas y proponer
líneas de trabajo futuro en vista a la amplicación y mejora del sistema.

\section{Resultados}
El resultado del proyecto es un sistema de monitorización y análisis de datos
funcional y escalable, que permite a los usuarios obtener insights valiosos a
partir de la información recopilada.

Se ha logrado implementar una arquitectura robusta y flexible, basada en
tecnologías modernas y en la nube, que facilita la ingesta, procesamiento y
visualización de datos de manera eficiente y sencilla.

El sistema es capaz de ingestar datos de diversas fuentes, como bases de datos
internas, logs de aplicaciones y servicios externos, y de presentarlos de forma
clara y útil a través de Kibana.

La infraestructura se despliega y orquesta de manera automática en la nube,
permitiendo una gestión sencilla y eficiente del sistema para los
administradores.

Pese a que no se han completado todas las historias de usuario planificadas,
se han logrado los objetivos principales del proyecto, sentando las bases para
futuras iteraciones y logrando así entregar un producto mínimo viable (MVP).

El motivo de no haber completado todas las historias de usuario planificadas
se debe al tamaño y naturaleza del proyecto, que ha resultado ser más complejo
de lo esperado en un principio. Sin embargo, se ha logrado superar los objetivos
principales y se ha entregado un producto funcional y de calidad.


\newpage{}
\section{Trabajo futuro}
El proyecto ha sentado las bases para un sistema de monitorización y análisis de
datos robusto y escalable. Sin embargo, existen áreas de mejora y ampliación que
podrían ser abordadas en futuras iteraciones.


\subsection{Historias de usuario restantes}
Lo primero de todo sería completar las historias de usuario menos críticas que
han quedado pendientes, como la ingesta de datos de APIs externas o el
\textit{web scraping}. Estas funcionalidades permitirían enriquecer los datos
disponibles y ampliar las fuentes de información.

\begin{table}[H]
	\centering
	\begin{tabular}{|p{0.7\linewidth}|c|c|}
		\hline
		\textbf{Nombre} & \textbf{Prioridad} & \textbf{Tamaño} \\
		\hline
		\hline
		Como desarrollador de Okticket, quiero que los datos contengan metadatos que faciliten su filtrado o búsqueda & P2\cellcolor{yellow!50} & XS\cellcolor{blue!25} \\
		\hline
		Como trabajador de Okticket, quiero poder ver y consultar datos de empresas cliente & P2\cellcolor{yellow!50} & S\cellcolor{green!25} \\
		\hline
		Como gestor de una empresa cliente, quiero poder ver información relevante sobre mi empresa que recoja Okticket & P2\cellcolor{yellow!50} & M\cellcolor{yellow!50} \\
		\hline
		Como desarrollador de Okticket, quiero poder ingestar datos de APIs externas a la empresa & P2\cellcolor{yellow!50} & M\cellcolor{yellow!50} \\
		\hline
		Como desarrollador de Okticket, quiero poder ingestar información de páginas web externas (\textit{scraping}) & P2\cellcolor{yellow!50} & S\cellcolor{green!25} \\
		\hline
	\end{tabular}
	\caption{Historias de usuario restantes para futuras iteraciones}
	\label{tab:remaining_tasks}
\end{table}

Como se puede observar en la tabla \ref{tab:remaining_tasks}, las historias de
usuario restantes son las menos prioritarias (siguiendo con
\fullref{sec:planif_inicial}).

\newpage{}
\subsection{Integración de lenguaje natural para búsqueda (DSL)}
Sería interesante explorar la posibilidad de integrar el sistema con
un sistema de búsqueda y filtrado de datos a través de lenguaje natural, como
\textit{Natural Language Processing} (NLP)\footnote{
	\url{https://www.elastic.co/guide/en/elasticsearch/reference/current/query-dsl-query-string-query.html}
}

Esto permitiría a los usuarios realizar consultas de manera más intuitiva y
eficiente, sin necesidad de conocer la sintaxis de Kibana Query Language (KQL).


\subsection{Aplicación de modelos de Lenguaje de Gran Escala (LLM)}
Desde la empresa, se ha propuesto la posibilidad de aplicar modelos de LLM
para la generación de texto automática, presumiblemente de manera agnóstica al
proveedor (OpenAI, Anthropic, Google\ldots).

Esto permitiría la generación de informes y análisis de manera automática a
partir de los datos recopilados, facilitando la toma de decisiones y la
comunicación de insights a los usuarios.

\subsection{Más perspectivas futuras}
Gracias a la flexibilidad del stack ELK, se podrían añadir nuevas fuentes de
datos y visualizaciones, como logs de otras partes de la aplicación (gestor web,
otros servicios internos como Hubspot o Holded, aplicación móvil\ldots) o
visualizaciones más avanzadas y personalizadas.

Lo bueno de haber diseñado la arquitectura de la manera que se ha hecho
es que se pueden añadir nuevas funcionalidades sin necesidad de modificar
la infraestructura existente, simplemente añadiendo nuevos servicios y
configuraciones.

La escalabilidad del sistema también es un punto a tener en cuenta. En caso de
necesitar más capacidad de procesamiento o almacenamiento, se podría establecer
un sistema de autoescalabilidad en base a las definiciones ya realizadas.

Por último, se podría explotar la funcionalidad del stack de generar alertas
en base a la información ingestada y procesada, para notificar a los usuarios
de eventos críticos o anomalías detectadas en los datos.


\newpage{}
\section{Conclusiones y retrospectiva}
El desarrollo de este proyecto me ha permitido, a nivel personal, adquirir
conocimientos y habilidades muy útiles en áreas que tienen una gran demanda en
el mercado laboral actual, como la ingesta y procesamiento de datos, la
orquestación de servicios en la nube y la experiencia con proveedores cloud como
AWS.

A nivel técnico, he aprendido a trabajar con tecnologías modernas y potentes,
como Kafka, Logstash, Elasticsearch y Kibana, que me han permitido implementar
una solución robusta y escalable para la monitorización y análisis de datos.

Además, he tenido la oportunidad de trabajar con metodologías ágiles y de
gestión de proyectos, como Scrum, en proyectos reales con repercusiones y
resultados tangibles.

En cuanto a la retrospectiva del proyecto, considero que el tiempo dedicado ha
merecido la pena, pero podría haber sido gestionado u organizado diferente.
