\section{Planificación inicial}\label{sec:planif_inicial}
Como se ha mencionado anteriormente, se utiliza la metodología \textit{Scrum}
para la planificación y desarrollo del proyecto. En la figura \ref{fig:backlog}
se puede ver el \textit{backlog} de tareas que se planifican en el proyecto.

\begin{figure}[H]
	\centering
	\includegraphics[width=\textwidth]{planif/backlog.png}
	\caption{Planificación inicial del proyecto}
	\label{fig:backlog}
\end{figure}

Las historias de usuario anteriores se clasifican y categorizan según su
prioridad y tamaño, haciendo uso de la estrategia de tallas de camiseta como
mencionado anteriormente. En el tablero \textit{Kanban}
(ver figura \ref{fig:kanban}) se puede ver en todo momento el estado de las HU,
su progreso y sus características. El listado inicial (ordenado según su prioridad)
es el siguiente:

\begin{table}[H]
	\centering
	\begin{tabular}{|p{0.7\linewidth}|c|c|}
		\hline
		\textbf{Nombre} & \textbf{Prioridad} & \textbf{Tamaño} \\
		\hline
		\hline
		Creación de la infraestructura base (técnica) & P0\cellcolor{red!50} & L\cellcolor{orange!50} \\
		\hline
		Como desarrollador de Okticket, quiero que la arquitectura se despliegue y orqueste de manera automática & P0\cellcolor{red!50} & XL\cellcolor{red!50} \\
		\hline
		Como desarrollador de Okticket, quiero que se ingesten de manera automática datos de la base de datos interna de MongoDB & P0\cellcolor{red!50} & M\cellcolor{yellow!50} \\
		\hline
		Como desarrollador de Okticket, quiero que se ingesten de manera automática datos de la base de datos interna de MySQL & P0\cellcolor{red!50} & M\cellcolor{yellow!50} \\
		\hline
		Como desarrollador de Okticket, quiero que los datos se limpien de manera automática & P0\cellcolor{red!50} & S\cellcolor{green!25} \\
		\hline
		Como trabajador de Okticket, quiero poder ver y consultar datos internos de la empresa & P1\cellcolor{orange!50} & M\cellcolor{yellow!50} \\
		\hline
		Como desarrollador de Okticket, quiero que se ingesten de manera automática logs de balanceador de AWS & P1\cellcolor{orange!50} & M\cellcolor{yellow!50} \\
		\hline
		Como desarrollador de Okticket, quiero poder ver el estado general de la infraestructura & P1\cellcolor{orange!50} & M\cellcolor{yellow!50} \\
		\hline
		Como desarrollador de Okticket, quiero que los datos contengan metadatos que faciliten su filtrado o búsqueda & P2\cellcolor{yellow!50} & XS\cellcolor{blue!25} \\
		\hline
		Como trabajador de Okticket, quiero poder ver y consultar datos de empresas cliente & P2\cellcolor{yellow!50} & S\cellcolor{green!25} \\
		\hline
		Como gestor de una empresa cliente, quiero poder ver información relevante sobre mi empresa que recoja Okticket & P2\cellcolor{yellow!50} & M\cellcolor{yellow!50} \\
		\hline
		Como desarrollador de Okticket, quiero poder ingestar datos de APIs externas a la empresa & P2\cellcolor{yellow!50} & M\cellcolor{yellow!50} \\
		\hline
		Como desarrollador de Okticket, quiero poder ingestar información de páginas web externas (\textit{scraping}) & P2\cellcolor{yellow!50} & S\cellcolor{green!25} \\
		\hline
	\end{tabular}
	\caption{Historias de usuario iniciales}
	\label{tab:initial_tasks}
\end{table}

Los puntos de historia de usuario (PHU) se calculan aplicando la secuencia de
Fibonacci como comentado anteriormente, donde $XS=1$, $S=2$, $M=3$, $L=5$ y
$XL=8$.

Debido a la estimación anterior, el valor total de los puntos de historia de
usuario es de 41. El valor total de los puntos de historia necesarios para
alcanzar el mínimo producto viable (MVP) es de 30, al no tener en cuenta las
historias de usuario de prioridad 2.
