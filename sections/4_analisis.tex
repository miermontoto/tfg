\chapter{Análisis}\label{chap:analisis}
Este capítulo se centra en desglosar los componentes críticos del proyecto, específicamente
dirigido a entender las necesidades de Okticket y cómo el desarrollo propuesto se alinea con estas.
Se analizarán los requisitos funcionales y no funcionales, evaluando cómo cada uno contribuye al
éxito del proyecto. Además, se identificarán las partes interesadas clave y se explorarán sus
expectativas y requisitos, para asegurar que el sistema desarrollado cumpla con sus necesidades
específicas. Este análisis detallado tiene como objetivo final proporcionar una hoja de ruta
clara para el desarrollo del proyecto, asegurando que se tomen decisiones informadas que maximicen
el valor entregado a la empresa y sus clientes.

\section{Partes interesadas (stakeholders)}\label{sec:stakeholders}
Las partes interesadas en el proyecto son aquellas personas o entidades que tienen un interés
en el mismo, ya sea porque se ven afectadas por el resultado del proyecto, o porque tienen
algún tipo de interés en el mismo. Las partes interesadas en este proyecto son las siguientes:

\begin{enumerate}
	\item \textbf{Okticket}: la empresa es la principal parte interesada en el proyecto, ya que
		es la que se beneficiará directamente de los resultados del mismo, así como de las
		oportunidades de negocio que se abren con la explotación de los datos. Dentro de la empresa,
		se pueden identificar dos entidades:
		\begin{itemize}
			\item \textbf{Equipo de desarrollo de la empresa}: el equipo de desarrollo es otra parte
				interesada en el proyecto, ya que son los encargados de llevar a cabo la implementación
				del sistema y de garantizar su correcto funcionamiento, además de gestionar el soporte
				de servicio a nivel técnico.
			\item \textbf{Equipo de soporte de la empresa}: el sistema planteado ahorraría tiempo al equipo de
				soporte, ya que les permitiría analizar los datos de forma más eficiente e identificar
				problemas antes de que tener que resolver las peticiones de los clientes afectados a
				nivel básico.
		\end{itemize}
	\item \textbf{Clientes}: los clientes de la empresa también son partes interesadas, puesto
		que se beneficiarán de los nuevos servicios que se ofrecen, como los dashboards de
		negocio que se han descrito anteriormente.
	\item \textbf{Investigador y desarrollador (\emph{\author}):} el desarrollador del
		proyecto tiene la oportunidad de aplicar los conocimientos adquiridos en el desarrollo de un
		proyecto real, y de adquirir nuevos conocimientos en el proceso.
\end{enumerate}

\section{Valoración de alternativas}\label{sec:alternativas}


\section{Definición del sistema}\label{sec:definicion}
