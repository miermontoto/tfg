\section{Visualización de datos}\label{sec:impl_visualizacion}
Una vez se cuentan con datos en Elasticsearch, se puede comenzar el desarrollo
de la visualización de los mismos mediante Kibana. Para ello, se han desarrollado
paneles de visualización que permiten la monitorización de los datos en tiempo
real y la creación de informes y \textit{dashboards} personalizados para cada
modelo de datos contemplado.

Esta sección de la memoria documenta el desarrollo de las siguientes historias
de usuario, siguiendo la planificación establecida en la sección \fullref{sec:planif_inicial}:

\begin{table}[H]
	\centering
	\begin{tabular}{|p{0.7\linewidth}|c|c|}
		\hline
		\textbf{Nombre} & \textbf{Prioridad} & \textbf{Tamaño} \\
		\hline
		\hline
		Como trabajador de Okticket, quiero poder ver y consultar datos internos de la empresa & P1\cellcolor{orange!50} & L\cellcolor{orange!50} \\
		\hline
		Como desarrollador de Okticket, quiero poder ver el estado general de la infraestructura & P1\cellcolor{orange!50} & L\cellcolor{orange!50} \\
		\hline
  \end{tabular}
  \caption{Lista de HUs cumplimentadas con la visualización de datos}
  \label{tab:impl_visualizacion}
\end{table}
